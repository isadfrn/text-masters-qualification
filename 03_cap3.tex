% ----------------------------------------------------------------------- %
% Arquivo: cap3.tex
% ----------------------------------------------------------------------- %
\chapter{Trabalhos Relacionados}
\label{c_cap3}

Este capítulo tem o objetivo demonstrar o estudo sobre trabalhos correlatos que possuem um grau de similaridade com o tema da dissertação. O capítulo deve identificar as contribuições de cada trabalho e depois comparar com a solução proposta, demonstrando o diferencial da dissertação. Para tal, é importante:

\begin{itemize}
    \item Definir critérios para a busca e para a análise dos trabalhos correlatos;
    
    \item Efetuar a análise de cada trabalho com base nos critérios definidos;
    
    \item Fazer uma análise comparativa (usualmente por meio de um quadro) entre os trabalhos, adicionando também a sua proposta.
\end{itemize}

Dessa forma, você poderá demonstrar como sua proposta se diferencia das demais, caracterizando a contribuição do trabalho.

Dependendo da cobertura da revisão realizada e da atualidade dos trabalhos analisados, este capítulo pode ser denominado Estado da Arte. Contudo, essa decisão deve ser tomada em comum acordo com o seu orientador.

É recomendado que a descrição do protocolo de busca e análise seja feita em um apêndice da dissertação, sendo que o nível de detalhamento a ser apresentando deve ser definido com o orientador.


% ----------------------------------------------------------------------- %
% ----------------------------------------------------------------------- %
\section{Trabalho 1 (ANO, exemplo: 2015)}
\label{s_c3_trabalho-1}

% ----------------------------------------------------------------------- %
% ----------------------------------------------------------------------- %
\section{Trabalho 2 (2017)}
\label{s_c3_trabalho-2}

Descrição e crítica do trabalho relacionado.






% ----------------------------------------------------------------------- %
% ----------------------------------------------------------------------- %
\section{Comparação dos Trabalhos Relacionados}
\label{s_c3_comparacao}

Nesta seção, apresente a análise comparativa dos trabalhos selecionados caracterizando-os em um quadro (conforme exemplificado no \autoref{q_c3-trabalhos}) e posicionando o seu trabalho em relação a eles (na última coluna). Dependendo da quantidade de trabalhos analisados, pode ser necessário elaborar mais de um quadro ou definir uma seção com página no formato paisagem.

Mesmo que você ainda não tenha apresentado os resultados da sua dissertação no texto, isso conduzirá o restante da leitura, permitindo que você indique a contribuição de cada um dos trabalhos para a solução do problema de pesquisa da sua dissertação e evidencie a sua contribuição, e o diferencial do seu trabalho em relação aos demais. Isso deve ser feito por meio de um texto de discussão apoiado pelos dados apresentados no quadro. 

Dica: utilizar o site ``\url{http://www.tablesgenerator.com/}'' para auxiliar na confecção dos quadros e tabelas.

\begin{quadro}[!htbp]
 \caption{Comparação dos trabalhos relacionados} 
 \begin{center}
 \begin{footnotesize} 
 \label{q_c3-trabalhos} 
 
\resizebox{\textwidth}{!}{%
\begin{tabular}{|c|c|c|c|c|c|c|c|c|}
\hline
\rowcolor[HTML]{EFEFEF} 
{\color[HTML]{000000} \rotatebox[origin=c]{90}{\textbf{ Artigo }}} & {\color[HTML]{000000} \rotatebox[origin=c]{90}{\textbf{ Coluna 1 }}} & {\color[HTML]{000000} \rotatebox[origin=c]{90}{\textbf{ Coluna 2 }}} & {\color[HTML]{000000} \rotatebox[origin=c]{90}{\textbf{ Coluna 3 }}} & {\color[HTML]{000000} \rotatebox[origin=c]{90}{\textbf{ Coluna 4 }}} & {\color[HTML]{000000} \rotatebox[origin=c]{90}{\textbf{ Coluna 5 }}} & {\color[HTML]{000000} \rotatebox[origin=c]{90}{\textbf{ Coluna 6 }}} & {\color[HTML]{000000} \rotatebox[origin=c]{90}{\textbf{ Coluna 7 }}} & {\color[HTML]{000000} \rotatebox[origin=c]{90}{\textbf{ Coluna 8 }}} \\ \hline

\rowcolor[HTML]{C0C0C0} 
\textbf{\begin{tabular}[c]{@{}c@{}}MARTENS\\ (2010)\end{tabular}} & texto & texto & texto & \begin{tabular}[c]{@{}c@{}}quebra\\ linha\end{tabular} & texto & texto & texto & texto \\ \hline

\rowcolor[HTML]{EFEFEF} 
\textbf{\begin{tabular}[c]{@{}c@{}}EN-NASRY\\ E KETTANI \\ (2011)\end{tabular}} & texto & texto & texto & texto & texto & texto & texto & texto \\ \hline

\rowcolor[HTML]{C0C0C0} 
{\color[HTML]{000000} \textbf{\begin{tabular}[c]{@{}c@{}}BICAKCI\\ (2014)\end{tabular}}} & {\color[HTML]{000000} texto} & {\color[HTML]{000000} texto} & {\color[HTML]{000000} texto} & {\color[HTML]{000000} texto} & {\color[HTML]{000000} texto} & {\color[HTML]{000000} texto} & {\color[HTML]{000000} texto} & {\color[HTML]{000000} texto} \\ \hline

\rowcolor[HTML]{EFEFEF} 
\textbf{\begin{tabular}[c]{@{}c@{}}ESTE\\ TRABALHO\end{tabular}} & texto & texto & texto & texto & texto & texto & texto & texto \\ \hline

 \end{tabular}%
 } 
 \end{footnotesize}
 \end{center} 
 \raggedright Fonte: Elaborado pelo próprio autor. 
\end{quadro}


As publicações tratam disso e daquilo, suas contribuições são tal, mas não estão previstas tais coisas...





% ----------------------------------------------------------------------- %
% ----------------------------------------------------------------------- %
\section{Considerações}
\label{s_c3_consideracoes}

Este capítulo pode ter uma última seção como esta denominada ``Considerações'' ou ``Discussão'' fazendo uma ligação entre a análise realizada sobre os trabalhos relacionados e o próximo capítulo, no qual você descreverá a sua contribuição.