% ----------------------------------------------------------------------- %
% Arquivo: cap3.tex
% ----------------------------------------------------------------------- %
\chapter{Trabalhos Relacionados}
\label{c_cap3}

Neste capítulo são apresentados os trabalhos relacionados que forneceram a base teórica para o desenvolvimento desta pesquisa. O objetivo é identificar o estado da arte atual, sobre programas de capacitação em cibersegurança voltados para o público feminino, bem como as estratégias de retenção e engajamento utilizadas. A partir desta análise, busca-se evidenciar as lacunas que este trabalho pretende preencher ao investigar o cenário do programa Hackers do Bem. O restante do capítulo descreve a metodologia utilizada para a seleção das obras e discute os principais achados.

\section{Metodologia}
\label{s_metodologia}

Para a condução deste estudo, adotou-se uma abordagem metodológica baseada em uma Revisão Sistemática da Literatura (RSL). O processo de seleção e análise dos trabalhos seguiu as diretrizes do método PRISMA (Preferred Reporting Items for Systematic Reviews and Meta-Analyses), a fim de garantir a transparência, a qualidade e a replicabilidade da pesquisa bibliográfica.

As etapas estabelecidas para a revisão incluíram a definição do protocolo de busca, a identificação das fontes de informação, a aplicação de critérios de inclusão e exclusão, e a seleção dos estudos relevantes. Visando a extração de dados qualitativos e quantitativos, bem como, sorver de literatura para a fundamentação teórica.

\begin{table}[htb]
\centering
\ABNTEXfontereduzida
\IBGEtab{%
  \caption{Processo de seleção dos artigos}
  \label{tab:selecao_artigos}
}{%
  \begin{tabular}{l|c|c|c}
  \toprule
  \textbf{Base de dados} & \textbf{1ª seleção} & \textbf{2ª seleção} & \textbf{Seleção final} \\
  \midrule
Google Scholar & 50  & 29 & XX \\
Scopus         & 54  & 24 & XX \\
IEEE Xplore    & 342 & 32 & XX \\
  \midrule
  \textbf{Total} & \textbf{446} & \textbf{85} & \textbf{85} \\
  \bottomrule
  \end{tabular}%
}{%
  \fonte{Elaborada pela autora.}
}
\end{table}


% ----------------------------------------------------------------------- %
% ----------------------------------------------------------------------- %
\section{Trabalho 1 (2015)}
\label{s_c3_trabalho-1}

% ----------------------------------------------------------------------- %
% ----------------------------------------------------------------------- %
\section{Trabalho 2 (2017)}
\label{s_c3_trabalho-2}


% ----------------------------------------------------------------------- %
% ----------------------------------------------------------------------- %
\section{Comparação dos Trabalhos Relacionados}
\label{s_c3_comparacao}

% ----------------------------------------------------------------------- %
% ----------------------------------------------------------------------- %
\section{Considerações}
\label{s_c3_consideracoes}
