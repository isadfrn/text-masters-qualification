%% abtex2-modelo-trabalho-academico.tex, v-1.9 laurocesar
%% Copyright 2012-2013 by abnTeX2 group at http://abntex2.googlecode.com/ 
%%
%% This work may be distributed and/or modified under the
%% conditions of the LaTeX Project Public License, either version 1.3
%% of this license or (at your option) any later version.
%% The latest version of this license is in
%%   http://www.latex-project.org/lppl.txt
%% and version 1.3 or later is part of all distributions of LaTeX
%% version 2005/12/01 or later.
%%
%% This work has the LPPL maintenance status `maintained'.
%% 
%% The Current Maintainer of this work is the abnTeX2 team, led
%% by Lauro César Araujo. Further information are available on 
%% http://abntex2.googlecode.com/
%%
%% This work consists of the files abntex2-modelo-trabalho-academico.tex,
%% abntex2-modelo-include-comandos and abntex2-modelo-references.bib
%%

% ------------------------------------------------------------------------
% ------------------------------------------------------------------------
% abnTeX2: Modelo de Trabalho Academico (tese de doutorado, dissertacao de
% mestrado e trabalhos monograficos em geral) em conformidade com 
% ABNT NBR 14724:2011: Informacao e documentacao - Trabalhos academicos -
% Apresentacao
% ------------------------------------------------------------------------
% ------------------------------------------------------------------------

\documentclass[
	% -- opções da classe memoir --
	12pt,				% tamanho da fonte
	openany,			% capítulos começam em pág ímpar (insere página vazia caso preciso) - openright ou openany
	oneside,			% para impressão em verso e anverso. Oposto a oneside
	a4paper,			% tamanho do papel. 
	% -- opções da classe abntex2 --
	%chapter=TITLE,		% títulos de capítulos convertidos em letras maiúsculas
	%section=TITLE,		% títulos de seções convertidos em letras maiúsculas
	%subsection=TITLE,	% títulos de subseções convertidos em letras maiúsculas
	%subsubsection=TITLE,% títulos de subsubseções convertidos em letras maiúsculas
	% -- opções do pacote babel --
	english,			% idioma adicional para hifenização
	french,				% idioma adicional para hifenização
	spanish,			% idioma adicional para hifenização
	brazil				% o último idioma é o principal do documento
	]{styles/abntex2}

% ---
% PACOTES
% ---

% ---
% Pacotes fundamentais 
% ---
\usepackage{lmodern}			% Usa a fonte Latin Modern			
\usepackage[T1]{fontenc}		% Selecao de codigos de fonte.
\usepackage[utf8]{inputenc}		% Codificacao do documento (conversão automática dos acentos)
\usepackage{lastpage}			% Usado pela Ficha catalográfica
\usepackage{indentfirst}		% Indenta o primeiro parágrafo de cada seção.
\usepackage{color}				% Controle das cores
\usepackage{graphicx}			% Inclusão de gráficos
\usepackage[export]{adjustbox}  % Permite usar a opção frame ou fbox, para criar margem nas figuras
\usepackage[singlelinecheck=false]{caption} % Manter caption à esquerda
\usepackage{microtype} 			% para melhorias de justificação
\usepackage{rotating}           %rotação de texto
%\usepackage[landscape]{geometry}% http://ctan.org/pkg/geometry
\usepackage{array}              % http://ctan.org/pkg/array
\usepackage[table,xcdraw]{xcolor}
\usepackage[printonlyused]{styles/acronimos-extendido}
\usepackage{textcase}
\usepackage{caption}
\usepackage{subcaption}
\usepackage{multirow}
\usepackage{booktabs}
\usepackage{pifont}
\usepackage{tikz} \usetikzlibrary{shapes,arrows,positioning}
% Utilizado para inserir arquivos PDF no documento
\RequirePackage{pdflscape}
\RequirePackage{pdfpages}
% EXEMPLO: \includepdf[pages=1-2,scale=1]{1st_texts/lista_quadros.pdf}

% Utilizado para inserir quadros no documento
%\usepackage{trivfloat}
%\trivfloat{quadro}
% Substituído comando acima por entradas no arquivo "abntex2.cls". Referência de comandos tiradas de: https://github.com/abntex/abntex2/wiki/HowToCriarNovoAmbienteListing

% ---

% Comando para girar texto da tabela em 90 graus
\newcommand*\giranov{\rotatebox{90}}	

% Comando para inserir um texto sobrescrito
\newcommand{\ts}{\textsuperscript}

% Comandos para auxiliar na revisão:
%\usepackage{cancel,soul,ulem}
\usepackage[normalem]{ulem}
\usepackage{color,pgf}
\usepackage{verbatim}
\usepackage{float}
\usepackage[brazilian]{babel}
\usepackage[style=brazilian]{csquotes}

\newcommand{\excluir}[2]{
\textcolor{red}{\textbf{#1: }\sout{#2}}
}

\newcommand{\incluir}[2]{
\textcolor{green}{\textbf{#1: }\textbf{#2}}
}

% Pacote para inserir TODO, para correção
\usepackage[colorinlistoftodos,prependcaption,textsize=tiny]{todonotes}

% ---
% Pacotes de citações
% ---
% Citações padrão ABNT (algumas opções):
% alf: lista autores em ordem alfabética
% abnt-emphasize=bf: coloca o título das referências em negrito. Se tirar, fica em itálico.

% \usepackage[alf,myoptions]{styles/abntex2cite} % Citações padrão ABNT
\usepackage[alf,abnt-etal-list=0,abnt-etal-cite=3,abnt-repeated-author-omit=yes,abnt-doi=expand,abnt-emphasize=bf]{styles/abntex2cite}	

% Pacote de Formatação da Univali
\usepackage{styles/univali_custom}

% ---
% Informações de dados para CAPA e FOLHA DE ROSTO
% ---
\titulo{Um \textit{Framework} de Diretrizes para ampliar o ingresso e a permanência de talentos femininos em Cibersegurança: Estudo de caso do programa \textit{Hackers} do Bem}
% \subtitulo{subtítulo se houver}
\autor{Isabella de Freitas Nunes}
\local{Itajaí (SC)}
\mes{Janeiro}
\ano{2026}
\orientador{Michelle Silva Wangham, Dra}
%\coorientador{Coorientador}
\instituicao{UNIVERSIDADE DO VALE DO ITAJAÍ\par
PRÓ-REITORIA DE PÓS-GRADUAÇÃO,\\ 
PESQUISA, EXTENSÃO E CULTURA\par
PROGRAMA DE MESTRADO ACADÊMICO EM\\ 
COMPUTAÇÃO APLICADA}

\tipotrabalho{Qualificação de Dissertação (Mestrado)}
% O preambulo deve conter o tipo do trabalho, o objetivo, 
% o nome da instituição e a área de concentração 
\preambulo{Dissertação apresentada como requisito parcial à obtenção do grau de Mestre em Computação Aplicada.}
% ---

% ---
% Informações para Resumo e Abstract
% ---
\linhadepesquisa{Tecnologias Sociais e Educacionais}
\researchline{Social and Educational Technologies}
%\numpaginas{175}
\palavraschave{educação em cibersegurança, gap de gênero, framework}
\keywords{cybersecurity education, gender gap, and framework}
\englishmonth{February}
\titleenglish{A Framework of Guidelines to Enhance the Entry and Retention of Female Talent in Cybersecurity: A Case Study of the Hackers do Bem Program}
%\subtitleenglish{A User-Centric Solution for e-Government}
% ---

% ---
% Configurações de aparência do PDF final

% informações do PDF
\makeatletter
\hypersetup{
     	%pagebackref=true,
		pdftitle={\@title}, 
		pdfauthor={\@author},
    	pdfsubject={\imprimirpreambulo},
	    pdfcreator={LaTeX with abnTeX2},
		pdfkeywords={trabalho acadêmico}, 
		colorlinks=true,       		% false: boxed links; true: colored links
    	linkcolor=black,          	% color of internal links
    	citecolor=black,        		% color of links to bibliography
    	filecolor=black,      		% color of file links
		urlcolor=black,
		bookmarksdepth=4
}
\makeatother
% --- 

% ---
% compila o indice
% ---
\makeindex

% ----
% Início do documento
% ----
\begin{document}

% Retira espaço extra obsoleto entre as frases.
\frenchspacing 

% ----------------------------------------------------------
% ELEMENTOS PRÉ-TEXTUAIS
% ----------------------------------------------------------
% \pretextual

% ---
% Capa
% ---
\imprimircapa
% ---

% ---
% Folha de rosto
% (o * indica que haverá a ficha bibliográfica)
% ---
\imprimirfolhaderosto*
% ---

% ---
% Inserir a ficha bibliografica
% ---
%\include{ficha_catalografica}
% ---

% ---
% Inserir folha de aprovação
% ---

% Isto é um exemplo de Folha de aprovação, elemento obrigatório da NBR
% 14724/2011 (seção 4.2.1.3). Você pode utilizar este modelo até a aprovação
% do trabalho. Após isso, substitua todo o conteúdo deste arquivo por uma
% imagem da página assinada pela banca com o comando abaixo:
%
% \includepdf{folhadeaprovacao_final.pdf}
%
%\include{1st_texts/folha_de_aprovacao}
%\begin{dedicatoria}
    \vspace*{\fill}
	\begin{flushright}
		\textit{}
	\end{flushright}
\end{dedicatoria}
%\begin{epigrafe}
    \vspace*{\fill}
	\begin{flushright}
		\textit{``Mensagem''}		
	\end{flushright}
\end{epigrafe}
%\begin{agradecimentos}

Texto destinado aos agradecimento.. vai só na versão final, não na qualificação!

\end{agradecimentos}
\begin{resumo}
    A demanda por profissionais em cibersegurança cresce, mas a baixa participação feminina no setor permanece um problema. Diversos relatórios mostram essa disparidade em âmbito internacional e nacional. Dentre as ações criadas a fim de aumentar o número de profissionais capacitados no mercado brasileiro, destaca-se o programa \textit{Hackers} do Bem busca capacitar profissionais em larga escala, usando métodos de gamificação e rankings. Este estudo examina se a ênfase em competição e ranqueamento pode atuar como uma barreira involuntária à continuidade da formação de mulheres em programas deste tipo. O objetivo é propor um conjunto de diretrizes para melhorar a inclusão e a retenção, baseado em uma análise quantitativa e qualitativa da participação feminina, usando o programa \textit{Hackers} do Bem como estudo de caso. A metodologia utiliza métodos mistos: revisão sistemática da literatura, mineração de dados educacionais para calcular taxas de evasão e eficiência em cada etapa do curso, e questionários de autoeficácia para entender fatores subjetivos da desistência. Como resultados, espera-se identificar gargalos de retenção por gênero, correlacionar pontos de experiência com desempenho e fornecer bases para possíveis ajustes na gamificação.
\end{resumo}
\begin{abstract}
 \begin{otherlanguage*}{english}
    The growing demand for cybersecurity professionals and the establishment of the National Cybersecurity Strategy (E-Ciber) have driven the creation of large-scale training programs in Brazil. However, the persistent underrepresentation of women in the sector, stagnant at around 25\% globally, challenges the effectiveness of these initiatives in promoting diversity. This dissertation investigates the leaky pipeline phenomenon in the context of the Hackers do Bem Program, analysing whether competitive gamification and ranking mechanics, typical of massive training initiatives, act as barriers to retention for women. The research adopts a mixed-methods approach, combining educational data mining to map critical dropout points in the training funnel with qualitative analysis of participants' self-efficacy perceptions. As a result, the development and validation of a \textbf{Framework of Guidelines for Inclusion and Retention} is proposed. This artefact is structured around three strategic axes: Narrative Contextualization, Inclusive Assessment Mechanics, and Scalable Social Support, aiming to reconcile the technical efficiency of mass training with the pedagogical practices needed to broaden the entry and retention of female talent.
 \end{otherlanguage*}
\end{abstract}
\include{frontmatter/lista_figuras}
\include{frontmatter/lista_tabelas}
%\includepdf[pages=1,scale=1]{_template/lista_quadros.pdf}
% ---
% inserir lista de quadros
% ---
%\newpage
%\phantomsection
\pdfbookmark[0]{\listofquadrosname}{loq}
\listofquadros*
\cleardoublepage
% ---


%\begin{siglas}
\item[6LoWPAN]{\textit{IPv6 over Low Power Wireless Personal Area Networks}}
\item[ABAC]{\textit{Attribute Based Access Control}}	 
\item[AC]{Autoridade Certificadora}
\item[ACL]{\textit{Access Control List}}
\item[ACM]{\textit{Access Control Mechanism}}
\item[AES]{\textit{Advanced Encryption Standard}}
\item[API]{\textit{Application Programming Interface}}
\item[AS]{\textit{Authorized Server}}
\item[BPEL]{\textit{Business Process Execution Language}}
\item[DAC]{\textit{Discretionary Access Control}}	
\item[DoS]{\textit{Denial of Service}}	
\item[DTLS]{\textit{Datagram Transport Layer Security}}
\item[EAP]{\textit{Extensible Authentication Protocol}}	
\item[ECC]{\textit{Elliptic Curve Cryptography}}
\item[ECCDH]{\textit{ECC Diffie-Hellman}}
\item[ECP]{\textit{Enhanced Client or Proxy}}
\item[EnHANTs]{\textit{Energy-Harvesting Active Networked Tags}}
\item[GID]{Gestão de Identidades Digitais}
\item[HTML]{\textit{Hypertext Markup Language}}
\item[HTTP]{\textit{Hypertext Transfer Protocol}}
\item[IAA]{Infraestrutura de Autenticação e Autorização}
\item[IdM]{\textit{Identity Management}}
\item[IdP]{\textit{Identity Provider}}
\item[IoT]{\textit{Internet of Things}}
\item[IP]{\textit{Internet Protocol}}
\item[IPv4]{\textit{Internet Protocol version 4}}
\item[IPv6]{\textit{Internet Protocol version 6}}
\item[JSON]{\textit{JavaScript Object Notation}}
\item[KDC]{\textit{Key Distribution Center}}
\item[LDAP]{\textit{Lightweight Directory Access Protocol}}	
\item[M2M]{\textit{Machine to Machine}}
\item[MAC]{\textit{Mandatory Access Control}}
\item[MITM]{\textit{Man-In-The-Middle}}
\item[NFC]{\textit{Near-Field Communication}}
\item[PAP]{\textit{Policy Administration Point}}
\item[PBAC]{\textit{Policy-Based Access Control}}
\item[PDP]{\textit{Policy Decision Point}}
\item[PEP]{\textit{Policy Enforcement Point}}
\item[PIP]{\textit{Policy Information Point}}
\item[RAM]{\textit{Random Access Memory}}
\item[RBAC]{\textit{Role Based Access Control}}
\item[REST]{\textit{REpresentational State Transfer}}
\item[RFID]{\textit{Radio-Frequency IDentification}}
\item[ROA]{\textit{Resource Oriented Architecture}}
\item[ROM]{\textit{Read-only Memory}}
\item[RSN]{\textit{RFID Sensor Network}}
\item[RSSF]{Redes de Sensores Sem Fio}
\item[SGML]{\textit{Standard Generalized Markup Language}}
\item[SAML]{\textit{Security Assertion Markup Language}}
\item[SP]{\textit{Service Provider}}
\item[SSO]{\textit{Single Sign-On}}
\item[TLS]{\textit{Transport Layer Security}}
\item[TPM]{\textit{Trusted Platform Module}}
\item[UDDI]{\textit{Universal Description, Discovery and Integration}}
\item[URI]{\textit{Uniform Resource Identifier}}
\item[URL]{\textit{Uniform Resource Locator}}
\item[WoT]{\textit{Web of Things}}
\item[WABAC]{\textit{Workflow-oriented Attributed Based Access Control}}
\item[WSAN]{\textit{Wireless Sensor and Actuator Network}}
\item[WSDL]{\textit{Web Services Description Language}}
\item[WSN]{\textit{Wireless Sensor Network}}
\item[XML]{\textit{eXtensible Markup Language}}
\item[XACML]{\textit{eXtensible Access Control Markup Language}}	
\end{siglas}
\begin{simbolos}
  \item[$ \Gamma $] Letra grega Gama
  \item[$ \Lambda $] Letra grega Lambda
  \item[$ \zeta $] Letra grega minúscula zeta
  \item[$ \in $] Pertence
\end{simbolos}
%% Como usar o pacote acronym


% Na primeira vez que for citado o acronimo, o nome completo 
% irá aparecer seguido do acronimo entre parênteses. Na 
% proxima vez somente o acronimo irá aparecer. Se usou a 
% opção footnote no pacote, entao o nome por extenso irá 
% aparecer no rodapé \ac{acronimo}


% Para aparecer com nome completo + acronimo
% \acf{acronimo}

% Para aparecer somente o acronimo
% \acs{acronimo}

% Nome por extenso somente, sem o acronimo
% \acl{acronimo}

% igual o \ac mas deixando no plural com S (ingles)
% \acp{acronimo}

% \acfp{acronimo}

% \acsp{acronimo}

% \aclp{acronimo}

%% ATENCAO
% Criei o comando \acfe{}, resultando em: Extenso -- ACRO

\chapter*{Lista de Abreviaturas}%
% \addcontentsline{toc}{chapter}{Lista de abreviaturas}
\markboth{Lista de abreviaturas}{}


\begin{acronym}







%A
\acro{AA}{\textit{Attribute Authority}}
\acro{ABAC}{\textit{Attribute-Based Access Control}}
\acro{AC}{Autoridade Certificadora}
\acro{AD}{\textit{Active Directory}}
\acro{AES}{\textit{Advanced Encryption Standard}}
\acro{AP}{\textit{Attribute Provider}}
\acro{APIs}{\textit{Application Programming Interfaces}}
\acro{APP}{\textit{Application}}
\acro{ARC}{\textit{Advanced Resource Connector}}
\acro{AS}{\textit{Attribute Service}}
\acro{ASM}{\textit{Authenticator Abstraction Layer}}
\acro{ASP}{\textit{Application Service Provider}}
\acro{ATN}{\textit{Automated Trust Negotiation}}
%B
\acro{BLE}{\textit{Bluetooth Low Energy}}
%C
\acro{CA}{\textit{Certificate Authority}}
\acro{CN}{\textit{Collaborative Networks}}
\acro{CNH}{Carteira Nacional de Habilitação}
\acro{CNPJ}{Cadastro Nacional de Pessoa Jurídica}
\acro{CPF}{Cadastro de Pessoa Física}
\acro{CS}{\textit{Consent Service}}
\acro{CSR}{\textit{Certificate Signing Request}}
\acro{CPU}{\textit{Central Processing Unit}}
%D
\acro{DFN}{\textit{Deutsches Forschungsnetz}}
\acro{DN}{\textit{Distinguised Name}}
\acro{DS}{\textit{Discovery Service}}
\acro{DVO}{\textit{Dynamic Virtual Organization}}
%E
\acro{e-CPF}{Cadastro de Pessoal Física Eletrônico}
\acro{e-CNPJ}{Cadastro Nacional de Pessoa Jurídica Eletrônico}
\acro{e-Gov}{\textit{Electronic Government}}
\acro{e-PING}{Padrões de Interoperabilidade em Governo Eletrônico}
\acro{eID}{\textit{Electronic Identity}}
\acro{eIDMS}{\textit{electronic Identity Management System}}
\acro{EGDI}{\textit{e-Government Development Index}}
\acro{ePWG}{Padrões Web para Governo Eletrônico}
\acro{ETSI}{\textit{European Telecommunications Standards Institute}}
%F
\acro{FIDO}{\textit{Fast IDentity Online}}
\acro{FIM}{\textit{Federated Identity Management}}
\acro{FO}{\textit{Federation Operator}}
%G
\acro{G2C}{Governo para Cidadão}
\acro{G2E}{\textit{Government to Employees}}
\acro{G2G}{Governo para Governo}
\acro{GId}{Gestão de Identidade}
\acro{GSI}{\textit{Grid Security Infrastructure}}
\acro{GTTI}{Grupo de Trabalho Interministerial}
%H
\acro{HCE}{\textit{Host-based Card Emulation}}
\acro{HCI}{\textit{Human Capital Index}}
\acro{HPC}{\textit{High Performance Computing}}
\acro{HSM}{\textit{Hardware Security Module}}
\acro{HTTP}{\textit{Hypertext Transfer Protocol}}
%I
\acro{IAA}{\textit{Identity, Authentication, Authorization}}
\acro{IAF}{\textit{Identity Assurance Framework}}
\acro{IAM}{\textit{Identity and Access Management}}
\acro{IBAC}{\textit{Identity-Based Access Control}}
\acro{ICFF}{\textit{Intercloud Federation Framework}}
\acro{ICP}{Infraestrutura de Chave Pública}
\acro{Id}{\textit{Identity}}
\acro{ID}{Identificador}
\acro{IDE}{\textit{Integrated Development Environment}}
\acro{IdM}{\textit{Identity Management System}}
\acro{IdP}{\textit{Identity Provider}}
\acro{IGF}{\textit{Identity Governance Framework}}
\acro{IMSI}{\textit{International Mobile Subscriber Identity}}
\acro{IP}{\textit{Internet Protocol}}
\acro{IRPF}{Imposto de Renda de Pessoal Física}
%J
\acro{JSON}{\textit{JavaScript Object Notation}}
%K
%L
\acro{LDAP}{\textit{Lightweight Directory Access Protocol}}
\acro{LSDMA}{\textit{Large Scale Data Management}}
%M
\acro{MITM}{\textit{Man-in-The-Middle}}
\acro{myVOCS}{\textit{my Virtual Organization Collaboration System}}
%N
\acro{NFC}{\textit{Near Field Communication}}
\acro{NREN}{\textit{National Research Education Networks}}
%O
\acro{OASIS}{\textit{Organization for the Advancement of Structured Information Standards}}
\acro{OCSP}{\textit{Online Certificate Status Protocol}}
\acro{ONG}{Organizações Não Governamentais}
\acro{OSI}{\textit{Online Service Index}}
\acro{OTP}{\textit{One-time Password}}
\acro{OV}{Organização Virtual}
%P
\acro{PaaS}{\textit{Plataform-as-a-Service}}
\acro{PAPI}{\textit{Point of Access to Providers of Information}}
\acro{PC}{\textit{Professional Communities}}
\acro{PDP}{\textit{Policy Decision Point}}
\acro{PEP}{\textit{Policy Enforcement Point}}
\acro{PERMIS}{\textit{Privilege Managerment Infrastructure}}
\acro{PKI}{\textit{Public Key Infrastructure}}
\acro{PIC}{\textit{Personal Identification Code}}
\acro{PIN}{\textit{Personal Identification Number}}
\acro{PoA}{\textit{Point of Authentication}}
\acro{PPP}{Parceria Público-Privada}
\acro{PVC}{\textit{Professional Virtual Communities}}
%Q
%R
\acro{RA}{\textit{Registration Authority}}
\acro{RBAC}{\textit{Role-Based Access Control}}
\acro{RCN}{Registro Civil Nacional}
\acro{REE}{\textit{Rich Execution Environment}}
\acro{RIC}{Registro de Identidade Civil}
\acro{RG}{Registro Geral}
\acro{RP}{\textit{Relying Party}}
\acro{RSA}{}
\acro{RSL}{Revisão Sistemática da Literatura}
%S
\acro{SAML}{\textit{Security Assertion Markup Language}}
\acro{SAT}{\textit{SIM Application Toolkit}}
\acro{SB}{\textit{Service Broker}}
\acro{SCE}{\textit{Secure Collaborative Environment}}
\acro{SE}{\textit{Secure Element}}
\acro{SEM}{\textit{Scanning Electron Microscope}}
\acro{SGId}{Sistema de Gestão de Identidade}
\acro{SIG}{\textit{Special Interest Group}}
\acro{SIM}{\textit{Subscriber Identification Module}}
\acro{SISP}{Sistema de Administração dos Recursos de Tecnologia da Informação}
\acro{SLCS}{\textit{Short Lived Credential Service}}
\acro{SMC}{\textit{Secure Memory Card}}
\acro{SO}{Sistema Operacional}
\acro{SOAP}{\textit{Simple Object Access Protocol}}
\acro{SoC}{\textit{System on Chip}}
\acro{SP}{\textit{Service Provider}}
\acro{SSL}{\textit{Secure Sockets Layer}}
\acro{SSO}{\textit{Single Sign-On}}
\acro{SSTC}{\textit{Security Services Technical Committee}}
\acro{STORK}{\textit{Secure idenTity acrOss boRders linKed}}
%T
\acro{TAL}{\textit{Trust Anchor List}}
\acro{TEE}{\textit{Trusted Execution Environment}}
\acro{TIC}{Tecnologia da Informação e Comunicação}
\acro{TII}{\textit{Telecommunication Infrastructure Index}}
\acro{TLS}{\textit{Transport Layer Security}}
\acro{TSE}{Tribunal Superior Eleitoral}
%U
\acro{U2F}{\textit{Universal Second Factor}}
\acro{UA}{\textit{User Agent}}
\acro{UAF}{\textit{Universal Authentication Framework}}
\acro{UICC}{\textit{Universal Integrated Circuit Card}}
\acro{URL}{\textit{Uniform Resource Locator}}
\acro{USB}{\textit{Universal Serial Bus}}
\acro{USIM}{\textit{Universal Subscriber Identity Module}}
%V
\acro{VC}{\textit{Virtual Communities}}
\acro{VE}{\textit{Virtual Enterprises}}
\acro{VL}{\textit{Virtual Laboratories}}
\acro{VMM}{\textit{Virtual Machine Monitor}}
\acro{VO}{\textit{Virtual Organizations}}
\acro{VOMS}{\textit{Virtual Organization Management System}}
\acro{VRE}{\textit{Virtual Research Environment}}
%W
\acro{WAYF}{\textit{Where Are You From}}
\acro{WWW}{\textit{World Wide Web}}
%X
\acro{XACML}{\textit{eXtensible Access Control Markup Language}}
\acro{XML}{\textit{EXtensible Markup Language}}
%Y
%Z






\end{acronym}
% ---
% inserir o sumario
% ---
% Ajuste na linha 185/186 do abntex2.cls
% Opções de diagramação de sumários
% sumario=tradicional    : Sumário tradicional do LaTeX/Memoir
% sumario=abnt-6027-2012 : Sumário conforme recomendação da ABNT NBR 6027:2012
\pdfbookmark[0]{\contentsname}{toc}
\tableofcontents*
% \cleardoublepage
% ---


% ----------------------------------------------------------
% ELEMENTOS TEXTUAIS
% ----------------------------------------------------------
\textual

% espaçamento entre linhas 
\DoubleSpacing

\chapter{Introdução}
\label{c_introducao}

A cibersegurança consolidou-se como um pilar estratégico para a soberania nacional e para a estabilidade econômica global. No entanto, o setor enfrenta um déficit significativo de profissionais qualificados, fenômeno globalmente conhecido como \textit{Cybersecurity Skills Gap}. Relatórios recentes indicam que a força de trabalho atual precisa crescer para atender à crescente demanda por proteção de dados e de infraestruturas críticas globais \cite{ISC2_2023, ISC2_2024, ISC2_2025, ISACA2025, WEF2025, ITU2024}.

O Estudo sobre a Força de Trabalho em Cibersegurança da ISC2\footnote{A ISC2 é uma associação global de membros sem fins lucrativos dedicada à formação, certificação e desenvolvimento de profissionais de segurança da informação, com foco na criação de padrões de excelência para o setor.}, discorre sobre como as pressões econômicas, exacerbadas por incertezas geopolíticas, levaram a reduções orçamentárias e de força de trabalho em diversos setores, enquanto as ameaças de cibersegurança e os incidentes de segurança de dados continuaram a crescer. Estima-se que a força de trabalho global de cibersegurança em 2024 seja de 5.468.173 profissionais. Este é um aumento de apenas 0,1\% em relação a 2023, ou seja, o crescimento da força de trabalho está desacelerando, considerando que houve um aumento de 8,7\% entre 2022 e 2023 \cite{ISC2_2023, ISC2_2024}.

Já em 2025, o mesmo relatório concluiu que a necessidade de habilidades específicas (\textit{skills}) tornou-se mais importante do que o aumento do quadro de funcionários (\textit{headcount}). Embora o número de profissionais tenha se estabilizado, o estudo revelou que 95\% dos entrevistados identificaram lacunas de competências em suas equipes. Essas lacunas não são genéricas, mas concentram-se em áreas técnicas (\textit{hard skills}), com destaque para Inteligência Artificial (41\%), Segurança em Nuvem (36\%) e Avaliação de Riscos (29\%). Simultaneamente, o relatório sublinha a premente necessidade de habilidades comportamentais (\textit{soft skills}), essenciais para a gestão de crises e a colaboração eficaz, sendo a capacidade de resolução de problemas e a comunicação assertiva as competências não técnicas mais requisitadas por gestores e profissionais da área \cite{ISC2_2025}.

A sub-representação feminina na cibersegurança constitui um fenômeno global persistente, corroborado por múltiplos relatórios da indústria. Embora a participação das mulheres tenha apresentado crescimento moderado na última década, elas ainda constituem uma minoria significativa. Dados do relatório global da Fortinet indicam que as mulheres correspondem, em média, apenas 27\% das equipes de \ac{TI} e segurança nas organizações pesquisadas \cite{Fortinet2025}. Esse resultado está alinhado às estimativas da ISC2 e a estudos acadêmicos recentes, que situam a participação feminina na força de trabalho global entre 20\% e 25\%, evidenciando que, apesar das iniciativas de diversidade, o setor continua majoritariamente masculino \cite{ISC2_2025, SelmanHousein2025, Pacheco2024}. Além disso, observa-se uma disparidade geracional: enquanto mulheres representam 26\% dos profissionais com menos de 30 anos, elas compõem apenas 13\% da força de trabalho acima de 60 anos, indicando que o ingresso de novas profissionais ainda ocorre em ritmo insuficiente para compensar o déficit histórico de participação feminina \cite{ISC2_2023, ISC2_2024, ISC2_2025}.

Paradoxalmente, essa sub-representação não decorre de uma falta de qualificação formal. Conforme apontado no relatório da ISC2, as mulheres na área são, em média, mais escolarizadas que os homens: 52\% delas possuem pós-graduação, comparado a apenas 44\% de seus pares masculinos. Contudo, essa qualificação superior não se traduz em equidade de poder ou remuneração. As mulheres ocupam apenas 17\% dos cargos de liderança executiva e enfrentam uma brecha salarial persistente, recebendo cerca de 12\% a menos que homens em posições equivalentes \cite{ISC2_2025}. Esses dados sugerem que a barreira não é apenas técnica, mas estrutural, impedindo a ascensão profissional feminina mesmo quando há competência comprovada.

A literatura e os relatórios de mercado indicam que o problema se manifesta tanto na atração quanto na retenção de talentos, fenômeno frequentemente descrito como "vazamento de talentos" (\textit{leaky pipeline}). Estudos apontam que uma parcela significativa de mulheres que ingressa em carreiras tecnológicas acaba deixando a área, muitas vezes devido a ambientes profissionais excludentes e da falta de modelos de referência (\textit{role models}) \cite{Musuva2025, Pacheco2024}. Adicionalmente, observa-se uma fragilidade na base educacional: pesquisas indicam que uma parcela significativa de jovens mulheres relata que a cibersegurança não lhes foi apresentada como uma opção de carreira viável por educadores, perpetuando estereótipos que associam a área a um domínio predominantemente masculino e dificultando o preenchimento das vagas necessárias para a resiliência cibernética global \cite{Ramonyai2023, WEF2025}.

No cenário brasileiro, a publicação do Decreto n.º 12.573/2025 \cite{Decreto12573}, que institui a nova Estratégia Nacional de Cibersegurança (E-Ciber), marca um avanço significativo na governança digital do país. O documento reforça a necessidade de fomentar linhas de pesquisa que estimulem a formação massiva de profissionais qualificados e a promoção da diversidade no setor. Dentre as ações estratégicas delineadas, destacam-se o incentivo à inclusão de temas de cibersegurança nos currículos de todos os níveis educacionais, a capacitação continuada de professores e gestores, e a implementação de campanhas nacionais de conscientização sobre o uso seguro do ciberespaço, visando criar uma cultura de segurança enraizada na sociedade \cite{GSINacional}.


Para alcançar seus objetivos, a E-Ciber foi estruturada em quatro eixos temáticos fundamentais, cada um com focos específicos de atuação:

\begin{itemize}
    \item \textbf{Proteção e Conscientização da Sociedade:} Este eixo concentra-se na criação de um ambiente digital seguro para o cidadão, com ênfase prioritária na proteção de grupos em situação de vulnerabilidade, como crianças, adolescentes, idosos e pessoas neurodivergentes, além de promover a educação digital em larga escala.
    
    \item \textbf{Segurança e Resiliência das Infraestruturas Críticas:} Visa garantir a continuidade e a integridade dos serviços essenciais (como energia, telecomunicações e finanças) e das infraestruturas críticas nacionais, assegurando que continuem operando mesmo diante de incidentes cibernéticos severos.
    
    \item \textbf{Cooperação e Integração:} Tem como objetivo romper os silos de informação, fomentando o intercâmbio ágil de dados sobre ameaças entre os setores público e privado, além de fortalecer a posição do Brasil em fóruns internacionais e redes de cooperação global.
    
    \item \textbf{Soberania Nacional e Governança:} Busca reduzir a dependência tecnológica externa do país em áreas estratégicas, incentivando o desenvolvimento de uma indústria nacional de cibersegurança robusta, o fomento à pesquisa e inovação (PD\&I) e o estabelecimento de um modelo nacional de maturidade em segurança cibernética.
\end{itemize}

Como resposta estratégica ao déficit de profissionais em cibersegurança no Brasil, foi instituído o Programa \textit{Hackers} do Bem. Trata-se de uma iniciativa financiada pelo \ac{MCTI} com recursos da Lei de Informática (Lei no 8.248) \cite{Lei8248}, coordenada pela Softex e executada pela \ac{RNP} e pelo SENAI São Paulo. O programa tem como meta qualificar mais de 30 mil profissionais por meio de um ecossistema de ensino gratuito, escalável e fundamentado em gamificação e atividades práticas simuladas. Projetado para atender estudantes do ensino técnico, médio ou da universidade, profissionais da área de TI que procuram se especializar e até quem quer migrar de área de conhecimento. O itinerário formativo é composto pelas trilhas de Nivelamento, Básico, Fundamental, Especializado e Residência Tecnológica, as quais são estruturadas para alinhar o desenvolvimento de competências técnicas às demandas da indústria e aos imperativos de soberania tecnológica nacional \cite{HackersDoBemManual}.

O Programa \textit{Hackers} do Bem registrou um total de 150.832 inscritos, dos quais aproximadamente 100.000 ingressaram efetivamente no \ac{AVA}. As metas de formação foram estruturadas para atender a esse volume massivo, visando qualificar 30.000 alunos nas etapas iniciais (Nivelamento e Básico), avançando para 3.000 no nível Fundamental e 2.000 no Especializado, culminando na especialização prática de 216 profissionais na Residência Tecnológica. Além da escala, o programa monitora indicadores demográficos para o setor, identificando que 22,45\% do total de inscritos se declaram mulheres, percentual que evidencia a necessidade contínua de ações afirmativas para a equidade de gênero na área. Recentemente, a iniciativa expandiu ainda mais seu alcance com a abertura de 25.000 novas vagas e já contabiliza mais de 36.000 certificados em todo o país\footnote{Dados do programa \textit{Hackers} do bem cedidos pela \ac{RNP}}.

Considerando este cenário, esta dissertação investiga de que forma os programas de capacitação  em cibersegurança com alcance nacional e capacidade de atendimento em larga escala, como o Programa \textit{Hackers} do Bem, podem ser customizados de modo a ampliar o ingresso e a retenção de talentos femininos. Busca-se, ainda, compreender se as estratégias utilizadas nesses programas atuam como catalisadores de inclusão ou, ao contrário, configuram barreiras adicionais à formação e permanência de novas profissionais de cibersegurança.

\section{Problema de Pesquisa}
\label{s_problema_pesquisa}

Apesar da crescente demanda por profissionais de cibersegurança e dos esforços globais para diversificar a força de trabalho, as mulheres permanecem sub-representadas, ocupando apenas cerca de 25\% dos postos de trabalho globais na área \cite{Kshetri2022, ISC2_2023, ISC2_2024, ISC2_2025, ISACA2025, ITU2024}. A literatura aponta para um fenômeno persistente de "vazamento de talentos" (\textit{leaky pipeline}), onde as taxas de evasão femininas superam as masculinas à medida que o nível técnico e a competitividade aumentam \cite{Pitman2022}. Este fenômeno não é apenas uma questão de recrutamento, mas de retenção, influenciada por barreiras estruturais, culturais e pedagógicas \cite{BothaBadenhorst2023, SelmanHousein2025}.

No contexto do Programa \textit{Hackers} do Bem, dados preliminares indicam uma materialização deste cenário: enquanto as mulheres representam 22,45\% do total de inscritos, sua participação cai para aproximadamente 13\% no início das fases síncronas e especializadas. A progressão entre os níveis do programa (Nivelamento, Básico, Fundamental, Especializado e Residência) é regida por um sistema de ranqueamento baseado no acúmulo de pontos de experiência (\textit{XP}), desempenho em \textit{quizzes}, atividades práticas e simulações \cite{HackersDoBemManual}.

Estudos indicam que competições de cibersegurança, frequentemente focadas em disputas individuais e rankings públicos, podem reforçar estereótipos masculinos de competitividade e isolamento, desestimulando a participação de grupos sub-representados que tendem a valorizar abordagens colaborativas e aprendizagem baseada em impacto social \cite{Horcher2021, Gough2024}. A literatura sugere que a gamificação competitiva, se não ajustada para a diversidade, pode atuar inadvertidamente como uma barreira de entrada e permanência, exacerbando a insegurança, a síndrome do impostor e a baixa sensação de autoeficácia entre estudantes mulheres \cite{Hogan2025, Kanij2025}.

No contexto das trajetórias femininas, a autoeficácia é definida na literatura como a crença das mulheres em sua própria capacidade de organizar e executar ações necessárias para alcançar determinados resultados, configurando-se como um importante preditor do engajamento, da persistência e do sucesso em carreiras tecnológicas \cite{Pacheco2024}. No contexto específico da cibersegurança, mulheres frequentemente relatam níveis inferiores de autoeficácia em comparação aos seus pares masculinos, o que pode resultar em uma postura de maior aversão ao risco e limitar a progressão na carreira, independentemente da competência real adquirida \cite{SelmanHousein2025, Benson2025}. Contudo, evidências indicam que a autoeficácia não é uma característica imutável, podendo ser fortalecida por meio de intervenções pedagógicas intencionais, como a utilização de apoio estruturado em desafios práticos e a promoção de ambientes colaborativos que validem a identidade profissional das estudantes \cite{Gough2024}.


Adicionalmente, há uma escassez de estudos que analisem quantitativamente e qualitativamente o impacto de intervenções no \textit{design} de cursos de cibersegurança especificamente voltados para a retenção de mulheres, em oposição a apenas esforços de recrutamento \cite{SelmanHousein2025}. Considerando que o \textit{design} de jogos educacionais muitas vezes reflete os vieses culturais de seus criadores, existe o risco de que os mecanismos de seleção do programa reproduzam as desigualdades que visam combater \cite{Coenraad2020}.

Diante deste cenário, define-se a seguinte Questão de Pesquisa Principal:

\begin{quote}
\textbf{QP:} Como programas de capacitação em cibersegurança com capacidade de atendimento em larga escala, como o \textit{Hackers} do Bem, podem ser customizados para ampliar o ingresso e a permanência de talentos femininos?
\end{quote}

Esta questão desdobra-se nas seguintes questões específicas:
\begin{itemize}
    \item \textbf{QP1:} Quais fatores socioculturais e estruturais são apontados na literatura como determinantes para a baixa adesão e permanência em programas de capacitação em cibersegurança, e quais intervenções têm sido adotadas para aumentar a adesão e mitigar a evasão do publico feminino?
    \item \textbf{QP2:} Quais abordagens pedagógicas são adotadas em programas de capacitação em cibersegurança e de que maneira elas impactam a adesão e a permanência do público feminino nesses programas?
    \item \textbf{QP3:} Em que medida a customização de trilhas de aprendizagem e a implementação de mecanismos colaborativos podem influenciar a percepção de autoeficácia e a intenção de permanência das estudantes, em programas como o \textit{Hackers} do Bem? \end{itemize}

\subsection{Solução Proposta}
\label{ss_solucao_proposta}

A solução proposta nesta pesquisa consiste no desenvolvimento de um \textit{framework} de diretrizes voltado à inclusão e retenção de mulheres em programas de capacitação em cibersegurança, fundamentado em evidências empíricas. A construção deste artefato baseia-se em uma abordagem de métodos mistos, que combina a mineração de dados educacionais de participantes do Programa \textit{Hackers} do Bem com a análise qualitativa da percepção das alunas. Nesse processo, correlacionam-se métricas de desempenho técnico às taxas de evasão por gênero, com o objetivo de compreender os fatores subjetivos que influenciam a permanência ou a desistência das estudantes.

A avaliação do \textit{framework} ocorrerá por meio de um processo de validação estruturado em duas etapas principais. Primeiramente, o artefato será submetido a um painel de especialistas nas áreas de cibersegurança, educação e diversidade, que analisarão a viabilidade técnica, a clareza e a relevância das diretrizes propostas. Posteriormente, a percepção sobre a utilidade da solução será verificada junto às próprias participantes do programa, visando confirmar se as recomendações propostas atendem às necessidades do público-alvo e contribuem para o fortalecimento da autoeficácia. Ao final, espera-se que o \textit{framework} forneça recomendações práticas para o \textit{design} de programas de treinamento, focando na neutralização de vieses de avaliação e na promoção de ambientes de aprendizado que favoreçam a retenção feminina em larga escala.

\subsection{Delimitação de Escopo}
\label{ss_delimitacao_escopo}

O escopo da pesquisa limita-se à análise dos dados das turmas do Programa \textit{Hackers} do Bem realizadas entre 2024 e 2026. A análise quantitativa é voltada para as transições do funil de formação (do Nivelamento à Residência Tecnológica). Não faz parte do escopo a alteração do código-fonte da plataforma de ensino ou a intervenção direta nas turmas em andamento, caracterizando-se como um estudo observacional e propositivo.

\subsection{Justificativa}
\label{ss_justificativa}

Sob a perspectiva socioeconômica, a relevância desta pesquisa fundamenta-se na necessidade de investigar meios de mitigar o déficit global da força de trabalho em cibersegurança, em um cenário no qual a demanda por profissionais qualificados supera a oferta disponível. Relatórios de mercado indicam que a escassez de talentos não pode ser resolvida sem a inclusão efetiva da população feminina, que atualmente representa uma parcela minoritária do setor \cite{Pacheco2024}. Para além do aspecto quantitativo, a literatura técnica aponta que a diversidade em equipes de segurança resulta em maior eficiência operacional e melhor desempenho na resolução de problemas complexos, uma vez que diferentes perspectivas cognitivas reduzem pontos cegos em estratégias de defesa \cite{Benson2025}. Portanto, investigar os fatores que influenciam a permanência de mulheres em programas de capacitação (QP1) constitui um dos requisitos para que iniciativas de formação em larga escala atinjam seus objetivos de suprir a demanda do mercado.

No âmbito acadêmico, este trabalho endereça uma lacuna identificada em revisões sistemáticas da literatura recentes, que apontam a escassez de estudos focados em métricas de impacto quantificáveis sobre a retenção de mulheres, em contraposição à vasta literatura existente sobre recrutamento e interesse inicial em formações em cibersegurança \cite{SelmanHousein2025}. Enquanto muitos trabalhos descrevem barreiras culturais de entrada de forma genérica, há uma carência de investigações que analisem como as metodologias pedagógicas impactam a continuidade da formação feminina. Ao analisar dados reais de um programa massivo, esta dissertação contribui para o estado da arte ao analisar as estratégias de retenção empregadas e poderá fazer proposições de melhorias e de novas estratégias.

Ainda na esfera acadêmica, a pesquisa aprofunda a discussão sobre as diferentes metodologias aplicadas em treinamentos de cibersegurança. Estudos indicam que mecanismos competitivos tradicionais, como competições do tipo \textit{Capture The Flag} (CTF) e \textit{leaderboards} públicos, podem inadvertidamente elevar a ansiedade de performance e diminuir a autoeficácia em grupos sub-representados, caso não sejam equilibrados com abordagens colaborativas \cite{Hogan2025, Costa2025}. Ao investigar como esses elementos técnicos influenciam o engajamento no \textit{Hackers} do Bem (QP2), este estudo oferece uma contribuição teórica sobre como adaptar ferramentas pedagógicas para transformar ambientes competitivos em espaços de fomento à competência técnica e ao pertencimento.

Por fim, a análise do Programa \textit{Hackers} do Bem como estudo de caso oferece uma oportunidade de validar diretrizes de customização em um ambiente de ensino em larga escala (QP3). Diferentemente de estudos limitados a salas de aula pequenas ou pilotos controlados, a escala deste programa permite verificar a aplicabilidade de estratégias de intervenção em um cenário real de formação massiva. O resultado esperado é a formulação de diretrizes baseadas em evidências que permitam reduzir as taxas de evasão.

\section{Objetivos}
\label{s_objetivos}

Esta seção formaliza os objetivos do trabalho, conforme descrito a seguir.

\subsection{Objetivo Geral}
\label{ss_objetivo_geral}

Definir e avaliar um conjunto de diretrizes para o aumento da atração e permanência feminina em programas de capacitação em cibersegurança.

\subsection{Objetivos Específicos}
\label{ss_objetivos_especificos}

\begin{itemize}
    \item Mapear as estratégias de retenção e barreiras de gênero documentadas na literatura, por meio de \ac{RSL}
    
    \item Identificar os pontos críticos de evasão feminina do Programa \textit{Hackers} do Bem, por meio de uma análise quantitativa do funil de progressão do programa e de uma análise qualitativa conduzida por meio de entrevistas com as participantes do programa.
    
    \item Avaliar a correlação entre os mecanismos de gamificação (\ac{XP}, Ranking), outras estratégias de engajamento e o desempenho das participantes do programa, por meio da analise dos dados do programa \textit{Hackers} do Bem.
    
    \item Elaborar o \textit{Framework} de Diretrizes e avaliá-lo por meio de um painel de especialistas e com as participantes do programa.
\end{itemize}

\section{Metodologia}
\label{s_metodologia}

Esta seção apresenta a metodologia de pesquisa adotada e o procedimento metodológico
aplicado para realização deste trabalho.

\subsection{Metodologia da Pesquisa}
\label{ss_metodologia_pesquisa}

Este trabalho adota como abordagem metodológica o método hipotético-dedutivo. A investigação a partir da lacuna de conhecimento identificada na literatura sobre o fenômeno do "vazamento de talentos" (\textit{leaky pipeline}) em cibersegurança \cite{Pitman2022} e a hipótese de que mecanismos de gamificação competitiva, quando não ajustados para a diversidade, impactam negativamente a autoeficácia e a retenção de mulheres \cite{Hogan2025, Coenraad2020}. Tais hipóteses são avaliadas a partir da análise dos documentos, regras de negócio e dados de participação do programa objeto deste estudo.

Além disso, este trabalho classifica-se como uma pesquisa aplicada, voltada à produção de conhecimento para problemas práticos, especificamente a disparidade de gênero em programas de capacitação de larga escala em cibersegurança. Sob a ótica da avaliação, adota-se uma abordagem mista: qualitativa, baseada em um estudo de caso que analisa os recursos pedagógicos, o sistema de ranqueamento e as regras de progressão do programa \textit{Hackers} do Bem, bem como a percepção das participantes; e quantitativa, por meio da análise de dados de desempenho dos estudantes vinculados ao programa \cite{HackersDoBemManual}.

Por fim, esta pesquisa possui uma natureza exploratória e descritiva. Exploratória, pois investiga a correlação entre o \textit{design} de programas de treinamento em cibersegurança e a percepção de competência técnica de grupos sub-representados \cite{Horcher2021}. Descritiva, pois detalha o funcionamento do ecossistema do programa \textit{Hackers} do Bem, estabelecendo conexões com a bibliografia geral sobre educação em cibersegurança e estratégias de retenção de talentos.

\subsection{Procedimentos Metodológicos}
\label{ss_procedimentos_metodologicos}

Nesta seção, são apresentados os procedimentos metodológicos aplicados para o atingimento dos objetivos propostos anteriormente, a fim de fundamentar e analisar a abordagem proposta. As fases já desenvolvidas e as que ainda serão desenvolvidas no trabalho são:

\begin{enumerate}
    \item \textbf{\ac{RSL}:} Executada conforme o protocolo \ac{PICO}, nas bases ACM, IEEE, Scopus e Google Scholar, para fundamentar o estado da arte atual.
    
    \item \textbf{Análise de trabalhos relacionados:} Leitura e análise dos trabalhos selecionados pela \ac{RSL} a fim de categorizar e extrair dados e quais conclusões esses artigos chegaram a cerca da presença feminina em programas de formação em cibersegurança.
    
    \item \textbf{Coleta de Dados:} Obtenção dos registros do programa junto à \ac{RNP} e aplicação de questionários junto às participantes do programa.
    
    \item \textbf{Análise de Dados:} Processamento estatístico para cálculo de taxas de evasão e correlação de variáveis.
    
    \item \textbf{Definição do \textit{Framework} de diretrizes:} Síntese dos achados em recomendações práticas para o programa.

    \item \textbf{Avaliação:} Aplicação de questionários para um painel de especialistas e participantes do programa.
    
\end{enumerate}

\section{Estrutura da Dissertação}
\label{s_estrutura_dissertacao}

O \autoref{c_fundamentacao_teorica} apresenta a fundamentação teórica bem como o Programa \textit{Hackers} do Bem e toda sua estrutura. O \autoref{c_trabalhos_relacionados} detalha o diagnóstico do estado da arte bem como os procedimentos metodológicos para se levantar os trabalhos utilizados. Por fim, o \autoref{c_proposta} apresenta a proposta que esse trabalho pretende desenvolver, incluindo o detalhamento do método de análise de dados, avaliação e o cronograma.

\chapter{Fundamentação Teórica}
\label{c_fundamentacao_teorica}

Este capítulo apresenta os conceitos fundamentais necessários para a compreensão do problema de pesquisa e da análise realizada. A fundamentação está dividida em quatro eixos principais: o cenário global e nacional da força de trabalho em cibersegurança; as disparidades de gênero e barreiras na área tecnológica; as metodologias de ensino, com foco em gamificação e competições; e o detalhamento do objeto de estudo, o Programa \textit{Hackers} do Bem.

\section{Cenário da Força de Trabalho em Cibersegurança}
\label{s_cenario_forca_trabalho_ciberseguranca}

A escassez de profissionais qualificados em segurança da informação é um desafio global que impacta diretamente a capacidade de defesa de organizações e nações. Nesta seção, discute-se a evolução do déficit de talentos (\textit{workforce gap}) e as políticas públicas recentes para mitigação.

\subsection{O Déficit Global e Nacional de Talentos}
\label{ss_deficit_global_nacional_talentos}

Dados recentes do \textit{ISC2 Cybersecurity Workforce Study} indicam que a lacuna global da força de trabalho atingiu a marca histórica de 4 milhões de profissionais, representando um aumento expressivo em relação aos anos anteriores \cite{ISC2_2024}. Este déficit é calculado pela diferença entre a demanda das organizações para proteger adequadamente seus ativos e a oferta real de profissionais disponíveis para contratação.

É importante, no entanto, qualificar a natureza desta escassez. Conforme observam \citeonline{Musuva2025} ao analisarem programas de capacitação, o problema não reside apenas na ausência de candidatos, mas na desconexão entre as competências possuídas pelos estudantes e as habilidades técnicas específicas exigidas pelo mercado. O fenômeno, frequentemente descrito como \textit{skills mismatch}, sugere que a massificação do ensino sem o devido alinhamento com as demandas práticas da indústria, resulta em posições não preenchidas, mesmo em cenários de desemprego estrutural.

O \ac{WEF}, em seu relatório \textit{Global Cybersecurity Outlook 2025}, reforça este cenário ao destacar que a iniquidade da força de trabalho afeta desproporcionalmente setores críticos com menos recursos, como educação, governo e \ac{PMEs} \cite{WEF2025}. No contexto brasileiro, o \ac{WEF} aponta que, embora o país esteja avançando na maturidade cibernética através de iniciativas como a E-Ciber, a escassez de talentos permanece um gargalo, que limita a resiliência das infraestruturas nacionais, frente à sofisticação crescente das ameaças.

Portanto, a resolução deste déficit exige estratégias que ultrapassem o recrutamento tradicional, focando no desenvolvimento de competências práticas e na retenção de talentos, uma vez que a simples expansão numérica da força de trabalho, sem a devida qualificação técnica e comportamental, mostra-se insuficiente para mitigar os riscos cibernéticos atuais. Segundo dados recentes \cite{ISC2_2024}, o déficit global ultrapassa milhões de profissionais, com o Brasil figurando como um dos países com a maior carência de mão de obra especializada.

\subsection{Políticas Públicas e a Estratégia Nacional (E-Ciber)}
\label{ss_politicas_publicas_estrategia_nacional}

No contexto brasileiro, a resposta governamental à escassez de mão de obra qualificada materializou-se através da instituição da Estratégia Nacional de Cibersegurança (E-Ciber), formalizada pelo Decreto nº 12.573, de 4 de agosto de 2025 \cite{Decreto12573}. Este instrumento legal redefine a governança do setor e estabelece a educação e a capacitação massiva como pilares fundamentais para a garantia da soberania digital do país.

A E-Ciber está estruturada em eixos temáticos que orientam as ações do Estado. Destaca-se o Eixo 4 (Soberania Nacional e Governança), cujo Artigo 9º, inciso III, determina explicitamente a necessidade de "formação e capacitação técnico-profissional em cibersegurança em escala compatível com as necessidades nacionais" \cite{Decreto12573}. Esta diretriz legal justifica a transição de ações de treinamento isoladas para iniciativas de grande porte, visando reduzir o débito tecnológico do país.

Adicionalmente, o Eixo 1 (Proteção e Conscientização) preconiza o incentivo à inclusão de temas de cibersegurança nos currículos educacionais e a capacitação de profissionais, buscando criar uma cultura de segurança sustentável na sociedade \cite{GSINacional}. É neste cenário regulatório que se insere o Programa \textit{Hackers} do Bem, objeto deste estudo. A iniciativa não surge isoladamente, mas como uma resposta operacional às demandas da E-Ciber e da Política Nacional de Cibersegurança. Financiado pelo \ac{MCTI} no âmbito do \ac{PPI} da Softex, e executado pela \ac{RNP} em parceria com o \ac{SENAI}, o programa visa estruturar um ecossistema nacional de segurança digital, integrando formação, pesquisa e inovação para também atender à escala demandada pela legislação vigente \cite{MCTI2025}.

\section{Desigualdade de Gênero na Cibersegurança}
\label{s_desigualdade_genero_ciberseguranca}

A sub-representação feminina na cibersegurança não é apenas uma questão de diversidade, mas de eficiência econômica e inovação. Esta seção explora as barreiras estruturais e culturais que impedem a entrada e a permanência de mulheres na área.

\subsection{Panorama da Representatividade Feminina}
\label{ss_panorama_representatividade_feminina}

Apesar dos avanços na última década, as mulheres ainda representam uma parcela minoritária da força de trabalho em cibersegurança. Estudos como o de \citeonline{Pacheco2024} indicam que a participação feminina estagnou em torno de 25\%, com disparidades ainda maiores em cargos de liderança técnica, onde mulheres ocupam menos de 20\% das posições de \ac{CISO} nas grandes corporações globais.

Esta estagnação não é uniforme em todo o setor de tecnologia. Conforme analisado por \citeonline{SelmanHousein2025}, ao comparar a cibersegurança com outras subáreas da computação, observa-se que o índice de 25\% é superior ao de campos emergentes como Inteligência Artificial (12\%) e Robótica (11\%), porém significativamente inferior à área de Educação em Computação, onde a representatividade feminina atinge 42\%.

Estes dados sugerem que a barreira não é apenas tecnológica, mas cultural e estrutural, específica do domínio de segurança. A persistência desse \textit{gap} ao longo dos anos, mesmo com o aumento da demanda por profissionais, corrobora com a existência de um fenômeno de "vazamento de talentos" (\textit{leaky pipeline}), onde as mulheres que ingressam na área encontram dificuldades de permanência e ascensão profissional superiores às observadas em outros setores da \ac{TI} \cite{SelmanHousein2025}.

\subsection{Barreiras de Entrada e o Fenômeno do \textit{Leaky Pipeline}}
\label{ss_barreiras_entrada_fenomeno_leaky_pipeline}

O conceito de \textit{Leaky Pipeline} (vazamento de talentos) descreve a perda progressiva de participação feminina ao longo da jornada educacional e profissional. \citeonline{SelmanHousein2025} identificam que a falta de modelos de referência (\textit{role models}), a percepção de um ambiente hostil e a ausência de mentorias estruturadas são fatores determinantes para a evasão precoce em programas de formação. A revisão da literatura aponta que, mesmo quando o interesse inicial existe, a retenção decai drasticamente entre o ensino médio e a especialização profissional devido à falta de suporte contínuo.

Esta evasão é intensificada por barreiras culturais e estereótipos profundamente enraizados na área. \citeonline{Costa2025} argumentam que a imagem predominante do profissional de cibersegurança, associada ao estereótipo do "\textit{hacker} de capuz" isolado ou à cultura excludente do \textit{"brogrammer"}, cria uma dissonância cognitiva para mulheres, que não se enxergam pertencentes a esse grupo. Tais estereótipos perpetuam a falsa noção de que a competência técnica em segurança, exige um perfil comportamental antissocial ou agressivamente competitivo.

Consequentemente, a ausência de representatividade feminina em posições de liderança e a carência de redes de mentoria reforçam este ciclo de exclusão. Sem intervenções deliberadas que desconstruam essas imagens e ofereçam suporte vocacional, como as propostas em programas de capacitação inclusivos, o funil de talentos continuará a perder candidatas qualificadas, antes mesmo de sua entrada formal no mercado de trabalho \cite{SelmanHousein2025, Costa2025}.

\subsection{A Importância das \textit{Soft Skills} na Retenção}
\label{ss_importancia_soft_skills_retencao}

A valorização excessiva de competências puramente técnicas (\textit{hard skills}) em detrimento de habilidades comportamentais e analíticas (\textit{soft skills}) atua como uma barreira de entrada. A percepção da cibersegurança como uma área restrita à operação técnica de baixo nível, ignora a complexidade das ameaças modernas, que exigem gestão de crise, comunicação estratégica e liderança adaptativa.

\citeonline{Benson2025} argumentam que a reformulação dos currículos para integrar competências como comunicação, gestão de riscos e análise forense é uma estratégia eficaz para aumentar a atratividade da área para o público feminino. Seu estudo qualitativo com profissionais da área na Europa destaca que mulheres frequentemente demonstram níveis elevados de resiliência relacional, empatia e colaboração, atributos essenciais para a resposta a incidentes e gestão de equipes diversificadas.

No entanto, os autores identificam uma lacuna crítica: \textit{frameworks} de competências amplamente utilizados, como o \ac{ECSF}, falham ao não formalizar essas \textit{soft skills} como requisitos de carreira, perpetuando processos de recrutamento e avaliação enviesados para o técnico.

Consequentemente, programas de capacitação que focam exclusivamente em pontuação técnica (como \ac{CTFs} tradicionais), podem inadvertidamente desviar participantes que possuem o perfil analítico e comunicacional demandado pelo mercado, mas que não se identificam com a cultura puramente técnica. A integração formal de \textit{soft skills} na trilha de aprendizado não apenas valida a competência feminina, mas fortalece a resiliência organizacional como um todo.

\section{Metodologias de Ensino e Gamificação}
\label{s_metodologias_ensino_gamificacao}

Para escalar o ensino de cibersegurança, programas modernos recorrem a metodologias ativas e gamificação. Esta seção define os conceitos que sustentam a arquitetura pedagógica do programa analisado.

\subsection{\acf{GBL}}
\label{ss_gamificacao_aprendizagem_baseada_jogos}

A gamificação consiste na aplicação de elementos de design de jogos em contextos não lúdicos para aumentar o engajamento e a motivação. É fundamental distinguir este conceito de \ac{GBL} ou Jogos Sérios: enquanto a gamificação incorpora mecânicas isoladas (como pontuação) em um processo de ensino tradicional, o \ac{GBL} utiliza o jogo completo como o próprio meio de instrução.

A implementação mais comum de gamificação em plataformas de ensino baseia-se na tríade \ac{PBL}. Neste modelo, o aluno acumula \ac{XP} por tarefas concluídas, desbloqueia emblemas (\textit{badges}) ao atingir marcos de competência e compara seu desempenho com outros participantes através de rankings (\textit{leaderboards}). Estas mecânicas visam fornecer \textit{feedback} imediato e fomentar a competitividade saudável, elementos centrais na arquitetura do programa objeto deste estudo.

Em sua revisão sistemática, \citeonline{Coenraad2020} demonstram que o uso de pontuações, emblemas e narrativas imersivas em cibersegurança facilita a abstração de conceitos técnicos complexos e promove a aprendizagem ativa. Os autores analisaram 181 jogos digitais e identificaram que, embora eficazes para a conscientização inicial (\textit{cyber safety}), muitas iniciativas limitam-se a "quizzes gamificados" que falham em oferecer um engajamento profundo com o conteúdo técnico (\textit{deep content engagement}).

Segundo \citeonline{Coenraad2020}, para que a gamificação seja efetiva na formação profissional, ela deve transcender a recompensa extrínseca (pontos) e promover o engajamento epistêmico, onde o aluno assume o papel e a identidade do profissional de segurança (ex: o defensor de uma rede), simulando a prática real da profissão.

\subsection{\acf{CTFs} e Competições}
\label{ss_ctfs_competicoes}

As competições do tipo \ac{CTFs} consolidaram-se como o padrão para o treinamento prático e a avaliação de competências em segurança ofensiva e defensiva. Conforme categorizado por \citeonline{Cole2022}, estes exercícios dividem-se predominantemente em dois formatos: \textit{Jeopardy}, onde equipes resolvem desafios estáticos segregados por categorias (como criptografia, forense e exploração web), a fim de obter pontuações cumulativas; e \textit{Attack-Defense}, onde participantes defendem sua própria infraestrutura, enquanto tentam comprometer os sistemas adversários em tempo real.

No âmbito do Programa \textit{Hackers} do Bem, o modelo \textit{Jeopardy} constitui a base das avaliações práticas e do sistema de gamificação, sendo o principal mecanismo de aferição técnica para o ranqueamento dos alunos entre as fases de especialização \cite{HackersDoBemManual}.

Entretanto, a literatura aponta limitações críticas na aplicação universal deste modelo. \citeonline{Horcher2021} alerta que o formato tradicional de \ac{CTFs}, frequentemente focado em velocidade, agressividade competitiva e cenários de "soma zero", reflete um viés cultural que pode afetar grupos sub-representados. A autora argumenta que mulheres tendem a apresentar melhor desempenho e retenção em ambientes que priorizam a colaboração e a resolução analítica de problemas, sugerindo que a ausência de adaptações no design das competições atua como uma barreira de entrada estrutural, e não como um filtro de competência.

\subsection{Impacto do Ranqueamento na Inclusão}
\label{ss_impacto_ranqueamento_inclusao}

A utilização de rankings públicos (\textit{leaderboards}) é controversa na literatura de educação inclusiva. Enquanto promovem a competitividade, podem reduzir a autoeficácia de participantes que não se veem representados no topo da lista. \citeonline{Costa2025} demonstram que ambientes de aprendizado que priorizam a colaboração sobre a competição direta apresentam melhores taxas de retenção feminina.

A Teoria da Autoeficácia sugere que a confiança na própria capacidade de realizar tarefas é um preditor fundamental da persistência em áreas técnicas. Em contextos gamificados onde o \textit{feedback} é predominantemente comparativo (posição relativa no ranking) em vez de formativo (evolução individual), participantes de grupos sub-representados tendem a subestimar suas competências, especialmente quando expostos a estereótipos de gênero que associam a cibersegurança a um domínio masculino \cite{Costa2025}.

Corroborando esta visão, \citeonline{Horcher2021} argumenta que as mecânicas tradicionais de pontuação em \ac{CTFs}, frequentemente desenhadas para recompensar a velocidade e a agressividade ("soma zero"), favorecem perfis comportamentais socializados para a competição direta. A autora alerta que a falta de adaptação destas mecânicas para valorizar a resolução colaborativa de problemas pode aumentar a ansiedade de desempenho e atuar como um fator de exclusão.

No contexto do Programa \textit{Hackers} do Bem, onde o ranqueamento é um critério eliminatório para o acesso às vagas da Residência Tecnológica \cite{HackersDoBemManual}, a análise destes fatores torna-se crítica para compreender se o modelo de avaliação está medindo apenas a competência técnica ou se está, inadvertidamente, filtrando candidatos com base em seu perfil de competitividade.

\section{O Programa \textit{Hackers} do Bem}
\label{s_programa_hackers_bem}

O Programa Hackers do Bem, instituído pelo \acf{MCTI} e executado pela \acf{RNP} em parceria com o Senai e a Softex, constitui o objeto de estudo central desta dissertação.

\subsection{Objetivos e Estrutura da Formação}
\label{ss_objetivos_estrutura_formacao}

O programa estabeleceu como meta prioritária a capacitação de mais de 30 mil alunos no nível básico, além de formar 3 mil profissionais em nível intermediário e especializar cerca de 200 alunos em níveis avançados \cite{MCTI2025}. Para atingir tal escala, a arquitetura pedagógica foi dividida em cinco níveis progressivos de complexidade, combinando ensino assíncrono massivo com atividades síncronas práticas.

Conforme detalhado no Manual de Aprovação e Gamificação \cite{HackersDoBemManual}, a estrutura curricular organiza-se da seguinte forma:

\begin{enumerate}
    \item \textbf{Nivelamento (80h):} Etapa de entrada, aberta ao público geral e realizada de forma assíncrona. O currículo abrange fundamentos de \textit{hardware}, sistemas operacionais (\textit{Windows} e \textit{Linux}), redes de computadores (TCP/IP), lógica de programação e desenvolvimento de \textit{scripts}. O objetivo é uniformizar o conhecimento técnico dos candidatos antes das fases específicas de segurança.
    
    \item \textbf{Básico (64h):} Também na modalidade assíncrona, esta fase introduz conceitos centrais de cibersegurança, incluindo identificação de ameaças, \textit{malwares}, criptografia, e fundamentos de \ac{GRC}. A aprovação nesta etapa é pré-requisito para o ranqueamento nas fases subsequentes.
    
    \item \textbf{Fundamental (96h):} A partir deste nível, o programa adota um formato híbrido com aulas ao vivo e laboratórios práticos. O conteúdo aprofunda-se em arquitetura segura, autenticação, segurança em nuvem e resposta a incidentes. O acesso é limitado a um número específico de vagas, preenchidas via ranqueamento baseado no desempenho das fases anteriores.
    
    \item \textbf{Especializado (80h):} Nesta etapa, os alunos optam por uma trilha de especialização técnica específica. As trilhas disponíveis, conforme o cronograma do programa, são:
    \begin{itemize}
        \item \textit{\textit{Red Team}} (Segurança Ofensiva e \textit{Pentest});
        \item \textit{Blue Team} (Defesa e Monitoramento);
        \item Resposta a Incidentes e Forense Computacional;
        \item \textit{DevSecOps} (Desenvolvimento Seguro);
        \item \acl{GRC}.
    \end{itemize}
    
    \item \textbf{Residência Tecnológica (6 meses):} A fase final consiste em uma imersão prática intensiva, onde os alunos atuam em projetos reais ou simulados, visando a transição direta para o mercado de trabalho.
\end{enumerate}

O modelo de progressão entre estas fases não é automático. O programa utiliza um sistema de gamificação e ranqueamento (detalhado na \autoref{ss_ctfs_competicoes}) como mecanismo de filtro, onde o desempenho em \textit{quizzes} e atividades práticas determina a elegibilidade do aluno para avançar do nível Básico para o Fundamental, e assim sucessivamente, criando um funil de seleção meritocrático \cite{HackersDoBemManual}.

\subsection{Modelo de Gamificação e Critérios de Aprovação}
\label{ss_modelo_gamificacao_criterios_aprovacao}

O avanço entre os níveis do programa não é automático, sendo regido por um sistema de gamificação e ranqueamento. Conforme o Manual \cite{HackersDoBemManual}, os alunos acumulam \acf{XP} através do consumo de conteúdo, realização de \textit{quizzes} e atividades práticas. A classificação para as fases subsequentes, especialmente para a Residência, depende estritamente da posição do aluno no ranking geral, calculado pela soma de \ac{XP} e critérios de desempate baseados no desempenho em avaliações práticas. 

Este mecanismo de "funil competitivo" é o ponto focal da análise de dados que será apresentada nos capítulos seguintes. A arquitetura de gamificação do programa foi desenhada para operar em duas camadas distintas: a progressão visual (Emblemas e Selos) e a progressão funcional (Ranqueamento).

\subsubsection{Mecânica de Pontuação e Emblemas}
\label{sss_mecanica_pontuacao_emblemas}

A unidade fundamental de progresso no sistema é o \ac{XP}. Cada interação do aluno com a plataforma, seja a visualização de uma videoaula até a conclusão de um laboratório prático, atribui uma quantidade predeterminada de pontos. O acúmulo destes pontos desbloqueia emblemas de nível, que servem como indicadores visuais de proficiência, conforme detalhado na \autoref{tab:emblemas_xp} \cite{HackersDoBemManual}.

\begin{table}[htb]
\centering
\caption{Estrutura de Emblemas e Pontuação do Programa}
\label{tab:emblemas_xp}
\begin{tabular}{l|l|l}
\hline
\textbf{Nível do Emblema} & \textbf{\ac{XP} Necessário} & \textbf{Estágio Associado} \\ \hline
Aspirante a Hacker & 251 pontos & Conclusão do Nivelamento \\ \hline
Explorador de Códigos & 603 pontos & Início do Básico \\ \hline
Mestre de Segurança & 2.413 pontos & Conclusão do Fundamental \\ \hline
Guerreiro Digital & 3.413 pontos & Conclusão do Especializado \\ \hline
Hacker Lendário & A definir & Residência Tecnológica \\ \hline
\end{tabular}
\fonte{Adaptado de \citeonline{HackersDoBemManual}.}
\end{table}

É importante ressaltar que os emblemas possuem caráter motivacional e não garantem, isoladamente, a aprovação técnica. A aprovação em cada curso exige o cumprimento de requisitos acadêmicos específicos, como nota mínima de 60\% nas avaliações finais e nas atividades práticas obrigatórias \cite{HackersDoBemManual}.

\subsubsection{O Funil de Ranqueamento e Classificação}
\label{sss_funil_ranqueamento_classificacao}

Enquanto os cursos de Nivelamento e Básico operam em modelo assíncrono com vagas ilimitadas, a transição para o nível Fundamental marca o início do gargalo estrutural do programa. O acesso às aulas síncronas e laboratórios avançados é restrito a um número limitado de vagas (definido por "ondas" de oferta), exigindo a aplicação de um algoritmo de seleção.

O ranqueamento utiliza a pontuação acumulada como critério primário. Em cenários de empate, o manual estabelece uma hierarquia rigorosa de critérios de desempate para a seleção dos candidatos, priorizando a competência técnica sobre o consumo de conteúdo:

\begin{enumerate}
    \item Maior nota nas Atividades Práticas (laboratórios);
    \item Maior nota na Avaliação Final (prova teórica);
    \item Maior nota nos \textit{Quizzes} de fixação;
    \item Maior pontuação no consumo de conteúdo (\ac{XP} de engajamento).
\end{enumerate}

Por fim, o modelo de avaliação cria um ambiente onde o "erro" na avaliação prática penaliza o aluno no ranking final, impactando diretamente suas chances de acesso à Residência Tecnológica. A seleção ocorre em três chamadas (listas classificatórias), onde as vagas não preenchidas na primeira lista são redistribuídas para os candidatos subsequentes no ranking \cite{HackersDoBemManual}. Um resumo do fluxo de avaliação pode ser verificado na \autoref{fig:fluxo_progresso}.

\begin{figure}[htb]
    \centering
    \caption{Fluxo de avaliação do programa}
    \label{fig:fluxo_progresso}
    \includegraphics[width=0.8\textwidth]{figs/fluxo_progresso.png}
    \fonte{Elaborada pela autora}
\end{figure}
\chapter{Trabalhos Relacionados}
\label{c_trabalhos_relacionados}

Neste capítulo, são apresentados os trabalhos relacionados que forneceram a base teórica para o desenvolvimento desta pesquisa. O objetivo foi identificar o estado da arte atual sobre programas de capacitação em cibersegurança com participação do público feminino, bem como as estratégias empregadas para ampliar o ingresso e a permanência de talentos femininos. A partir desta análise, busca-se evidenciar as lacunas que este trabalho pretende preencher ao investigar o cenário do programa \textit{Hackers} do Bem. O restante do capítulo descreve a metodologia utilizada para a seleção das obras e discute os principais achados.

\section{Metodologia}
\label{s_metodologia}

Para a condução deste estudo, adotou-se uma abordagem metodológica baseada em uma \ac{RSL}. O processo de seleção e análise dos trabalhos seguiu as diretrizes do método \ac{PRISMA}, a fim de garantir a transparência, a qualidade e a replicabilidade da pesquisa bibliográfica

As etapas estabelecidas para a revisão incluíram a definição do protocolo de busca, a identificação das fontes de informação e a aplicação de critérios de inclusão e exclusão. O objetivo da revisão foi viabilizar a extração de dados qualitativos e quantitativos, bem como obter subsídios na literatura para a fundamentação teórica.

\subsection{Pesquisa bibliográfica}
\label{s_s_pesquisa_bibliografica}

Esta etapa consistiu no mapeamento do estado da arte sobre iniciativas de treinamento em cibersegurança com foco na participação de mulheres. Para a recuperação dos trabalhos, foram consultadas quatro bases de dados:

\begin{itemize}
    \item \hyperlink{https://ieeexplore.ieee.org/Xplore/home.jsp}{IEEE Xplore}
    \item \hyperlink{https://www.elsevier.com/pt-br/products/scopus/search}{Scopus} 
    \item \hyperlink{https://scholar.google.com/}{Google Scholar}
    \item \hyperlink{https://dl.acm.org/}{ACM}
\end{itemize}

A estratégia de busca foi construída com base no \ac{PICO}. Os termos foram definidos para abranger conceitos de gênero, ações de capacitação em cibersegurança e estratégias voltadas à ampliação do ingresso e da permanência de talentos femininos, conforme descrito no quadro a seguir:

\begin{quadro}[htb]
\centering
\caption{Estratégia PICO utilizada na revisão sistemática}
\label{qua:estrategia_pico}
\begin{footnotesize}
\begin{tabular}{|l|p{10cm}|}
    \hline
    \textbf{Elemento} & \textbf{Definição na Pesquisa} \\
    \hline
    \textbf{População} & Meninas, mulheres, estudantes e profissionais em cibersegurança. \\
    \hline
    \textbf{Intervenção} & Projetos, programas, ações de inclusão, capacitação, mentoria, oficinas, eventos e \textit{hackathons}. \\
    \hline
    \textbf{Comparação} & Projetos em cibersegurança (gerais/com recorte de gênero). \\
    \hline
    \textbf{Resultado (\textit{Outcome})} & Aumento do número de ações voltadas às mulheres na área de cibersegurança. \\
    \hline
\end{tabular}
\end{footnotesize}
\fonte{Elaborado pela autora}
\end{quadro}

Considerando os três termos usados na estratégia \ac{PICO}, foram construídas três conjuntos de palavras:

\begin{itemize}
    \item \textit{(\enquote{women} OR \enquote{girls} OR \enquote{female} OR \enquote{gender})}
    \item \textit{(\enquote{cybersecurity} OR \enquote{digital security} OR \enquote{information security} OR \enquote{cyber security})}
    \item \textit{(\enquote{inclusion} OR \enquote{outreach} OR \enquote{mentor} OR \enquote{train} OR \enquote{empower} OR \enquote{project} OR \enquote{hackathon} OR \enquote{workshop} OR \enquote{bootcamp})}
\end{itemize}

Resultando por fim em uma \textit{string} de busca maior, formada pela união desses três grupos:

\textit{(\enquote{women} OR \enquote{girls} OR \enquote{female} OR \enquote{gender}) AND (\enquote{cybersecurity} OR \enquote{digital security} OR \enquote{information security} OR \enquote{cyber security}) AND (\enquote{inclusion} OR \enquote{outreach} OR \enquote{mentor} OR \enquote{train} OR \enquote{empower} OR \enquote{project} OR \enquote{hackathon} OR \enquote{workshop} OR \enquote{bootcamp})}

Para o refinamento dos resultados, além da \textit{string} de busca, foram incluídos outros filtros nos motores de busca. O levantamento deveria retornar apenas livros, artigos completos publicados em periódicos ou eventos, entre o período de 2020 a 2025. Após a execução nas bases de dados, todos os metadados dos artigos foram exportados para a ferramenta Rayyan\footnote{\url{https://www.rayyan.ai/}}, que foi utilizada para gerenciar quais artigos deveriam ser incluídos nas etapas subsequentes da \ac{RSL}, bem como incluir critérios de aceito e exclusão, eliminação de referências duplicadas e incompletas.

Na etapa de Identificação, a execução inicial da string de busca retornou 661 registros, distribuídos entre IEEE Xplore (342), Google Scholar (50), Scopus (54) e ACM (215). Após a importação para a ferramenta Rayyan e a remoção de 16 duplicatas, os trabalhos foram submetidos a três fases de seleção. A primeira fase consistiu na Triagem por Título, na qual cada trabalho foi avaliado quanto à sua disponibilidade integral e à aderência temática inicial, buscando-se responder às seguintes perguntas:

\begin{itemize}
    \item O artigo completo está disponível?
    \item O artigo aborda treinamento em cibersegurança ou a inclusão de mulheres no setor?
\end{itemize}

Esta fase resultou na seleção de 114 artigos. Na sequência, procedeu-se à Triagem por Resumo (segunda fase), com o objetivo de verificar a conexão direta entre o ensino de cibersegurança e estratégias de inclusão ou análise de gênero. Nesta etapa, descartaram-se estudos puramente técnicos, focados em gestão sem viés educacional ou que não apresentassem dados com recorte de gênero, restando 32 artigos elegíveis. Utilizou-se como critério de inclusão a seguinte questão:

\begin{itemize}
    \item O artigo relata alguma estratégia utilizada para ampliar o ingresso e a permanência de talentos femininos em programas de treinamento em cibersegurança?
\end{itemize}

Por fim, a fase de Elegibilidade compreendeu a leitura na íntegra (Full Text Screening) dos 32 textos selecionados. Nesta fase final, foram excluídos estudos sem dados empíricos ou relatos de experiência sem validação. Ao concluir o processo, o portfólio bibliográfico final foi composto por 9 artigos, que fundamentam as análises desta revisão. A \autoref{tab:selecao_artigos} detalha a redução do volume de trabalhos em cada base de dados ao longo do funil de seleção.

\begin{table}[htb]
\centering
\ABNTEXfontereduzida
\IBGEtab{%
  \caption{Processo de seleção dos artigos}
  \label{tab:selecao_artigos}
}{%
  \begin{tabular}{l|c|c|c|c}
  \toprule
  \textbf{Base de dados} & \textbf{Identificação} & \textbf{Primeira fase} & \textbf{Segunda fase} & \textbf{Seleção final} \\
  \midrule
Google Scholar & 50  & 29 & 6  & 1 \\
Scopus         & 54  & 23 & 5  & 1 \\
IEEE Xplore    & 342 & 32 & 11 & 1 \\
ACM            & 215 & 30 & 10 & 6 \\
  \midrule
  \textbf{Total} & \textbf{661} & \textbf{114} & \textbf{32} & \textbf{9} \\
  \bottomrule
  \end{tabular}%
}{%
  \fonte{Elaborada pela autora}
}
\end{table}

A predominância de artigos selecionados provenientes da base ACM (cerca de 67\% da seleção final) reflete a concentração de publicações sobre educação em computação (\textit{Computing Education}) nesta comunidade. Destacam-se conferências como SIGCSE TS, ITiCSE e Koli Calling, além de outras conferências de segurança como a ARES, demonstrando que a discussão sobre inclusão de gênero em cibersegurança está fortemente inserida nos fóruns de educação em computação.

\subsection{Panorama da Seleção Final}
\label{ss_panorama_selecao_final}

Para avaliar a atualidade e a relevância dos estudos selecionados, foi realizada uma análise temporal das publicações. A \autoref{fig:timeline} ilustra a distribuição dos trabalhos por ano de publicação. É possível observar uma concentração expressiva de trabalhos no ano de 2025, indicando que a discussão sobre a intersecção entre inclusão de gênero e metodologias de ensino em cibersegurança é um tema emergente e de alta relevância no cenário acadêmico atual.

\begin{figure}[htb]
    \centering
    \caption{Distribuição temporal dos trabalhos selecionados}
    \label{fig:timeline}
    \includegraphics[width=0.8\textwidth]{figs/distribuicao-temporal.png}
    \fonte{Elaborada pela autora}
\end{figure}

A seleção final demonstra que a discussão sobre escala e intervenções específicas de gênero teve um salto de publicações recentemente. Isso ajuda a validar a justificativa desta pesquisa, pois o programa \textit{Hackers} do Bem opera neste contexto temporal.

\subsection{Análise Multidimensional dos Trabalhos}
\label{ss_analise_multidimensional_trabalhos}

Com o objetivo de compreender as contribuições individuais e as limitações de cada estudo em relação ao objeto desta dissertação, especialmente ao considerar a estrutura de programas massivos como o \textit{Hackers} do Bem, os trabalhos foram avaliados sob cinco dimensões críticas derivadas diretamente das questões de pesquisa:

\begin{enumerate}
    \item \textbf{Foco em Gênero e Barreiras (QP1):} Profundidade da análise sobre os fatores socioculturais e estruturais determinantes para a baixa adesão feminina, bem como as intervenções específicas adotadas para mitigar a evasão. \item \textbf{Abordagens Pedagógicas (QP2):} Avaliação das estratégias de ensino utilizadas, tais como gamificação, laboratórios práticos e aprendizado ativo, com foco no impacto dessas abordagens na adesão e permanência do público feminino.
    \item \textbf{Colaboração e Autoeficácia (QP3):} Análise da implementação de mecanismos colaborativos, mentoria e customização de trilhas que influenciam a percepção de autoeficácia e a intenção de permanência das estudantes.
    \item \textbf{Escalabilidade e Alcance:} Viabilidade de aplicação das estratégias em programas de capacitação em larga escala e em ambientes distribuídos ou online, alinhando-se à característica massiva do \textit{Hackers} do Bem.
    \item \textbf{Validação Empírica:} Presença de dados reais e rigor metodológico na validação das intervenções propostas, distinguindo propostas teóricas de métodos efetivamente testados.
\end{enumerate}

Para consolidar a análise comparativa, os nove trabalhos selecionados foram avaliados qualitativamente e receberam atribuição de notas em uma escala de 1 (insuficiente ou não aborda) a 5 (excelente ou foco central) nas cinco dimensões estabelecidas. Esta pontuação reflete a aderência de cada estudo às questões de pesquisa (QP1, QP2 e QP3), bem como sua robustez metodológica e capacidade de escala. O \autoref{quadro:notas_radar} detalha a pontuação atribuída a cada trabalho:

\clearpage

\begin{quadro}[htb]
\centering
\caption{Avaliação multidimensional dos trabalhos selecionados (Escala 1-5)}
\label{quadro:notas_radar}
\ABNTEXfontereduzida
\begin{tabular}{|l|c|c|c|c|c|}
\hline
\textbf{Trabalho} & \textbf{\shortstack{Gênero\\(QP1)}} & \textbf{\shortstack{Pedagogia\\(QP2)}} & \textbf{\shortstack{Colab.\\(QP3)}} & \textbf{\shortstack{Escala-\\bilidade}} & \textbf{\shortstack{Vali-\\dação}} \\ \hline
\citeonline{Costa2025} & 5 & 5 & 3 & 4 & 5 \\ \hline
\citeonline{Musuva2025} & 4 & 3 & 5 & 3 & 4 \\ \hline
\citeonline{Tshekiso2025} & 3 & 2 & 3 & 5 & 4 \\ \hline
\citeonline{Benson2025} & 5 & 2 & 5 & 3 & 3 \\ \hline
\citeonline{Costa2023} & 5 & 5 & 4 & 3 & 4 \\ \hline
\citeonline{Casey2023} & 4 & 5 & 4 & 2 & 4 \\ \hline
\citeonline{Thomas2024} & 4 & 5 & 2 & 2 & 3 \\ \hline
\citeonline{Rahman2022} & 5 & 3 & 4 & 3 & 4 \\ \hline
\citeonline{Hogan2025} & 2 & 3 & 5 & 5 & 4 \\ \hline
\end{tabular}
\fonte{Elaborado pela autora}
\end{quadro}

A projeção visual destes dados, apresentada na \autoref{fig:radar}, permite a identificação dos pontos de convergência e divergência entre os estudos. O gráfico de radar evidencia a formação de eixos temáticos específicos.

\begin{figure}[htb]
    \centering
    \caption{Comparativo multidimensional dos trabalhos selecionados}
    \label{fig:radar}
    \includegraphics[width=0.8\textwidth]{figs/analise-multidimensional.png}
    \fonte{Elaborada pela autora}
\end{figure}

\subsection{Foco em Gênero e Barreiras (QP1)}
\label{ss_foco_genero}

A primeira dimensão do gráfico de radar, \textbf{Foco em Gênero (QP1)}, avalia a profundidade com que cada estudo analisa as barreiras socioculturais específicas para mulheres e propõe intervenções desenhadas exclusivamente para este público. Neste quesito, destacam-se os trabalhos de \citeonline{Costa2025} e \citeonline{Benson2025}, que obtiveram a pontuação máxima. O estudo de \citeonline{Costa2025}, referente ao programa \textit{CyberTrials}, diferencia-se por não apenas incluir mulheres, mas por estruturar todo o ambiente de aprendizado para combater estereótipos de gênero, alcançando 779 alunas com uma abordagem gamificada inclusiva que mitiga a percepção de masculinidade associada à área.

De forma complementar, \citeonline{Benson2025} foca nas barreiras de ascensão profissional, estabelecendo as \textit{soft skills} e a mentoria como ferramentas essenciais para a permanência feminina na carreira. \citeonline{Rahman2022} foca no nicho de mulheres em fase de reingresso no mercado de trabalho (\textit{re-entry}), propondo intervenções específicas para recuperar a confiança técnica. Em contrapartida, estudos como o de \citeonline{Hogan2025}, embora relevantes para a dinâmica de grupos, pontuaram menos nesta categoria por analisarem competições com times mistos sem uma intervenção desenhada primariamente para o recorte de gênero.

\subsection{Abordagens Pedagógicas (QP2)}
\label{ss_abordagens_pedagogicas}

No que tange às \textbf{Abordagens Pedagógicas (QP2)}, a análise priorizou trabalhos que detalham metodologias instrucionais capazes de reduzir a carga cognitiva e aumentar o engajamento de iniciantes. As pontuações mais altas foram atribuídas a \citeonline{Costa2023} e \citeonline{Costa2025}, devido à implementação bem-sucedida de narrativas (\textit{storytelling}) e \ac{RPG}. Nestes estudos, os autores demonstram que transformar exercícios técnicos em missões investigativas aumenta significativamente o interesse do público feminino, alterando a percepção da cibersegurança de uma área puramente técnica para uma voltada à resolução de problemas sociais.

Igualmente relevantes, \citeonline{Casey2023} e \citeonline{Thomas2024} destacam-se pela aplicação de \textit{scaffolding} (andaimes cognitivos) e currículos baseados em problemas reais (\textit{Problem-Based Learning}). Na Teoria Sociointeracionista de Lev Vygotsky, \textit{scaffolding} (andaimes cognitivos) é o suporte temporário e ajustável oferecido por um instrutor (ou uma ferramenta) para ajudar o estudante a realizar uma tarefa que ele ainda não consegue completar sozinho. \citeonline{Thomas2024}, especificamente, detalha como a decomposição de tarefas complexas em etapas menores é vital para evitar a frustração em grupos sub-representados. Trabalhos focados primariamente em gestão ou infraestrutura, como os de \citeonline{Tshekiso2025}, receberam pontuações menores nesta dimensão por utilizarem plataformas sem detalhar modificações curriculares profundas no escopo do artigo.

\subsection{Colaboração e Autoeficácia (QP3)}
\label{ss_colaboracao_autoeficacia}

A dimensão de \textbf{Colaboração e Autoeficácia (QP3)} examina a presença de mecanismos que fomentam o trabalho em equipe, o suporte de pares e a mentoria, fatores importantes para a retenção em programas massivos onde o isolamento é um risco. \citeonline{Hogan2025} apresenta a contribuição mais significativa neste aspecto, recebendo nota máxima ao investigar como times distribuídos geograficamente constroem confiança e colaboram eficazmente em ambientes online, uma descoberta diretamente aplicável ao modelo do \textit{Hackers} do Bem. Similarmente, \citeonline{Musuva2025} e \citeonline{Benson2025} são fundamentais por posicionarem a mentoria e o \textit{networking} não como acessórios, mas como pilares centrais para a construção da autoeficácia das participantes. \citeonline{Musuva2025}, no contexto do programa \textit{Cyber Shujaa}, demonstra que a conexão direta com mentores da indústria valida a identidade profissional das alunas, combatendo a síndrome do impostor.

\subsection{Escalabilidade e Alcance}
\label{ss_escalabilidade_alcance}

Em relação à \textbf{Escalabilidade}, buscou-se identificar estratégias viáveis para programas de abrangência nacional ou continental. O trabalho de \citeonline{Tshekiso2025} sobressai-se com pontuação máxima ao descrever a implementação de competições em diversos países africanos, demonstrando como o uso de plataformas abertas e agnósticas de infraestrutura permite escalar o ensino de cibersegurança em regiões com recursos limitados. \citeonline{Hogan2025} e \citeonline{Costa2025} também apresentam alta escalabilidade devido ao uso de ambientes virtuais e modelos híbridos que suportam centenas ou milhares de alunos simultâneos sem a necessidade de laboratórios físicos complexos. Por outro lado, intervenções como as de \citeonline{Casey2023} e \citeonline{Thomas2024}, embora pedagogicamente ricas, baseiam-se em \textit{workshops} presenciais (como acampamentos de verão) ou estudos de caso locais com turmas reduzidas, apresentando desafios maiores para uma replicação massiva imediata no contexto de um programa nacional.

\subsection{Validação Empírica}
\label{ss_validacao_empirica}

Por fim, a categoria de \textbf{Validação Empírica} ponderou o rigor metodológico e a presença de dados quantitativos ou qualitativos que sustentem as conclusões dos autores. \citeonline{Costa2025} estabelece o padrão de referência nesta seleção, apresentando resultados estatisticamente significantes derivados de grupos de controle e testes pré e pós-intervenção com uma amostra expressiva de participantes. \citeonline{Casey2023} também contribui com dados longitudinais coletados ao longo de três anos, oferecendo uma visão sólida sobre a eficácia de currículos inclusivos na mudança de percepção de carreira a longo prazo. 

\citeonline{Musuva2025} valida seu modelo através de métricas concretas de empregabilidade e taxas de conclusão. Estudos classificados com pontuação intermediária nesta dimensão, como \citeonline{Benson2025} e \citeonline{Thomas2024}, apoiam-se majoritariamente em análises qualitativas ou revisões de literatura que, apesar de oferecerem \textit{insights} valiosos sobre o comportamento e as barreiras subjetivas, carecem da validação estatística de larga escala presente nos demais trabalhos selecionados.

\section{Trabalhos Selecionados}
\label{s_trabalhos_selecionados}

As próximas subseções descrevem as características principais, metodologias e as contribuições específicas de cada um dos trabalhos selecionados.

\subsection{Gamificação Narrativa e Redução de Ansiedade em CyberTrials} \label{ss_costa_2025}

O trabalho de \citeonline{Costa2025} apresenta o desenho e a validação empírica do programa \textit{CyberTrials}, uma iniciativa realizada na Itália voltada para estudantes do ensino médio. Diferentemente de competições tradicionais de \ac{CTF}, que frequentemente priorizam a velocidade e o conhecimento técnico prévio, fatores que a literatura aponta como barreiras para iniciantes (QP1), este estudo propõe uma abordagem pedagógica baseada em gamificação narrativa e \ac{RPG}. O objetivo central dos autores foi investigar se o uso de metáforas lúdicas e ambientes colaborativos poderia mitigar a ansiedade tecnológica e aumentar a autoeficácia das participantes em tópicos avançados, como criptografia, segurança web e \ac{OSINT}.

Em relação aos resultados, \citeonline{Costa2025} demonstram, através de grupos de controle e pré/pós-testes, que a intervenção aumentou significativamente a intenção das participantes em seguir carreiras nas áreas de \ac{STEM} e cibersegurança. Para o desenvolvimento do framework de diretrizes desta dissertação, a contribuição crucial deste estudo reside na validação da gamificação estrutural (QP2): ao envolver os desafios técnicos em uma narrativa de investigação (onde as alunas atuam como \enquote{detetives} em vez de \enquote{hackers}), o programa reduziu a barreira de entrada e o estereótipo masculino associado à área. Além disso, o estudo confirma a hipótese da QP3, evidenciando que a formação de times e a colaboração intrínseca ao jogo foram determinantes para a manutenção do engajamento, contrastando com o isolamento comum em cursos massivos autodidatas.

Apesar dos resultados positivos, o trabalho apresenta limitações que devem ser consideradas na adaptação para outros programas. Primeiramente, a validação foca na mudança de intenção de carreira a curto prazo, carecendo de dados longitudinais que confirmem se essa motivação se traduz em ingresso efetivo no mercado de trabalho. Ademais, embora o modelo híbrido/online descrito suporte escalabilidade (atingindo cerca de 1.000 alunas), o estudo não detalha profundamente como a mentoria personalizada pode ser sustentada em programas de escala massiva (dezenas de milhares de alunos) sem incorrer em custos proibitivos, um desafio central para a QP2 em políticas públicas nacionais.

\subsection{Mentoria e Empregabilidade no Programa \textit{Cyber Shujaa}} \label{ss_musuva_2025}

O estudo de \citeonline{Musuva2025} detalha a implementação e os resultados do programa \textit{Cyber Shujaa}, uma iniciativa desenvolvida no Quênia para combater o desemprego juvenil e a escassez de mão de obra qualificada em cibersegurança. Diferentemente de cursos puramente técnicos, este programa adota uma abordagem holística que integra capacitação prática, imersão (residência) e suporte direto para colocação no mercado de trabalho. No contexto da QP3, o trabalho é fundamental por demonstrar que a formação técnica isolada é insuficiente para garantir a transição de carreira; a colaboração estruturada entre a academia e a indústria, materializada através de mentorias intensivas e feiras de carreira, provou ser o fator determinante para a retenção e o sucesso profissional dos participantes.

Em termos de resultados, o programa alcançou uma taxa de participação feminina de 41\%, um número expressivo obtido através de estratégias de recrutamento afirmativo e parcerias com comunidades locais. Para o \textit{framework} de diretrizes proposto nesta dissertação, a contribuição central de \citeonline{Musuva2025} reside na validação da mentoria de carreira como um mecanismo de redução da evasão. O estudo indica que o acompanhamento próximo por profissionais da indústria não apenas desenvolve competências técnicas, mas também fortalece as \textit{soft skills} e a identidade profissional das alunas, mitigando a síndrome do impostor frequentemente relatada por mulheres na área (QP1). A estrutura do programa sugere que a criação de pontes claras com o mercado de trabalho aumenta a percepção de utilidade do curso, incentivando a permanência.

Contudo, o trabalho apresenta limitações relacionadas à escalabilidade (QP2). O modelo do \textit{Cyber Shujaa} depende fortemente de interações presenciais ou híbridas e de um alto número de mentores humanos para um grupo relativamente menor de alunos, o que representa um desafio logístico e financeiro para programas nacionais que visam atingir dezenas de milhares de estudantes simultaneamente. Portanto, o desafio para o \textit{framework} será adaptar esses mecanismos de \enquote{alta interação} descritos pelo estudo para um ambiente virtual massivo, possivelmente através de sistemas de mentoria em pares ou comunidades de prática online, sem perder a qualidade do suporte humano que os autores identificaram como crucial.

\subsection{Escalabilidade Continental via Plataformas Abertas: O Caso \textit{picoCTF-Africa}}
\label{ss_tshekiso_2025}

O trabalho de \citeonline{Tshekiso2025} documenta a implementação e os desafios da iniciativa \textit{picoCTF-Africa}, um programa de capacitação desenhado para mitigar a escassez crítica de força de trabalho em cibersegurança no continente africano. Diferentemente de intervenções locais ou de curta duração, o objetivo central deste estudo foi estabelecer um modelo de escalabilidade sustentável que pudesse operar em múltiplos países simultaneamente, superando barreiras de infraestrutura e custos. O contexto da pesquisa envolve a adaptação da plataforma global de competições \textit{picoCTF} para o cenário regional, integrando escolas de ensino médio e universidades através de uma estrutura hierárquica de capítulos (\textit{chapters}) locais. Para a \textbf{QP2}, este estudo é particularmente relevante por demonstrar como o uso de ambientes de treinamento gamificados e agnósticos de \textit{hardware} permite a democratização do acesso ao conhecimento técnico avançado em regiões com recursos limitados.

Em relação aos resultados, os autores relatam um crescimento expressivo na participação, saltando de números incipientes para milhares de estudantes engajados em competições anuais, validando a eficácia da estratégia de descentralização via (\textit{chapters}) locais. Uma contribuição vital para o \textit{framework} de diretrizes desta dissertação reside na abordagem de comunidades de prática distribuídas (QP3): o estudo evidencia que a criação de clubes locais de cibersegurança, liderados por embaixadores regionais, foi essencial para manter a motivação dos alunos e fornecer o suporte técnico inicial que a plataforma online sozinha não supriria. Além disso, a iniciativa implementou uma categoria específica para mulheres e meninas como ação afirmativa, reconhecendo que a competição aberta, por si só, não garante a equidade de gênero, alinhando-se às preocupações da \textbf{QP1} sobre barreiras estruturais.

Entretanto, ao analisar o trabalho sob a ótica da customização pedagógica, nota-se uma limitação importante. Embora \citeonline{Tshekiso2025} tenham alcançado sucesso em escala, a metodologia pedagógica baseia-se fundamentalmente no modelo padrão de \ac{CTF} estilo \textit{Jeopardy}\footnote{Um \ac{CTF} do tipo Jeopardy é um modelo de competição de cibersegurança onde os participantes ou equipes devem resolver uma série de desafios técnicos organizados por categorias e níveis de dificuldade. Diferente do modelo Attack-Defense (onde as equipes atacam umas às outras), no Jeopardy os competidores enfrentam uma plataforma centralizada que disponibiliza \enquote{cartões} de questões. Cada desafio resolvido fornece uma sequência de caracteres única, chamada de \textit{flag} (bandeira), que é submetida ao sistema em troca de pontos.}, sem modificações profundas no desenho instrucional para torná-lo intrinsecamente mais inclusivo ou menos intimidante para iniciantes absolutos. O estudo foca mais na logística de expansão e na infraestrutura de competição do que na adaptação curricular ou em estratégias de \textit{scaffolding} cognitivo para retenção de longo prazo. Essa lacuna reforça a necessidade de o \textit{framework} proposto nesta pesquisa ir além da disponibilização massiva de desafios, integrando narrativas e trilhas de aprendizado adaptativas que sustentem a permanência do público feminino para além do evento competitivo inicial.

\subsection{Soft Skills e Mentoria como Estratégias de Retenção} \label{ss_benson_2025}

O trabalho de \citeonline{Benson2025} investiga o papel crítico das competências não técnicas (\textit{soft skills}) na permanência e ascensão de mulheres na carreira de cibersegurança. Diferentemente de abordagens focadas exclusivamente na capacitação técnica, este estudo qualitativo utiliza a Teoria Social Cognitiva de Carreira para analisar como atributos como resiliência, comunicação e liderança atuam como fatores de proteção contra barreiras estruturais e o isolamento profissional, respondendo diretamente à QP1. O objetivo central dos autores foi identificar quais competências comportamentais e redes de suporte são percebidas pelas próprias mulheres como determinantes para a superação de inseguranças e para a manutenção da identidade profissional em um ambiente predominantemente masculino.

Em seus resultados, \citeonline{Benson2025} identificaram quatro pilares temáticos essenciais: confiança e resiliência, mentoria e patrocínio (\textit{sponsorship}), colaboração na resolução de problemas e políticas inclusivas. Para a construção do \textit{framework} de diretrizes desta dissertação, a contribuição mais significativa reside na validação da colaboração e mentoria (QP3) como mecanismos indispensáveis de retenção. O estudo demonstra que ambientes de aprendizado que fomentam a resolução colaborativa de problemas (\textit{collaborative problem-solving}) não apenas desenvolvem habilidades técnicas, mas criam um senso de comunidade que ajuda a mitigar a síndrome do impostor. Os dados indicam que a inclusão de programas formais de mentoria e \textit{workshops} colaborativos é tão vital quanto o conteúdo técnico para garantir a autoeficácia das participantes, sugerindo que a estrutura do curso deve prever espaços seguros para o desenvolvimento destas competências interpessoais.

Entretanto, a aplicação direta destes achados no contexto de programas massivos apresenta desafios significativos relacionados à escalabilidade. O estudo de \citeonline{Benson2025} baseia-se em entrevistas de profundidade e dinâmicas interpessoais ricas, cuja replicação em um ambiente virtual com milhares de alunos exige adaptações complexas. A principal lacuna do trabalho, sob a ótica da QP2, é a ausência de uma proposta curricular técnica integrada; o foco recai quase inteiramente sobre aspectos comportamentais, sem detalhar como essas \textit{soft skills} podem ser pedagogicamente embutidas nos exercícios técnicos de um curso \ac{EAD}. Portanto, o desafio para o \textit{framework} será operacionalizar a \enquote{mentoria} e a \enquote{colaboração} descritas pelos autores de forma assíncrona ou escalável, sem depender exclusivamente da interação humana síncrona intensiva que é financeiramente inviável em larga escala.

\subsection{Narrativa e Engajamento Feminino: O Caso \textit{Why Mary Can Hack}} \label{ss_costa_2023}

O trabalho de \citeonline{Costa2023} aborda a sub-representação feminina na área de cibersegurança através de uma intervenção pedagógica denominada \textit{CyberTrials}, voltada para estudantes do ensino médio. O contexto da pesquisa parte da premissa de que a baixa adesão feminina (QP1) não decorre da falta de habilidade técnica, mas sim de estereótipos culturais e da forma como o conteúdo é tradicionalmente apresentado. O objetivo central dos autores foi investigar se a introdução de técnicas de \textit{storytelling} (narrativa) e gamificação não competitiva poderia alterar a percepção das alunas sobre a área, transformando exercícios técnicos em missões de investigação com propósito social.

Em termos de resultados, o estudo demonstrou que a contextualização dos desafios técnicos dentro de uma narrativa de mistério (onde as alunas atuam como investigadoras para resolver um caso de \textit{ransomware}\footnote{Um tipo de \textit{software} malicioso que sequestra os dados de um dispositivo por meio de criptografia. Uma vez infectado, os arquivos tornam-se ilegíveis para o usuário. Os criminosos então exigem o pagamento de um resgate para fornecer a chave que desbloqueia os dados.}) aumentou significativamente o engajamento e a autoeficácia das participantes. Para a construção do \textit{framework} de diretrizes desta dissertação, a contribuição fundamental de \citeonline{Costa2023} para a QP2 é a validação de que a \enquote{camada de apresentação} do conteúdo é tão importante quanto o conteúdo em si. O uso de metáforas lúdicas e a conexão dos laboratórios práticos com problemas do mundo real permitiram que as alunas superassem a barreira da ansiedade tecnológica, vendo a ferramenta técnica como um meio para um fim socialmente relevante, e não como um fim em si mesmo.

Apesar do sucesso da intervenção, o trabalho apresenta limitações quando contrastado com a escala massiva de programas como o \textit{Hackers} do Bem. O modelo descrito em \citeonline{Costa2023} beneficiou-se de uma interação síncrona e monitorada, com forte componente humano de suporte, o que é difícil de replicar em cursos autoinstrucionais com milhares de alunos. Além disso, embora a narrativa tenha sido eficaz para a atração inicial, o estudo não aprofunda como essa estratégia de \textit{storytelling} pode ser sustentada em níveis avançados de formação técnica, onde a complexidade dos conceitos exige maior abstração e rigor. O desafio para o \textit{framework} será, portanto, adaptar os princípios de narrativa inclusiva para trilhas de aprendizado assíncronas, criando um senso de propósito e contexto sem depender da mediação humana constante.

\subsection{Aprendizado Baseado em Problemas e Contexto Social no \textit{Cyber Sleuth Science Lab}}
\label{ss_casey_2023}

O trabalho de \citeonline{Casey2023} apresenta o desenvolvimento e a validação do \ac{CSSL}, um ambiente de aprendizagem inteligente desenhado especificamente para aumentar a participação de grupos sub-representados, com ênfase no público feminino, em carreiras de tecnologia. Diferenciando-se de cursos tradicionais que focam na aquisição de habilidades técnicas isoladas (como configuração de \textit{firewalls} ou codificação de \textit{scripts}), a abordagem pedagógica adotada fundamenta-se no \ac{PBL} e na narrativa investigativa. O objetivo central dos autores foi contextualizar a cibersegurança e a forense digital dentro de problemas do mundo real que possuem relevância social imediata para os estudantes, como o combate ao \textit{cyberbullying} e a recuperação de dados perdidos, alterando a percepção da área de uma disciplina puramente técnica para uma ferramenta de justiça social (QP1).

Em relação aos resultados, o estudo demonstrou que a utilização de casos práticos guiados por uma estrutura de \enquote{andaimes cognitivos} (\textit{scaffolding}) aumentou significativamente a autoeficácia das participantes. Para a construção do \textit{framework} de diretrizes desta dissertação, a contribuição vital de \citeonline{Casey2023} para a QP2 e QP3 reside na demonstração de que a redução da carga cognitiva inicial, obtida através de guias passo a passo e decomposição de tarefas complexas, é essencial para evitar a frustração em iniciantes. Além disso, o estudo validou que a colaboração em pequenos grupos e a presença de modelos femininos (\textit{role models}) atuando como facilitadoras foram determinantes para que as meninas se sentissem pertencentes ao ambiente, sugerindo que mecanismos colaborativos não são opcionais, mas sim componentes estruturais da retenção.

O sucesso das intervenções descritas por \citeonline{Casey2023} dependeu fortemente de mediação humana síncrona, com instrutores e facilitadores presentes em sala de aula ou em sessões remotas intensivas para guiar a discussão e prover suporte emocional e técnico imediato. Em um programa massivo e assíncrono, replicar esse nível de \enquote{toque humano} é financeiramente e logisticamente complexo. Portanto, a lacuna que o \textit{framework} proposto precisará preencher é como simular esse \textit{scaffolding} e esse senso de colaboração investigativa utilizando recursos automatizados da plataforma e comunidades de prática, sem depender da proporção de instrutor-aluno utilizada no estudo de caso descrito pelo estudo.

\subsection{Adaptação Curricular e \textit{Scaffolding} para Populações Sub-representadas}
\label{ss_thomas_2024}

O estudo de \citeonline{Thomas2024} aborda a implementação de currículos de cibersegurança voltados para estudantes do ensino fundamental provenientes de comunidades sub-representadas e economicamente desfavorecidas. Diferentemente de treinamentos corporativos padronizados, o contexto desta pesquisa foca na equidade do acesso (QP1), investigando como as barreiras de letramento digital e a complexidade técnica inerente à área podem ser mitigadas através de intervenções pedagógicas intencionais. O objetivo central dos autores foi avaliar a eficácia de um currículo ajustado que utiliza atividades práticas (\textit{hands-on}) e a redução da carga teórica expositiva para manter o engajamento de alunos com diferentes níveis de proficiência prévia.

Em relação aos resultados, \citeonline{Thomas2024} identificaram que a estrutura tradicional de aulas expositivas gerava desengajamento rápido, sendo necessário limitar explanações teóricas a curtos intervalos de tempo, intercalados com experimentação prática. Para a construção do \textit{framework} de diretrizes desta dissertação, a contribuição fundamental deste trabalho para a QP2 é a validação da necessidade de \textit{scaffolding} (andaimes cognitivos) multinível. Os autores demonstram que, para garantir a permanência de grupos heterogêneos, é crucial oferecer \enquote{extensões de lições} para alunos avançados e suportes visuais ou pictóricos para aqueles com dificuldades de leitura ou barreiras linguísticas. Isso sugere que o \textit{design} instrucional deve ser flexível o suficiente para acomodar o ritmo individual, evitando que a frustração técnica inicial se converta em evasão.

Apesar da relevância pedagógica, a aplicação direta do modelo de \citeonline{Thomas2024} encontra limitações na dimensão da escalabilidade. A intervenção descrita baseou-se em acampamentos de verão presenciais (\textit{summer camps}) com forte mediação de instrutores que ajustavam o conteúdo em tempo real conforme a reação da turma. Em um programa massivo e predominantemente assíncrono, essa personalização humana intensiva é inviável. Portanto, a falha do trabalho em propor mecanismos de \textit{scaffolding} automatizado representa uma lacuna que o \textit{framework} proposto precisará preencher, traduzindo o suporte presencial descrito pelos autores em funcionalidades de plataforma e trilhas adaptativas que possam operar em larga escala sem intervenção manual constante.

\subsection{Iniciativas de Reingresso e Recuperação da Confiança Técnica}
\label{ss_rahman_2022}

O trabalho de \citeonline{Rahman2022} aborda um segmento crítico e frequentemente negligenciado nas políticas de inclusão: mulheres que buscam retornar à força de trabalho em computação após um período de afastamento (\textit{career break}). O contexto da pesquisa identifica que este grupo enfrenta barreiras distintas (QP1), como a defasagem técnica percebida e uma severa diminuição da autoeficácia profissional, exacerbada pelo ritmo acelerado de atualização das tecnologias de cibersegurança. O objetivo central dos autores é validar intervenções de \textit{re-entry} (reingresso) que combinam atualização técnica rápida com suporte emocional e reconstrução de identidade profissional, questionando a eficácia de cursos que ignoram a bagagem prévia e as inseguranças específicas deste público.

Em termos de contribuições para o \textit{framework} de diretrizes, o estudo é particularmente relevante para a \textbf{QP3} ao demonstrar que a recuperação da confiança técnica não ocorre em isolamento. \citeonline{Rahman2022} evidenciam que o elemento de \textit{networking} e a convivência com pares que compartilham a mesma trajetória de retorno são tão vitais quanto o conteúdo técnico em si. Isso sugere que a criação de comunidades ou trilhas específicas para alunas em transição de carreira ou retorno ao mercado pode ser um fator decisivo de retenção, mitigando a sensação de não pertencimento que ocorre quando estas alunas são colocadas em competição direta com jovens nativos digitais sem as mesmas responsabilidades familiares ou lacunas curriculares.

Contudo, a transposição deste modelo apresenta desafios de escalabilidade e validação massiva. A intervenção descrita baseia-se em \textit{workshops} presenciais ou síncronos de menor escala, típicos de conferências acadêmicas. Uma limitação do trabalho, sob a ótica da \textbf{QP2}, é a falta de detalhamento sobre como a pedagogia de \enquote{atualização rápida} pode ser automatizada em uma plataforma \ac{EAD} sem a mediação humana constante de mentores. O desafio para o \textit{framework} será, portanto, adaptar os princípios de acolhimento e validação de identidade descritos pelos autores para mecanismos escaláveis, como sistemas de recomendação de pares ou mentorias coletivas virtuais.

\subsection{Dinâmicas de Colaboração em Times Distribuídos e Confiança Rápida} \label{ss_hogan_2025}

O trabalho de \citeonline{Hogan2025} investiga os comportamentos e padrões de comunicação de equipes de alto desempenho em competições de cibersegurança do tipo \ac{CTF} realizadas em formato totalmente remoto. O contexto da pesquisa é particularmente relevante para o estudo de caso do \textit{Hackers} do Bem (QP2) pois, diferentemente de intervenções presenciais, este estudo analisa como a aprendizagem e a resolução de problemas ocorrem em ambientes virtuais distribuídos, onde a interação física é inexistente. O objetivo central dos autores foi identificar como grupos geograficamente dispersos constroem a chamada \enquote{confiança rápida} (\textit{swift trust}) e estabelecem lideranças emergentes para superar desafios técnicos complexos sem a supervisão direta de instrutores.

Em relação aos resultados, o estudo revela que o sucesso na retenção e na performance técnica não depende apenas da habilidade individual, mas da capacidade do grupo em realizar \textit{scaffolding} entre pares, ou seja, membros mais experientes suportando o aprendizado dos novatos em tempo real. Para o \textit{framework} de diretrizes desta dissertação, a contribuição de \citeonline{Hogan2025} para a QP3 é a constatação de que a colaboração online eficaz exige canais de comunicação que permitam não apenas a troca de dados técnicos, mas também a validação emocional e o gerenciamento de frustração. Os autores demonstram que a autoeficácia dos participantes é ampliada quando o ambiente virtual simula a camaradagem de um laboratório físico, sugerindo que plataformas de ensino massivo devem incorporar ferramentas de chat e formação de comunidades persistentes para mitigar o isolamento.

Apesar dos \textit{insights} valiosos sobre colaboração remota, o trabalho apresenta limitações importantes quando contrastado com o objetivo de inclusão feminina (QP1). A amostra do estudo é composta por times \enquote{altamente bem-sucedidos}, o que introduz um viés de sobrevivência; não há uma análise profunda sobre os times que falharam ou desistiram, onde provavelmente se encontram as barreiras que afetam desproporcionalmente as mulheres e iniciantes inseguros. Além disso, ao focar na observação de dinâmicas orgânicas em competições mistas, o trabalho não propõe intervenções estruturais para combater a toxicidade ou o domínio masculino nesses espaços, uma lacuna que o \textit{framework} proposto precisará preencher para garantir que a \enquote{colaboração} descrita não se torne um ambiente hostil para o público feminino.

\section{Comparação e Lacunas Identificadas}
\label{s_comparacao_lacunas_identificadas}

O \autoref{quadro:comparacao_trabalhos} sintetiza a comparação entre os nove trabalhos selecionados e a proposta desta dissertação. A análise da literatura revela que, embora existam iniciativas robustas em isolamento, há uma carência de modelos que integrem simultaneamente pedagogias de inclusão profunda (como narrativas e \textit{scaffolding}) com estratégias de larga escala (como plataformas \ac{EAD} massivas).

Enquanto trabalhos como os de \citeonline{Costa2025}, \citeonline{Casey2023} e \citeonline{Thomas2024} oferecem excelentes evidências sobre a eficácia de adaptações curriculares e gamificação narrativa para o público feminino (QP1 e QP2), eles operam majoritariamente em escalas locais ou regionais (\textit{workshops}, turmas controladas), limitando a generalização para políticas públicas nacionais. Por outro lado, estudos focados em alta escalabilidade, como \citeonline{Tshekiso2025} e \citeonline{Hogan2025}, abordam a infraestrutura e a colaboração em times distribuídos (QP3), mas frequentemente carecem de um recorte de gênero específico ou de intervenções pedagógicas desenhadas para mitigar a ansiedade tecnológica de iniciantes.

A lacuna central identificada reside entre escalabilidade massiva e customização da experiência de aprendizado. Embora \citeonline{Musuva2025} e \citeonline{Benson2025} evidenciem a eficácia de práticas de \enquote{alto toque humano}, como mentoria intensiva, acolhimento e desenvolvimento de \textit{soft skills}, para a retenção feminina (QP1 e QP3), não foi encontrado na literatura um modelo que operacionalize essas estratégias em ambientes assíncronos e de alcance nacional. Desta forma, esta dissertação propõe preencher essa lacuna utilizando o programa \textit{Hackers} do Bem como estudo de caso, com o objetivo de desenvolver um \textit{framework} de diretrizes que oriente a adaptação e a integração dessas abordagens inclusivas já validadas para o contexto de formação em larga escala.

\begin{quadro}[!htbp]
    \centering
    \caption{Comparação dos trabalhos relacionados com a proposta atual}
    \label{quadro:comparacao_trabalhos}
    \begin{footnotesize}
    \resizebox{\textwidth}{!}{%
    \begin{tabular}{|l|c|c|c|c|l|}
        \hline
        \rowcolor[HTML]{EFEFEF} 
        \textbf{Trabalho} & \textbf{Foco Gênero} & \textbf{Escalabilidade} & \textbf{Pedagogia Inclusiva} & \textbf{Colab./Mentoria} & \textbf{Limitação Principal (Gap)} \\ \hline
        
        \citeonline{Costa2025}    & Alta  & Média & Alta  & Média & Intervenção pontual, falta dados de carreira a longo prazo. \\ \hline
        \citeonline{Musuva2025}   & Média & Média & Média & Alta  & Modelo presencial intensivo, difícil de escalar massivamente. \\ \hline
        \citeonline{Tshekiso2025} & Baixa & Alta  & Baixa & Média & Foco em infraestrutura/CTF padrão, sem pedagogia de gênero. \\ \hline
        \citeonline{Benson2025}   & Alta  & Baixa & Baixa & Alta  & Estudo teórico/qualitativo, sem plataforma de ensino aplicada. \\ \hline
        \citeonline{Costa2023}    & Alta  & Média & Alta  & Alta  & Foco em atração (Ensino Médio), não em formação profissional. \\ \hline
        \citeonline{Casey2023}    & Alta  & Baixa & Alta  & Alta  & Depende de mediação humana intensa (workshops presenciais). \\ \hline
        \citeonline{Thomas2024}   & Alta  & Baixa & Alta  & Baixa & Estudo de caso pequeno (N=24), foco em suporte individual. \\ \hline
        \citeonline{Rahman2022}   & Alta  & Baixa & Média & Alta  & Foco específico em retorno ao trabalho, escala de workshop. \\ \hline
        \citeonline{Hogan2025}    & Baixa & Alta  & Média & Alta  & Analisa times mistos \enquote{de sucesso}, viés de sobrevivência. \\ \hline
        
        \rowcolor[HTML]{C0C0C0} 
        \textbf{Esta Dissertação} & \textbf{Alta} & \textbf{Alta} & \textbf{Alta} & \textbf{Média} & \textbf{Validação em ambiente real massivo (\textit{Hackers do Bem}).} \\ \hline
    \end{tabular}%
    }
    \end{footnotesize}
    \fonte{Elaborado pela autora com base na análise sistemática.}
\end{quadro}

\section{Considerações Finais}
\label{s_consideracoes_finais}

A análise dos trabalhos relacionados permite concluir que a decomposição do problema em fatores isolados, focando puramente em gamificação técnica ou exclusivamente em políticas de gênero, é insuficiente para abordar a complexidade da evasão feminina em programas de formação massiva. Existe uma clara dicotomia na literatura vigente: de um lado, intervenções altamente eficazes e inclusivas baseadas em narrativas e \textit{scaffolding}, como demonstrado por \citeonline{Costa2025} e \citeonline{Casey2023}, que operam em escalas reduzidas e controladas; do outro, modelos de alta escalabilidade técnica focados em infraestrutura, como apresentado por \citeonline{Tshekiso2025} e \citeonline{Hogan2025}, que frequentemente negligenciam as nuances pedagógicas necessárias para a retenção de grupos sub-representados.

O programa \textit{Hackers} do Bem, que será utilizado como estudo de caso desta dissertação, situa-se no centro desta dicotomia. Ao combinar o ensino em larga escala com um sistema de ranqueamento gamificado e pontos de experiência (\ac{XP}), o programa apresenta um cenário único onde as mecânicas de competição podem atuar de forma ambivalente: como fator de engajamento para alguns ou como barreira de exclusão para iniciantes inseguras, conforme alertam os achados sobre autoeficácia de \citeonline{Rahman2022}. A ausência de um modelo na literatura que integre a \enquote{pedagogia do acolhimento} (baseada em mentoria e suporte de pares) com a \enquote{eficiência da escala} (baseada em automação e \ac{EAD}) evidencia a lacuna teórica e prática que este trabalho busca preencher.
\chapter{Proposta de Pesquisa}
\label{c_proposta}

Neste capítulo, apresenta-se o plano de trabalho para o desenvolvimento da pesquisa aplicada, visando responder às questões de pesquisa (QP1, QP2, QP3 e QP4) definidas na introdução. A proposta consiste na análise quantitativa e qualitativa dos dados do Programa \textit{Hackers do Bem}, correlacionando-os com as diretrizes identificadas na literatura.

\section{Visão Geral e Premissas}
\label{s_visao_geral_premissas}

A presente pesquisa classifica-se como aplicada e exploratória, adotando uma abordagem de métodos mistos (\textit{mixed methods}) que triangula a análise quantitativa de dados educacionais com a análise qualitativa da percepção das participantes. O objetivo central é a proposição de um \textbf{Framework de Diretrizes de Inclusão}, fundamentado em evidências empíricas do Programa \textit{Hackers do Bem}.

Para a viabilização e integridade ética deste estudo, estabelecem-se as seguintes premissas e procedimentos mandatórios:

\begin{enumerate}
    \item \textbf{Protocolo Ético e Acesso aos Dados:} O acesso à base de dados dos participantes não é automático. A execução da pesquisa está condicionada à submissão e aprovação do projeto pelo Comitê de Ética em Pesquisa (CEP) da Univali, via Plataforma Brasil, conforme as normas institucionais vigentes \cite{UnivaliCEP}. Após a aprovação ética, será formalizada a solicitação de acesso aos dados junto à gestão do Programa \textit{Hackers do Bem} (RNP/Softex/Senai).
    
    \item \textbf{Privacidade e LGPD:} Todo o tratamento de dados respeitará rigorosamente a Lei Geral de Proteção de Dados Pessoais (Lei nº 13.709/2018). A análise quantitativa utilizará exclusivamente dados pseudoanonimizados, garantindo que nenhuma participante seja identificada individualmente nos resultados divulgados. Para a coleta de dados primários (formulários), será aplicado o Termo de Consentimento Livre e Esclarecido (TCLE).
    
    \item \textbf{Rastreabilidade da Gamificação:} Premissa-se que a plataforma de ensino registra de forma granular as interações dos alunos. Conforme o Manual de Aprovação \cite{HackersDoBemManual}, o sistema contabiliza XP, emblemas e notas, dados essenciais para correlacionar o desempenho técnico com as taxas de evasão feminina.
\end{enumerate}

\subsection{Estratégia de Coleta e Análise (Sugestão Metodológica)}
\label{ss_estrategia_coleta}

Para responder às questões de pesquisa (QP2 e QP3), a metodologia será dividida em duas frentes complementares:

\subsubsection{Frente 1: Mineração de Dados (O "Onde" e o "Quando")}
Nesta etapa, serão analisados os dados brutos solicitados para mapear o "funil de evasão". O objetivo é identificar estatisticamente em qual momento exato ocorre a maior perda de participantes femininas.
\begin{itemize}
    \item \textbf{Métrica de Evasão:} Comparativo da taxa de abandono entre as fases de Nivelamento (assíncrono/sem concorrência) e Fundamental (síncrono/competitivo).
    \item \textbf{Correlação de Desempenho:} Verificação se mulheres com notas altas (competência comprovada) desistem antes da fase de Ranqueamento, indicando barreiras não-técnicas.
\end{itemize}

\subsubsection{Frente 2: Levantamento de Percepção (O "Porquê")}
Será aplicado um questionário eletrônico direcionado às mulheres que participaram do programa (tanto as que persistiram quanto as que evadiram). Sugere-se que o instrumento seja fundamentado em escalas validadas na literatura para medir três construtos identificados na RSL:

\begin{enumerate}
    \item \textbf{Autoeficácia em Cibersegurança:} Baseado em \citeonline{Costa2025}, para medir se a participante se sentia capaz de realizar as tarefas, independentemente da nota real.
    \item \textbf{Percepção de Competitividade:} Baseado em \citeonline{Horcher2021}, para avaliar se o \textit{leaderboard} (ranking público) e a mecânica de "soma zero" atuaram como fator motivacional ou ansiogênico.
    \item \textbf{Pertencimento e Soft Skills:} Baseado em \citeonline{Benson2025}, para verificar se a participante percebeu valorização de habilidades colaborativas durante a formação.
\end{enumerate}


\section{Metodologia de Análise de Dados (Design da Solução)}
\label{s_metodologia_analise}

A abordagem metodológica proposta para esta pesquisa é de natureza mista (\textit{mixed-methods}), estruturada em um \textit{pipeline} de processamento de dados que integra a mineração de dados educacionais (quantitativo) com a análise de percepção subjetiva (qualitativo). O "design da solução", neste contexto, refere-se à arquitetura analítica desenvolvida para mensurar o impacto das variáveis do Programa \textit{Hackers do Bem} na retenção feminina.

A estratégia de análise foi desenhada para isolar variáveis de gamificação (XP, Nível) e desempenho (Notas) e correlacioná-las com os pontos de evasão no funil de formação.

\subsection{Pipeline de Processamento e Fontes de Dados}
\label{ss_pipeline_dados}

A execução da pesquisa seguirá um fluxo de trabalho dividido em três camadas, conforme a arquitetura de referência para análise de dados educacionais: (1) Coleta e Anonimização, (2) Processamento de Métricas e (3) Correlação. As fontes de dados primárias e os procedimentos de tratamento estão definidos a seguir:

\begin{enumerate}
    \item \textbf{Base de Dados Institucional (RNP):} Dados transacionais da plataforma de ensino, contendo registros de acesso, pontuação de gamificação (XP), notas em avaliações e progresso por trilha. O acesso a estes dados será formalizado junto à gestão do programa, respeitando os protocolos de segurança da informação \cite{HackersDoBemManual}.
    \item \textbf{Dados de Percepção (Levantamento):} Coleta primária via formulários eletrônicos aplicados às participantes, focando em construtos de autoeficácia e percepção da competitividade.
\end{enumerate}

\subsection{Definição das Métricas e Variáveis}
\label{ss_metricas_variaveis}

Para quantificar o fenômeno da evasão e permitir a comparação objetiva entre os gêneros, foram definidas métricas específicas baseadas na literatura de avaliação de treinamento e gamificação.

\subsubsection{Métricas de Evasão e Retenção}
A taxa de retenção ($TR$) será calculada para cada fase do programa (Nivelamento, Básico, Fundamental, Especializado). Seja $N_{i}$ o número de inscritos no início da fase e $N_{f}$ o número de concluintes aprovados, a métrica é definida pela Equação \ref{eq:taxa_retencao}:

\begin{equation}
    TR_{fase} = \left( \frac{N_{f}}{N_{i}} \right) \times 100
    \label{eq:taxa_retencao}
\end{equation}

Adicionalmente, será calculado o \textit{Gap de Retenção por Gênero} ($GRG$), que mede a disparidade de sucesso entre participantes do gênero feminino ($TR_{fem}$) e masculino ($TR_{masc}$), permitindo identificar gargalos específicos:

\begin{equation}
    GRG = TR_{fem} - TR_{masc}
    \label{eq:gap_retencao}
\end{equation}

Valores negativos de $GRG$ indicarão fases onde o programa falha desproporcionalmente na retenção de mulheres.

\subsubsection{Métricas de Gamificação e Desempenho}
Para investigar a hipótese de que a competição afeta a permanência (QP3), será analisada a correlação entre o acúmulo de Pontos de Experiência ($XP$) e a taxa de evasão. Conforme o manual do programa, o $XP$ é acumulado por engajamento e acertos \cite{HackersDoBemManual}.

Define-se a métrica de \textit{Eficiência de Engajamento} ($EE$) para isolar alunas que possuem alto desempenho técnico, mas baixo engajamento nos elementos competitivos (ranking):

\begin{equation}
    EE = \frac{Nota_{pratica}}{XP_{total}}
    \label{eq:eficiencia_engajamento}
\end{equation}

Uma $EE$ alta sugere que a participante domina o conteúdo técnico (nota alta), mas não se engaja nas mecânicas de gamificação massiva (XP baixo), o que corrobora a teoria de \citeonline{Horcher2021} sobre a preferência feminina por aprendizado menos competitivo.

\subsection{Instrumentos de Coleta Qualitativa}
\label{ss_instrumentos_quali}

Complementando a análise métrica, será aplicado um questionário estruturado baseado em escalas Likert de 5 pontos (1 = Discordo Totalmente, 5 = Concordo Totalmente), similar à metodologia de validação utilizada na dissertação de referência. O instrumento visa validar se os dados quantitativos de evasão correspondem a uma percepção de baixa autoeficácia ou falta de pertencimento.

Os construtos avaliados serão:
\begin{itemize}
    \item \textbf{Autoeficácia Técnica:} Baseado em \citeonline{Costa2025}, mede a confiança da aluna em realizar tarefas de cibersegurança.
    \item \textbf{Ansiedade Competitiva:} Avalia o impacto dos \textit{leaderboards} (rankings) na motivação da aluna.
    \item \textbf{Percepção de Suporte:} Avalia a eficácia das mentorias e da comunidade.
\end{itemize}

\subsection{Aspectos Éticos e Tratamento de Dados}
\label{ss_aspectos_eticos}

Considerando que a pesquisa envolve dados de seres humanos, todo o procedimento seguirá as diretrizes da Resolução 466/12 do Conselho Nacional de Saúde. O projeto será submetido ao Comitê de Ética em Pesquisa (CEP) da Univali via Plataforma Brasil.

Para garantir a conformidade com a Lei Geral de Proteção de Dados (LGPD), será aplicado um processo de pseudoanonimização nos dados fornecidos pela RNP antes de qualquer análise estatística, removendo identificadores diretos (CPF, E-mail, Nome) e mantendo apenas os atributos de interesse (Gênero, Notas, XP, Região).


\subsection{Definição das Métricas e Variáveis}
\label{ss_metricas_variaveis}

Para a avaliação quantitativa da eficácia do programa na retenção de talentos femininos, foram estabelecidas métricas baseadas na análise de funil e correlação de desempenho. A formalização matemática destas métricas visa isolar o comportamento dos grupos em relação aos mecanismos de gamificação (XP) e ranqueamento descritos no Manual do Programa \cite{HackersDoBemManual}.

Define-se o conjunto total de participantes como $P$, onde $P_{f} \subset P$ representa o subconjunto de participantes do gênero feminino e $P_{m} \subset P$ o subconjunto do gênero masculino. As métricas de avaliação são detalhadas a seguir.

\subsubsection{Taxa de Evasão por Fase ($\epsilon$)}
A taxa de evasão ($\epsilon$) mensura a perda de participantes entre o início e o fim de um módulo específico (Nivelamento, Básico, Fundamental). Seja $N_{i}$ o número de inscritos no início da fase e $N_{c}$ o número de concluintes aprovados, a métrica é definida pela Equação \ref{eq:taxa_evasao}:

\begin{equation}
    \epsilon = 1 - \left( \frac{N_{c}}{N_{i}} \right)
    \label{eq:taxa_evasao}
\end{equation}

O objetivo é calcular o $\Delta\epsilon = \epsilon_{f} - \epsilon_{m}$, onde um valor positivo indica que a evasão feminina é superior à masculina, evidenciando gargalos específicos naquela etapa do funil.

\subsubsection{Eficiência de Engajamento Gamificado ($\gamma$)}
Considerando que o programa utiliza Pontos de Experiência (XP) como critério de ranqueamento, é necessário medir se o acúmulo de XP reflete a competência técnica de forma equitativa. Define-se a Eficiência de Engajamento ($\gamma$) como a razão entre a nota obtida nas atividades práticas ($Nota_{pratica}$) e o total de XP acumulado por consumo de conteúdo ($XP_{conteudo}$), conforme Equação \ref{eq:eficiencia_xp}:

\begin{equation}
    \gamma = \frac{Nota_{pratica}}{XP_{conteudo}}
    \label{eq:eficiencia_xp}
\end{equation}

Uma alta taxa $\gamma$ sugere que a participante possui alta competência técnica (nota alta) com menor dependência dos mecanismos de engajamento massivo (XP), validando a hipótese de \citeonline{Horcher2021} sobre perfis de aprendizado.

\subsubsection{Síntese das Variáveis de Análise}
A \autoref{tab:variaveis_analise} apresenta o resumo das variáveis independentes e dependentes que compõem o estudo, estruturadas para permitir a comparação entre os grupos, seguindo o modelo de organização de experimentos adotado em trabalhos anteriores \cite{Nunes2018}.

\begin{table}[htb]
\centering
\caption{Variáveis e Métricas definidas para os experimentos}
\label{tab:variaveis_analise}
\begin{tabular}{l|l|l}
\hline
\textbf{Tipo} & \textbf{Variável/Métrica} & \textbf{Descrição e Fonte de Dados} \\ \hline
Independente & Gênero ($g$) & Autodeclaração no cadastro (M/F). \\ \hline
Independente & Fase do Funil ($f$) & Nivelamento, Básico, Fundamental, Residência. \\ \hline
Dependente & Score de Autoeficácia ($S_{ae}$) & Média obtida via questionário (Escala Likert 1-5) \cite{Costa2025}. \\ \hline
Dependente & Evasão ($\epsilon$) & Status de desistência antes da conclusão do módulo. \\ \hline
Dependente & Ranking ($R$) & Posição final na lista classificatória (Critério de corte). \\ \hline
Dependente & Desempenho Técnico ($D_t$) & Notas em laboratórios práticos e CTFs. \\ \hline
\end{tabular}
\fonte{Elaborada pela autora.}
\end{table}

\subsection{Coleta e Tratamento dos Dados do Programa}
\label{ss_coleta_dados}

A estratégia de aquisição de dados para esta pesquisa foi desenhada para operar em duas frentes complementares: a extração de dados secundários (logs e registros acadêmicos) e a coleta de dados primários (levantamento de percepção), garantindo uma análise mista robusta sobre a retenção feminina.

\subsubsection{Extração de Dados da Plataforma (RNP)}
A fonte primária de dados quantitativos reside no Ambiente Virtual de Aprendizagem (AVA) do Programa \textit{Hackers do Bem}, gerenciado pela Rede Nacional de Ensino e Pesquisa (RNP). A coleta será realizada mediante solicitação formal à coordenação do programa, visando a obtenção de \textit{dumps} ou relatórios estruturados (formatos CSV ou JSON) contendo os registros de interação dos alunos.

Para viabilizar a análise do "funil de engajamento", serão solicitados os seguintes conjuntos de dados, conforme a estrutura curricular descrita no Manual de Aprovação \cite{HackersDoBemManual}:

\begin{itemize}
    \item \textbf{Dados Demográficos (Anonimizados):} Gênero declarado, faixa etária e região geográfica.
    \item \textbf{Dados de Desempenho:} Notas nas avaliações finais, pontuação em atividades práticas e status de aprovação/reprovação por módulo (Nivelamento, Básico, Fundamental e Especializado).
    \item \textbf{Métricas de Gamificação:} Total de Pontos de Experiência (XP) acumulados, emblemas desbloqueados (ex: "Aspirante a Hacker", "Mestre de Segurança") e frequência de interação com \textit{quizzes}.
    \item \textbf{Logs de Evasão:} Data da última interação registrada na plataforma, permitindo calcular o tempo de permanência antes do abandono.
\end{itemize}

\subsubsection{Levantamento de Percepção (Dados Primários)}
Complementarmente, será aplicado um questionário eletrônico voltado especificamente às participantes do gênero feminino (tanto evadidas quanto concluintes). Este instrumento visa capturar aspectos subjetivos não registrados nos logs do sistema, como a percepção de autoeficácia e o impacto do ambiente competitivo (ranqueamento) na motivação de aprendizado.

\subsubsection{Aspectos Éticos e Conformidade (LGPD e CEP)}
Considerando que a pesquisa envolve o tratamento de dados pessoais e a participação de seres humanos, todos os procedimentos seguirão rigorosamente as diretrizes éticas e legais vigentes.

O projeto será submetido à apreciação do Comitê de Ética em Pesquisa (CEP) da Universidade do Vale do Itajaí (Univali), via Plataforma Brasil, conforme os fluxos institucionais de validação documental e relatoria \cite{UnivaliCEP}. A coleta de dados primários será condicionada à assinatura digital do Termo de Consentimento Livre e Esclarecido (TCLE) pelas participantes.

No que tange à Lei Geral de Proteção de Dados (LGPD - Lei nº 13.709/2018), será aplicado um protocolo de pseudoanonimização nos dados fornecidos pela RNP antes de qualquer processamento analítico. Identificadores diretos (como Nome, CPF e E-mail) serão substituídos por chaves alfanuméricas únicas (\textit{hashs}), garantindo que a análise estatística de evasão e desempenho ocorra sem expor a identidade individual das alunas, preservando a privacidade e a segurança das informações \cite{Wangham2021}.

\subsection{Estratégia de Análise do Funil (Pipeline)}
\label{ss_analise_funil}

A avaliação da retenção feminina será conduzida através de um \textit{pipeline} de análise sequencial, modelado para espelhar a arquitetura de progressão do Programa \textit{Hackers do Bem}. O processo de formação não é linear, mas sim composto por "portões de controle" (\textit{gates}) que utilizam algoritmos de ranqueamento para filtrar os participantes aptos à próxima etapa.

A estratégia de análise consiste em decompor o fluxo de alunos em quatro estágios de transição crítica, confrontando os dados de evasão real com as regras de negócio estabelecidas no Manual do Programa \cite{HackersDoBemManual}. O \textit{pipeline} de análise é definido pelas seguintes etapas:

\begin{itemize}
    \item \textbf{Transição $T_1$ (Nivelamento $\rightarrow$ Básico):} Nesta etapa, a análise é puramente volumétrica, visto que não há critérios de corte eliminatórios, apenas a conclusão dos módulos. O foco será mensurar a taxa de desistência espontânea ($\delta_{espontanea}$) entre alunas que concluíram o Nivelamento mas não iniciaram o Básico.
    
    \item \textbf{Transição $T_2$ (Básico $\rightarrow$ Fundamental):} Este é o primeiro ponto de ranqueamento competitivo. A análise verificará se a pontuação de gamificação (XP), utilizada como critério de classificação nesta fase, apresenta correlação negativa com o gênero feminino, validando a hipótese de que mecânicas de competição podem desestimular este grupo \cite{Horcher2021}.
    
    \item \textbf{Transição $T_3$ (Fundamental $\rightarrow$ Especializado - Mecânica de Ondas):} O acesso ao nível Especializado é regido por um limite rígido de vagas distribuídas em "Ondas" (1ª e 2ª Onda). A estratégia aqui consiste em simular o algoritmo de desempate descrito na Seção 6.3 do Manual, que prioriza a "Nota Prática" sobre o "Engajamento (XP)" \cite{HackersDoBemManual}. O objetivo é identificar se mulheres são desproporcionalmente alocadas para a "Lista de Espera" ou "2ª Onda", o que pode aumentar a probabilidade de evasão por perda de ênfase (tempo de espera).
    
    \item \textbf{Transição $T_4$ (Especializado $\rightarrow$ Residência Tecnológica):} A etapa final utiliza um ranqueamento acumulativo por trilha de especialização (Blue Team, Red Team, GRC, etc.). A análise focará na segregação vertical, verificando se há concentração feminina em trilhas de Gestão (GRC) em detrimento de trilhas técnicas (Red/Blue Team), e se a taxa de conversão para a Residência varia conforme a trilha escolhida.
\end{itemize}

Para cada transição $T_i$, será calculado o índice de sobrevivência ($S_{gen}$) por gênero, permitindo isolar se a perda de participantes ocorre por falha técnica (nota insuficiente) ou por desistência sistêmica (abandono apesar de notas suficientes), correlacionando estes eventos com as datas de liberação das listas de classificação.

\section{Validação da Proposta}
\label{s_validacao}

A validação do \textit{Framework de Diretrizes} proposto não se dará pela implementação imediata de um novo curso, dada a inviabilidade temporal, mas sim através de uma validação analítica comparativa (\textit{Benchmarking}). O objetivo é demonstrar que as diretrizes propostas cobrem as lacunas quantitativas identificadas no programa, utilizando como referência os indicadores de sucesso reportados no estado da arte.

A estratégia de validação consiste em confrontar as métricas de evasão e engajamento extraídas do Programa \textit{Hackers do Bem} (Diagnóstico - QP2 e QP3) com os resultados de intervenções internacionais bem-sucedidas mapeadas na RSL (QP1).

\subsection{Definição dos Benchmarks de Referência}
\label{ss_benchmarks}

Para estabelecer um critério de "sucesso" ou "falha" nas taxas de retenção e engajamento do programa, foram definidos indicadores-chave baseados nos trabalhos correlatos selecionados. O \autoref{qua:benchmarks_validacao} apresenta a matriz de referência que será utilizada para validar a gravidade dos gargalos encontrados no \textit{Hackers do Bem}.

\begin{quadro}[htb]
\centering
\caption{Matriz de Benchmarks para Validação dos Indicadores do Programa}
\label{qua:benchmarks_validacao}
\begin{footnotesize}
\begin{tabular}{|l|p{5.5cm}|p{6cm}|}
    \hline
    \textbf{Dimensão} & \textbf{Indicador de Referência (Benchmark)} & \textbf{Fonte e Justificativa} \\
    \hline
    Escalabilidade & Capacidade de manter a qualidade e retenção com o aumento exponencial de inscritos (ex: 600 para 1.700 alunos). & \citeonline{Tshekiso2025}: Estabelece parâmetros para programas massivos em países em desenvolvimento. \\
    \hline
    Autoeficácia & Aumento mensurável na confiança técnica após intervenções práticas (\textit{hands-on}), independente da nota final. & \citeonline{Costa2025}: Valida que a percepção de competência é mais crítica que a competência bruta para a retenção feminina. \\
    \hline
    Empregabilidade & Taxa de conversão de alunos formados para o mercado de trabalho (transição Academia-Indústria). & \citeonline{Musuva2025}: Define o sucesso não apenas pela conclusão do curso, mas pela inserção profissional efetiva. \\
    \hline
    Métricas de Gênero & Existência de dados desagregados e monitoramento contínuo de \textit{gaps} específicos. & \citeonline{SelmanHousein2025}: Critica a ausência de métricas quantitativas na maioria das intervenções. \\
    \hline
\end{tabular}
\end{footnotesize}
\fonte{Elaborado pela autora.}
\end{quadro}

\subsection{Análise de Gap e Aderência}
\label{ss_analise_gap}

A validação final consistirá no cálculo do \textit{Gap de Desempenho} ($\Delta P$) entre o cenário atual do \textit{Hackers do Bem} e os benchmarks estabelecidos. Para cada dimensão $d$ listada no \autoref{qua:benchmarks_validacao}, será realizada uma análise qualitativa e quantitativa da discrepância:

\begin{equation}
    \Delta P_{d} = Métrica_{HdB} - Benchmark_{Literatura}
    \label{eq:gap_performance}
\end{equation}

Se $\Delta P_{d}$ apresentar um valor negativo significativo, a diretriz proposta para aquela dimensão será considerada validada se, e somente se, atacar a causa raiz identificada na análise do funil (Seção \ref{ss_analise_funil}). Por exemplo, se a evasão feminina no nível "Fundamental" for 30\% superior à do estudo de \citeonline{Tshekiso2025}, a diretriz de "Gamificação Colaborativa" será validada como uma intervenção necessária para mitigar este desvio específico.

Desta forma, assegura-se que o \textit{framework} não é composto por sugestões genéricas, mas por soluções cirúrgicas para os problemas métricos reais do programa, respondendo à QP4.

\section{Cronograma de Execução}
\label{ss_cronograma}

O planejamento temporal desta pesquisa foi elaborado considerando a natureza aplicada do trabalho e o prazo regimental para a defesa da dissertação, estipulado para ocorrer impreterivelmente até 23 de maio de 2026 \cite{EmailCoordenacao2026}.

A execução das atividades está condicionada à aprovação do projeto pelo Comitê de Ética em Pesquisa (CEP), cuja submissão está programada para fevereiro de 2026. Para assegurar a viabilidade do cronograma, as etapas foram organizadas de forma a permitir o paralelismo entre os trâmites burocráticos e a preparação técnica do ambiente de análise.

As macroatividades foram distribuídas conforme o detalhamento a seguir:

\begin{enumerate}
    \item \textbf{Aprovação Ética e Regulatória (Fev):} Submissão do protocolo de pesquisa na Plataforma Brasil e solicitação formal de acesso aos dados junto à RNP, em conformidade com as normas da Univali \cite{UnivaliCEP}.
    \item \textbf{Coleta e Pré-processamento (Fev/Mar):} Extração dos dados brutos das turmas finalizadas (Ondas 1 e 2) e envio dos formulários de pesquisa para as participantes, iniciando-se imediatamente após a aprovação do CEP (previsão para o fim de fevereiro).
    \item \textbf{Mineração e Análise (Mar/Abr):} Execução do \textit{pipeline} de análise de dados para cálculo das métricas de evasão ($\epsilon$) e correlação de gamificação, simultaneamente à tabulação dos dados qualitativos dos formulários.
    \item \textbf{Redação e Defesa (Abr/Mai):} Escrita dos capítulos de resultados e discussão, revisão final do documento e realização da defesa pública.
\end{enumerate}

O \autoref{qua:cronograma_detalhado} sintetiza a alocação temporal das atividades, evidenciando a concentração de esforços na análise de dados durante o mês de março para garantir a entrega da versão final em abril.

\begin{quadro}[htb]
\centering
\caption{Cronograma de atividades para conclusão da dissertação (2026)}
\label{qua:cronograma_detalhado}
\begin{footnotesize}
\begin{tabular}{|l|c|c|c|c|}
\hline
\textbf{Atividade / Etapa} & \textbf{Fev} & \textbf{Mar} & \textbf{Abr} & \textbf{Mai} \\ \hline
Submissão ao CEP e Ajustes da Qualificação & X & & & \\ \hline
Solicitação e Acesso aos Dados (RNP) & X & & & \\ \hline
Aplicação de Questionários e Coleta de Dados & X & X & & \\ \hline
Processamento dos Dados e Análise do Funil & & X & & \\ \hline
Redação dos Capítulos de Resultados (4 e 5) & & X & X & \\ \hline
Revisão Final e Entrega da Dissertação & & & X & \\ \hline
Defesa da Dissertação (Prazo: 23/05) & & & & X \\ \hline
\end{tabular}
\end{footnotesize}
\fonte{Elaborado pela autora com base no calendário acadêmico \cite{EmailCoordenacao2026}.}
\end{quadro}

\section{Considerações Finais}
\label{s_consideracoes}

Neste capítulo, foi detalhada a proposta metodológica para o diagnóstico e a mitigação da evasão feminina no Programa \textit{Hackers do Bem}. A abordagem mista adotada, combinando a mineração de dados educacionais com a análise qualitativa da percepção, foi desenhada para superar as limitações identificadas na revisão da literatura, especificamente a escassez de métricas quantificáveis sobre inclusão em larga escala \cite{SelmanHousein2025}.

A estratégia de pesquisa fundamenta-se na premissa de que a gamificação, quando não ajustada para a diversidade, pode atuar como uma barreira inadvertida. O \textit{pipeline} de análise de dados proposto (Seção \ref{ss_analise_funil}) e as métricas de eficiência de engajamento (Seção \ref{ss_metricas_variaveis}) fornecem as ferramentas analíticas necessárias para testar essa hipótese com rigor estatístico.

Ressalta-se que a viabilidade da execução está assegurada pelo cronograma estabelecido, que prevê o paralelismo entre os trâmites éticos (CEP) e a preparação do ambiente de análise. A utilização de instrumentos validados para a coleta de dados primários, alinhada às melhores práticas de pesquisa social em computação \cite{Denscombe2014}, garante a confiabilidade dos resultados qualitativos que complementarão os dados massivos da plataforma.

Desta forma, a proposta aqui apresentada não apenas busca responder às questões de pesquisa formuladas, mas também visa entregar um artefato prático — o Framework de Diretrizes — capaz de influenciar a evolução do \textit{design} instrucional de programas de capacitação cibernética em nível nacional.

%\chapter{Plano de Avaliação}
\label{c_plano_avaliacao}

\section{Resultado Esperados}
\label{s_resultados_esperados}

\section{Considerações}
\label{s_consideracoes}
%\chapter{Conclusão}
\label{c_conclusao}

\section{Contribuição da Dissertação}
\label{s_contribuicao_dissertacao}

\section{Trabalhos Futuros}
\label{s_trabalhos_futuros}

% ---
% Finaliza a parte no bookmark do PDF
% para que se inicie o bookmark na raiz
% e adiciona espaço de parte no Sumário
% ---
%\phantompart

% ----------------------------------------------------------
% ELEMENTOS PÓS-TEXTUAIS
% ----------------------------------------------------------
\postextual

% ----------------------------------------------------------
% Referências bibliográficas
% ----------------------------------------------------------
%\bibliographystyle{abnt-alf}
\bibliography{bibliography/11_referencias}

% ----------------------------------------------------------
% Glossário
% ----------------------------------------------------------
%
% Consulte o manual da classe abntex2 para orientações sobre o glossário.
%
%\glossary

% ----------------------------------------------------------
% Apêndices
% ----------------------------------------------------------

% ---
% Inicia os apêndices
% ---
%\begin{apendicesenv}

% Imprime uma página indicando o início dos apêndices
%\partapendices
%\setlength\afterchapskip{\lineskip}
%\chapter{Título do apêndice A}
\label{titulo_do_apendice_a}

Exemplo de inclusão de um documento em PDF:

\includepdf[pages=1,scale=1]{pdf/exemplo.pdf}
%\chapter{Título do apêndice B}
\label{apendice_b}

\section{Planejamento da Revisão Sistemática}
\label{s_apendiceB_planejamento}

\subsection{Questões de pesquisa}
\label{ss_apendiceB_objetivos}

\subsection{\textit{String} de Busca}
\label{ss_apendiceB_protocolo-busca}

\subsection{Identificação dos recursos}
\label{ss_apendiceB_id-recursos}

\subsection{Critérios de inclusão e exclusão}
\label{ss_apendiceB_criterios}

\subsection{Seleção dos artigos e extração dos dados}
\label{ss_apendiceB_extraction}

\section{Execução da Revisão Sistemática}
\label{s_apendiceB_execução}

\subsection{ACM}
\label{ss_apendiceB_acm}

\subsection{IEEExplore}
\label{ss_apendiceB_ieee}

\subsection{ScienceDirect}
\label{ss_apendiceB_sciencedirect}

\subsection{Scopus}
\label{ss_apendiceB_scopus}

\subsection{Springer}
\label{ss_apendiceB_springer}

\section{Resultados da Revisão Sistemática}
\label{s_apendiceB_resultado}
%\chapter{Título do apêndice C}
\label{apendice_c}
%\chapter{Título do apêndice D}
\label{apendice_d}
%\chapter{Título do apêndice E}
\label{apendice_e}
%% ----------------------------------------------------------------------- %
% Arquivo: 25_apendiceF.tex
% ----------------------------------------------------------------------- %

\chapter{Título do apêndice F}
\label{apendice_f}



%\chapter{Título do apêndice G}
\label{apendice_g}

%\end{apendicesenv}
% ---

% ----------------------------------------------------------
% Anexos
% ----------------------------------------------------------

% ---
% Inicia os anexos
% ---
%\begin{anexosenv}

% Imprime uma página indicando o início dos anexos
%\partanexos

%\setlength\afterchapskip{18pt}
%\include{30_anexoA}

%\end{anexosenv}

%---------------------------------------------------------------------
% INDICE REMISSIVO
%---------------------------------------------------------------------

\phantompart
\printindex

\end{document}