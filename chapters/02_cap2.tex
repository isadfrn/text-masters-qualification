\chapter{Fundamentação Teórica}
\label{c_fundamentacao_teorica}

Este capítulo apresenta os conceitos fundamentais necessários para a compreensão do problema de pesquisa e da análise realizada. A fundamentação está dividida em quatro eixos principais que sustentam as questões de pesquisa: o cenário global e nacional da força de trabalho em cibersegurança e as políticas públicas recentes (E-Ciber); as disparidades de gênero e as barreiras socioculturais (QP1); as metodologias de ensino, com foco em gamificação, andaimes cognitivos e competições (QP2 e QP3); e o detalhamento do estudo de caso, o Programa \textit{Hackers} do Bem.

\section{Cenário da Força de Trabalho em Cibersegurança}
\label{s_cenario_forca_trabalho_ciberseguranca}

A escassez de profissionais qualificados em segurança da informação é um desafio global que impacta diretamente a capacidade de defesa de organizações e nações. Nesta seção, discute-se a evolução do déficit de talentos (\textit{workforce gap}) e as políticas públicas para sua mitigação.

\subsection{O Déficit Global e Nacional de Talentos}
\label{ss_deficit_global_nacional_talentos}

A disparidade entre a demanda por profissionais de segurança da informação e a oferta de mão de obra qualificada constitui um desafio sistêmico para a resiliência das organizações. Dados do \textit{ISC2 Cybersecurity Workforce Study 2024} estimam que a força de trabalho global em cibersegurança atingiu aproximadamente 5,5 milhões de pessoas, um crescimento estagnado de apenas 0,1\% em relação ao ano anterior. Em contrapartida, o déficit global (\textit{workforce gap}), ou seja, o número de profissionais adicionais necessários para proteger adequadamente os ativos organizacionais alcançou a marca de 4,8 milhões, o maior índice registrado na história do estudo \cite{ISC2_2024, Fortinet2025}.

Mais recentemente, a análise do cenário em 2025 indica uma mudança qualitativa na percepção desse déficit. Segundo a \citeonline{ISC2_2025}, embora a escassez numérica absoluta persista, a prioridade das organizações deslocou-se para a lacuna de habilidades (\textit{skills gap}). O estudo aponta que 95\% dos entrevistados identificam carências de competências específicas em suas equipes atuais, superando a preocupação com o simples aumento do quadro de funcionários (\textit{headcount}). As lacunas mais críticas concentram-se em áreas emergentes, como Inteligência Artificial (41\%) e Segurança em Nuvem (36\%), além de competências comportamentais (\textit{soft skills}) necessárias para a gestão de crises e resolução de problemas complexos \cite{ISC2_2025, ISACA2025}.

Essa escassez não afeta todos os setores de maneira uniforme. O relatório \textit{Global Cybersecurity Outlook 2025} do \ac{WEF} destaca que a iniquidade cibernética se aprofundou, com o setor público e as pequenas e médias empresas (\ac{PMEs}) enfrentando dificuldades desproporcionais para atrair e reter talentos. O documento revela que 49\% das organizações do setor público relatam não possuir a força de trabalho necessária para atingir seus objetivos de segurança, um aumento de 33\% em relação ao ano anterior \cite{WEF2025}. No Brasil, estima-se um déficit de 230 mil especialistas \cite{PalestraHDB}, situando o país em uma posição crítica que demanda políticas públicas de formação em larga escala.

\subsection{Políticas Públicas e a Estratégia Nacional (E-Ciber)}
\label{ss_politicas_publicas_estrategia_nacional}

No Brasil, a resposta governamental ao cenário de escassez de mão de obra qualificada e à necessidade de fortalecimento da soberania digital materializou-se com a publicação do Decreto nº 12.573, de 4 de agosto de 2025, que institui a Política Nacional de Cibersegurança e a Estratégia Nacional de Cibersegurança, denominada E-Ciber \cite{Decreto12573}. Este marco legal reconhece a educação como um pilar estratégico para a resiliência das infraestruturas críticas nacionais e estabelece diretrizes para a atuação coordenada entre o Estado e a sociedade.

A E-Ciber estrutura-se em quatro eixos temáticos que orientam as iniciativas públicas e privadas, detalhados a seguir conforme o Artigo 1º do referido decreto:
\begin{enumerate}
    \item \textbf{Proteção e conscientização do cidadão e da sociedade:} Este eixo foca na criação de condições seguras para o uso de serviços digitais. O Artigo 3º destaca a atenção especial a grupos em situação de vulnerabilidade, enquanto o Artigo 4º, inciso V, preconiza explicitamente o \enquote{incentivo à inclusão de temas relacionados à cibersegurança nos currículos de todos os níveis educacionais}.
    
    \item \textbf{Segurança e resiliência dos serviços essenciais e das infraestruturas críticas:} Visa fornecer instrumentos efetivos para a prevenção e resposta a ciberincidentes, garantindo a continuidade de serviços vitais para o Estado e a economia.
    
    \item \textbf{Cooperação e integração entre os órgãos e entidades, públicas e privadas:} Estabelece o compartilhamento de informações e a atuação conjunta como mecanismos de defesa, alinhando-se às recomendações internacionais sobre a necessidade de romper silos de informação entre setores.
    
    \item \textbf{Soberania nacional e governança:} Este eixo é fundamental para a justificativa de programas massivos. O Artigo 9º, inciso III, estabelece como objetivo a \enquote{formação e capacitação técnico-profissional em cibersegurança em escala compatível com as necessidades nacionais}.
\end{enumerate}

A diretriz de formação em escala (Eixo IV) fundamenta a transição de modelos de treinamento restritos para iniciativas de grande abrangência (QP2), alinhando-se à necessidade urgente de reduzir o débito tecnológico do país em áreas emergentes. O relatório \textit{Global Cybersecurity Outlook 2025} do \ac{WEF} observa que a E-Ciber, ao priorizar o investimento em educação, busca abordar a iniquidade cibernética (\textit{cyber inequity}) que afeta desproporcionalmente setores com menos recursos \cite{WEF2025}.

É neste contexto regulatório que se insere o Programa \textit{Hackers} do Bem, financiado pelo \acf{MCTI} com recursos da Lei de Informática (Lei nº 8.248/1991) e executado pela \acf{RNP} em parceria com a Softex e o \acf{SENAI} São Paulo \cite{Lei8248, MCTI2025}. O programa atua como um instrumento operacional para atingir as metas de capacitação massiva estipuladas pelo Estado, oferecendo um itinerário formativo gratuito que busca atender à escala demandada pela E-Ciber.

Nesta dissertação, o Programa \textit{Hackers} do Bem é analisado como estudo de caso para investigar como essa diretriz de massificação interage com as necessidades específicas de inclusão e retenção, verificando se a escala proposta pela política pública é compatível com as estratégias de personalização necessárias para a permanência de grupos sub-representados, como o público feminino.

\section{Desigualdade de Gênero na Cibersegurança}
\label{s_desigualdade_genero_ciberseguranca}

A sub-representação feminina na cibersegurança não é apenas uma questão de diversidade, mas de eficiência econômica e inovação. Esta seção explora as barreiras estruturais e culturais que impedem a entrada e a permanência de mulheres na área.

\subsection{Panorama da Representatividade Feminina}
\label{ss_panorama_representatividade_feminina}

A participação feminina na força de trabalho global de cibersegurança apresenta um cenário de estagnação relativa, apesar da crescente demanda por profissionais. Estudos demográficos recentes situam a representatividade feminina entre 20\% e 27\%, variando conforme a metodologia de aferição utilizada pelas organizações \cite{ISC2_2025, Fortinet2025}. \citeonline{SelmanHousein2025} observam que, embora este percentual seja superior ao de áreas emergentes como Inteligência Artificial (12\%), ele permanece significativamente inferior a outros subcampos da tecnologia, como a Educação em Computação, onde a presença feminina alcança 42\%, sugerindo que a cultura específica da segurança da informação impõe barreiras de entrada distintas (QP1).

Existe um paradoxo documentado na literatura entre a qualificação formal e a ascensão profissional. Dados da \citeonline{ISC2_2025} revelam que as mulheres na área são, em média, mais escolarizadas que os homens, com 52\% delas possuindo pós-graduação, comparado a apenas 44\% de seus pares masculinos. Contudo, essa qualificação superior não se traduz em equidade hierárquica ou salarial. Conforme detalha \citeonline{Pacheco2024}, as mulheres ocupam apenas 17\% dos cargos de liderança executiva (\textit{C-level}) e enfrentam uma disparidade salarial persistente, recebendo cerca de 12\% a menos que homens em posições equivalentes. Esses indicadores corroboram a existência de barreiras invisíveis de progressão que desestimulam a permanência a longo prazo.

No âmbito do recrutamento, o relatório da \citeonline{Fortinet2025} indica que, embora a dificuldade declarada pelas empresas em encontrar candidatas qualificadas tenha diminuído estatisticamente, 70\% das organizações ainda relatam que as mulheres compõem metade ou menos de suas equipes de segurança. Este cenário reforça a relevância de analisar o Programa \textit{Hackers} do Bem como estudo de caso, investigando se um modelo de capacitação em larga escala consegue romper com essas estatísticas globais ou se os mecanismos de seleção e engajamento reproduzem os mesmos filtros que limitam a diversidade no mercado tradicional.

\subsection{Barreiras de Entrada e o Fenômeno do \textit{Leaky Pipeline}}
\label{ss_barreiras_entrada_fenomeno_leaky_pipeline}

O conceito de \textit{Leaky Pipeline} (vazamento de talentos) é amplamente utilizado na literatura para descrever a perda progressiva e desproporcional da participação feminina ao longo das etapas de formação acadêmica e transição para a carreira. \citeonline{SelmanHousein2025} destacam que esse fenômeno não ocorre por falta de capacidade técnica, mas devido a barreiras estruturais que filtram mulheres em taxas superiores aos homens em cada estágio de transição. Em contextos de países em desenvolvimento, \citeonline{BothaBadenhorst2023} identificam que essas barreiras iniciam-se no acesso desigual à educação \ac{STEM} e agravam-se pela escassez crítica de modelos de referência (\textit{role models}) femininos em posições de destaque, o que dificulta a visualização de uma trajetória de sucesso na área.

No contexto específico da cibersegurança, as barreiras de entrada são exacerbadas por estereótipos culturais arraigados. \citeonline{Pacheco2024} e \citeonline{Ricci2021} apontam que a área é frequentemente associada a uma cultura \enquote{militarista} e masculina, ou à figura do \textit{hacker} solitário e socialmente isolado. \citeonline{Costa2025} argumentam que essa representação gera uma dissonância de identidade para jovens mulheres, que muitas vezes não se identificam com essas características comportamentais, levando-as a descartar a carreira mesmo possuindo as competências lógicas necessárias. Além disso, \citeonline{Coenraad2020} observam que a mídia e os jogos digitais frequentemente reforçam esses estereótipos, retratando a cibersegurança como um domínio de agressividade e competição de soma zero, o que pode afastar perfis mais voltados à colaboração.

Para aquelas que superam as barreiras iniciais de entrada, a permanência torna-se o próximo desafio do \textit{leaky pipeline}. \citeonline{BothaBadenhorst2023} descrevem a prevalência de uma \enquote{cultura de brogrammer} (\textit{bro culture}) em ambientes de tecnologia, caracterizada por comportamentos excludentes e vieses inconscientes que minam a autoconfiança feminina. A percepção de que o ambiente de estudo ou trabalho é hostil pode atuar como um forte fator de desengajamento. Conforme alertam \citeonline{SelmanHousein2025}, intervenções focadas apenas no recrutamento são insuficientes se não houver mudanças na estrutura de suporte e na cultura interna dos programas de formação, sugerindo que a retenção no Programa \textit{Hackers} do Bem dependerá da capacidade de mitigar esses fatores culturais através de diretrizes de inclusão intencionais.

\subsection{A Importância das \textit{Soft Skills} na Retenção}
\label{ss_importancia_soft_skills_retencao}

Uma lacuna crítica identificada na formação tradicional em cibersegurança é a subvalorização das competências comportamentais e interpessoais em detrimento das habilidades puramente técnicas (\textit{hard skills}). O relatório \textit{State of Cybersecurity 2025} destaca que, além do conhecimento técnico, atributos como automotivação (32\%), liderança (31\%), ética de trabalho (30\%) e comunicação (21\%) são considerados essenciais pelos empregadores para o desempenho eficaz das funções de segurança \cite{ISACA2025}. A ausência de desenvolvimento intencional dessas competências nos currículos de formação cria um descompasso entre a preparação acadêmica e a realidade da gestão de crises e defesa cibernética.

No recorte de gênero, \citeonline{Benson2025} demonstram que as mulheres frequentemente aportam competências vitais para o setor, como resiliência relacional, empatia, colaboração e pensamento inovador. Contudo, os autores argumentam que os \textit{frameworks} de competências e os processos de avaliação em cursos de cibersegurança falham ao não formalizar essas \textit{soft skills} como requisitos de carreira. Essa invisibilidade curricular contribui para a baixa autoeficácia feminina (QP3), pois as estudantes não veem suas aptidões de liderança e resolução colaborativa de problemas serem validadas em ambientes focados exclusivamente em \textit{flags} técnicas e competição individual.

A integração de \textit{soft skills} não é apenas uma questão de ajuste curricular, mas uma estratégia de permanência. \citeonline{Musuva2025} observam no programa \textit{Cyber Shujaa} que a combinação de treinamento técnico com mentoria voltada para habilidades de vida e carreira foi determinante para a empregabilidade e retenção das participantes. \citeonline{Benson2025} corroboram essa visão, sugerindo que a criação de espaços que fomentem a comunicação e o suporte mútuo atua como um fator de proteção contra o isolamento profissional. Para o estudo de caso do Programa \textit{Hackers} do Bem, isso implica investigar se o sistema de gamificação atual, focado em pontuação técnica (\textit{XP}), oferece mecanismos suficientes para reconhecer e recompensar essas competências não técnicas, ou se a sua ausência atua como um fator silencioso de evasão feminina.

\section{Metodologias de Ensino e Gamificação}
\label{s_metodologias_ensino_gamificacao}

Para escalar o ensino de cibersegurança, programas modernos recorrem a metodologias ativas e gamificação. Esta seção define os conceitos que sustentam a arquitetura pedagógica do programa analisado.

\subsection{\acf{GBL}}
\label{ss_gamificacao_aprendizagem_baseada_jogos}

É fundamental distinguir a gamificação da \acf{GBL}. Segundo a revisão sistemática conduzida por \citeonline{Coenraad2020}, a \ac{GBL} refere-se ao uso de jogos completos (digitais ou analógicos) como o meio principal de instrução, no qual o conteúdo de aprendizado é intrínseco à mecânica do jogo. Em contraste, a gamificação envolve a aplicação de elementos de \textit{design} de jogos — como sistemas de pontos, medalhas (\textit{badges}) e listas de classificação (\textit{leaderboards}) — em contextos não lúdicos, como plataformas de ensino ou ambientes de trabalho, visando motivar comportamentos específicos.

No contexto da educação em cibersegurança, \citeonline{Coenraad2020} alertam para uma distinção qualitativa na aplicação dessas estratégias. Ao analisarem 181 jogos educativos, os autores identificaram que muitas iniciativas se limitam a \enquote{quizzes gamificados} focados em segurança do usuário final (\textit{cyber safety}), falhando em promover um engajamento profundo com o conteúdo técnico (\textit{deep content engagement}). Para a formação de profissionais (QP2), a literatura sugere que a eficácia pedagógica depende da capacidade do sistema em promover o engajamento epistêmico, permitindo que o estudante assuma a identidade e as funções de um profissional de segurança em cenários realistas, em vez de apenas memorizar conceitos para obter recompensas extrínsecas.

Sob a perspectiva de inclusão de gênero, a estrutura da gamificação desempenha um papel crítico. \citeonline{Costa2023} propõem o conceito de \enquote{Gamificação Narrativa}, onde os desafios técnicos são envoltos em um enredo (\textit{storytelling}) e os alunos assumem papéis investigativos com propósito social. Seus estudos empíricos demonstram que essa abordagem, ao contextualizar o problema técnico, reduz a ansiedade tecnológica e aumenta a autoeficácia de estudantes do sexo feminino. Corroborando essa visão, \citeonline{Gough2024} argumentam que abordagens colaborativas e o uso de \textit{scaffolding} (apoio estruturado) são preferíveis a modelos puramente competitivos, pois mitigam as barreiras de entrada que frequentemente afetam grupos sub-representados em ambientes de alta pressão.

\subsection{\acf{CTFs} e Competições}
\label{ss_ctfs_competicoes}

As competições do tipo \acf{CTF} consolidaram-se como o padrão \textit{de facto} para o treinamento prático e a avaliação de competências técnicas em segurança da informação. Segundo \citeonline{Cole2022}, estes exercícios diferenciam-se de laboratórios tradicionais por possuírem duas características essenciais: um \enquote{caminho de solução subespecificado}, que exige do aluno a descoberta autônoma das etapas de resolução, e um \enquote{símbolo tangível de sucesso} (a \textit{flag}), que valida inequivocamente a conclusão do desafio.

A literatura categoriza os \ac{CTFs} em formatos distintos, sendo os mais relevantes para o ensino massivo:

\begin{itemize}
    \item \textbf{\textit{Jeopardy}:} Desafios estáticos segregados por categorias (como Criptografia, Web, Forense e Engenharia Reversa). Neste modelo, os participantes resolvem problemas independentes para obter pontos. \citeonline{Tshekiso2025} argumentam que este formato é o mais adequado para iniciantes e para escala massiva, pois permite o aprendizado progressivo e \textit{feedback} imediato sem a complexidade de interação direta com adversários. É o formato tipicamente adotado em eventos competitivos promovidos por programas de capacitação, como os relatados nos eventos do \textit{Hackers} do Bem (ex: \enquote{CTF Para Elas}) \cite{PalestraHDB}.

    \item \textbf{\textit{Attack-Defense}:} Equipes possuem sua própria rede ou servidor para defender, enquanto tentam atacar a infraestrutura das equipes adversárias em tempo real. Este formato exige conhecimentos avançados e coordenação de time, simulando um cenário de ciberguerra ativo \cite{Cole2022}.
\end{itemize}

Embora eficazes para o desenvolvimento de \textit{hard skills}, a aplicação universal destes modelos apresenta desafios sob a ótica da inclusão (QP2). \citeonline{Horcher2021} alerta que os \ac{CTFs} tradicionais frequentemente refletem o viés cultural de seus criadores, priorizando dinâmicas de \enquote{soma zero} e agressividade competitiva. A autora sugere que este ambiente pode desestimular participantes de grupos sub-representados, como mulheres, que tendem a apresentar maior engajamento em contextos que valorizam a colaboração e a resolução de problemas com propósito social, em detrimento da pura dominação técnica de um oponente.

Adicionalmente, \citeonline{Costa2023} introduzem o conceito de \enquote{\textit{Hack Quest}} ou gamificação narrativa, onde os desafios de estilo \textit{Jeopardy} são encapsulados em uma história investigativa. Estudos indicam que essa abordagem reduz a ansiedade de desempenho e aumenta a autoeficácia feminina, transformando a busca pela \textit{flag} em uma missão de justiça ou proteção, alinhando a competência técnica a um contexto de relevância social (QP1). No estudo de caso desta dissertação, investiga-se como a estrutura do \textit{Hackers} do Bem, que combina trilhas de aprendizado em simuladores com eventos competitivos de CTF, interage com essas diferentes percepções de competência e motivação.

\subsection{Impacto do Ranqueamento na Inclusão}
\label{ss_impacto_ranqueamento_inclusao}

A utilização de listas de classificação pública (\textit{leaderboards}) e sistemas de ranqueamento é amplamente debatida na literatura de educação em cibersegurança. Embora tais mecanismos visem promover a competitividade e o reconhecimento técnico, estudos recentes indicam que eles podem exercer efeitos adversos sobre a autoeficácia de grupos sub-representados. \citeonline{Horcher2021} argumenta que as mecânicas tradicionais de pontuação, frequentemente desenhadas sob uma lógica de \enquote{soma zero} (onde para um ganhar, outro deve perder), refletem um viés cultural que favorece perfis socializados para a agressividade competitiva, em detrimento daqueles que valorizam a aprendizagem colaborativa.

No contexto específico da experiência feminina, \citeonline{Hogan2025} observam que mulheres em competições de cibersegurança frequentemente relatam uma pressão adicional para performar acima da média, visando contrapor estereótipos negativos sobre sua competência técnica. O estudo destaca que a exposição em rankings pode exacerbar a ansiedade de desempenho e a síndrome do impostor, levando algumas participantes a preferir o anonimato ou a autoexclusão de ambientes onde a comparação é pública e constante. \citeonline{Costa2025} corroboram essa visão (QP3), demonstrando que ambientes que substituem a competição direta por narrativas colaborativas apresentam taxas superiores de engajamento e retenção feminina.

Para o estudo de caso desta dissertação, a análise do impacto do ranqueamento é crítica, uma vez que o Programa \textit{Hackers} do Bem utiliza a pontuação acumulada (\ac{XP}) e o desempenho em atividades práticas como critérios eliminatórios para o acesso às vagas restritas da Residência Tecnológica \cite{HackersDoBemManual}. A literatura sugere que, se não houver mecanismos de compensação que valorizem competências colaborativas ou o progresso individual (\textit{mastery orientation}) em vez da simples posição relativa (\textit{performance orientation}), o algoritmo de seleção pode inadvertidamente filtrar candidatas qualificadas que, devido a barreiras de autoeficácia (QP1), optam por não competir no ambiente de alta pressão descrito por \citeonline{SelmanHousein2025}.

\section{O Programa \textit{Hackers} do Bem}
\label{s_programa_hackers_bem}

Nesta dissertação, o Programa \textit{Hackers} do Bem é analisado como estudo de caso para investigar a aplicação de estratégias de capacitação massiva e seus impactos na retenção de talentos femininos. O programa não é o objeto de intervenção direta (alteração de plataforma ou regras), mas a fonte de dados primária para o diagnóstico do funil de formação e para a avaliação das diretrizes propostas.

\subsection{Objetivos e Estrutura da Formação}
\label{ss_objetivos_estrutura_formacao}

Instituído no contexto da Lei de TICs (Lei nº 8.248/1991) e alinhado aos objetivos da Estratégia Nacional de Cibersegurança (E-Ciber) de formação em escala \cite{Decreto12573, Lei8248}, o Programa \textit{Hackers} do Bem tem como meta a formação de recursos humanos qualificados em cibersegurança e privacidade. Sua execução é realizada pela \acf{RNP} em parceria com o SENAI-SP e sob a coordenação da Softex \cite{MCTI2025}.

A arquitetura pedagógica do programa foi desenhada para atender a um público massivo, estruturando-se em cinco níveis progressivos de complexidade, que funcionam como um funil de especialização, conforme detalhado no Manual do Aluno \cite{HackersDoBemManual}:

\begin{itemize}
    \item \textbf{Nivelamento e Básico:} Etapas de entrada, abertas ao público geral e realizadas na modalidade de ensino a distância (\ac{EAD}) assíncrono. O objetivo é uniformizar o conhecimento técnico em redes, sistemas operacionais, lógica de programação e fundamentos de segurança, permitindo o acesso de estudantes sem formação prévia na área.

    \item \textbf{Fundamental:} Esta etapa marca a transição para um modelo híbrido, introduzindo atividades síncronas e laboratórios práticos assistidos. O acesso a esta fase é limitado a um número específico de vagas, preenchidas por meio de processos de seleção baseados no desempenho das fases anteriores.

    \item \textbf{Especializado:} Nível voltado para o aprofundamento técnico, no qual os alunos optam por trilhas específicas de carreira, tais como: \textit{Red Team} (Segurança Ofensiva), \textit{Blue Team} (Defesa e Monitoramento), Resposta a Indicidentes e Forense, Governança, Riscos e Conformidade (GRC) e Desenvolvimento Seguro (\textit{DevSecOps}).

    \item \textbf{Residência Tecnológica:} A etapa final consiste em uma imersão prática supervisionada, onde os estudantes atuam em projetos reais ou simulados nos Pontos de Presença da \ac{RNP}, visando a transição direta para o mercado de trabalho.
\end{itemize}
Esta estrutura de múltiplos estágios permite que o programa atue simultaneamente na conscientização massiva (nos níveis iniciais) e na formação de especialistas de alta performance (nos níveis finais), respondendo à demanda da E-Ciber por capacitação em larga escala. No entanto, a transição entre o modelo assíncrono e o síncrono impõe barreiras de seleção que serão analisadas nesta pesquisa sob a ótica da retenção feminina.

\subsection{Modelo de Gamificação e Critérios de Aprovação}
\label{ss_modelo_gamificacao_criterios_aprovacao}

O avanço dos estudantes entre os níveis do Programa \textit{Hackers} do Bem não ocorre de maneira automática, sendo regido por um sistema híbrido de gamificação estrutural e avaliação acadêmica. Conforme detalhado no Manual de Aprovação \cite{HackersDoBemManual}, o modelo utiliza a acumulação de Pontos de Experiência (\ac{XP}) para mensurar o engajamento, enquanto notas em atividades práticas e teóricas aferem a competência técnica. A classificação para as fases com vagas limitadas (Fundamental, Especializado e Residência) depende estritamente da posição do aluno em um ranking geral, configurando um mecanismo de seleção competitivo.

Esta arquitetura de progressão é analisada nesta pesquisa sob a ótica das QPs 2 e 3, investigando se a ênfase na competição por vagas atua como fator motivacional ou como barreira de permanência para o público feminino. O sistema opera em duas camadas distintas: a progressão visual (Emblemas) e a progressão funcional (Ranqueamento).

\subsubsection{Mecânica de Pontuação e Emblemas}
\label{sss_mecanica_pontuacao_emblemas}

A unidade fundamental de progresso no \ac{AVA} do Programa \textit{Hackers} do Bem é o Ponto de Experiência (\ac{XP}). Conforme a taxonomia apresentada por \citeonline{Coenraad2020}, este modelo classifica-se como \enquote{gamificação estrutural} (\textit{structural gamification}), caracterizada pela aplicação de elementos de \textit{design} de jogos — como pontos e recompensas — ao redor do conteúdo instrucional, sem necessariamente alterar a natureza do material de estudo. Neste sistema, cada interação do aluno, como o consumo de videoaulas, a leitura de \textit{e-books} e a realização de \textit{quizzes} de fixação, atribui uma quantidade predeterminada de pontos ao seu perfil.

O acúmulo destes pontos desbloqueia emblemas (\textit{badges}) que atuam como indicadores visuais de senioridade e progressão na trilha formativa. A hierarquia de emblemas, descrita no manual do programa, estabelece a seguinte trajetória de evolução do estudante \cite{HackersDoBemManual}:

\begin{itemize}
    \item \textbf{\enquote{Aspirante a Hacker}:} Concedido ao atingir a pontuação correspondente à conclusão do Nivelamento;

    \item \textbf{\enquote{Explorador de Códigos}:} Concedido no início da etapa Básica;

    \item \textbf{\enquote{Mestre de Segurança}:} Concedido após a conclusão do nível Fundamental;

    \item \textbf{\enquote{Guerreiro Digital}:} Concedido aos concluintes do nível Especializado;

    \item \textbf{\enquote{Hacker Lendário}:} Nível máximo, reservado aos participantes da Residência Tecnológica.
\end{itemize}

Sob a ótica das questões de pesquisa desta dissertação (QP2 e QP3), a análise deste mecanismo é relevante pois a literatura sugere que recompensas extrínsecas, como emblemas, operam de forma distinta em diferentes grupos demográficos. \citeonline{Coenraad2020} alertam que a gamificação estrutural, se não acompanhada de narrativas que promovam o \enquote{engajamento epistêmico} (onde o aluno assume a identidade profissional), pode falhar em sustentar a retenção a longo prazo. Nesta pesquisa, os dados de engajamento (\ac{XP}) serão correlacionados com as taxas de evasão feminina para verificar se esses elementos visuais de conquista atuam como fatores de motivação eficazes ou se são insuficientes para mitigar as barreiras de autoeficácia enfrentadas pelas participantes.

\subsubsection{O Funil de Ranqueamento e Classificação}
\label{sss_funil_ranqueamento_classificacao}

Enquanto os cursos de Nivelamento e Básico operam em modelo assíncrono massivo, a transição para o nível Fundamental marca o início do gargalo estrutural do programa, denominado nesta dissertação como \enquote{funil de seleção}. O acesso às aulas síncronas, mentorias e laboratórios avançados é restrito a um número limitado de vagas, preenchidas através de um algoritmo de ranqueamento.

Em cenários onde o número de candidatos aptos supera o número de vagas — situação comum em programas de larga escala \cite{MCTI2025} — o manual estabelece critérios de desempate que priorizam a performance técnica. A ordem de prioridade para a classificação é:

\begin{enumerate}
    \item Maior nota acumulada nas Atividades Práticas (laboratórios e simuladores);

    \item Maior nota na Avaliação Final (teórica);

    \item Maior nota nos \textit{Quizzes};

    \item Maior pontuação de engajamento (\ac{XP}).
\end{enumerate}

Este modelo cria um ambiente de alta pressão por performance, onde o erro em atividades práticas penaliza diretamente a posição do aluno no ranking. \citeonline{Hogan2025} e \citeonline{Horcher2021} alertam que sistemas competitivos de \enquote{soma zero}, onde a ascensão de um participante implica na queda de outro, podem exacerbar a ansiedade e a síndrome do impostor em mulheres, levando à autoexclusão mesmo quando elas possuem a competência técnica necessária. Nesta pesquisa, os dados de evasão nas transições de nível serão correlacionados com essas métricas de ranqueamento para verificar a existência de disparidades de gênero no funil.

\subsection{Visão Geral do Fluxo de Avaliação e Progressão}
\label{ss_visao_geral_fluxo}

Para consolidar o entendimento sobre a mecânica de seleção do Programa \textit{Hackers} do Bem, apresenta-se uma visão sistêmica do fluxo de avaliação. Conforme ilustrado na \autoref{fig:fluxo_avaliacao}, a jornada do estudante inicia-se no Curso de Nivelamento, caracterizado pelo acesso universal e progressão automática mediante a conclusão do conteúdo. Nesta etapa inicial, o foco reside na uniformização de conhecimentos basilares, sem a imposição de barreiras competitivas ou limitadores de vagas.
A transição crítica ocorre após a conclusão do Curso Básico. Neste ponto, o sistema abandona a lógica de \enquote{avanço livre} e introduz o algoritmo de ranqueamento. O diagrama evidencia que a aprovação acadêmica (nota mínima) é uma condição necessária, mas não suficiente para a continuidade. O fator determinante torna-se o cálculo da pontuação acumulada (\ac{XP}) e o posicionamento do candidato dentro do limite rígido de vagas disponíveis para a etapa síncrona (Fundamental).
O fluxo demonstra que os candidatos classificados avançam para a convocação e matrícula no Curso Fundamental, seguindo posteriormente para o Especializado e a Residência Tecnológica. Por outro lado, aqueles que atingem os critérios acadêmicos mas não obtêm pontuação suficiente para o corte de vagas não são reprovados, mas sim direcionados para uma lista de espera, aguardando uma \enquote{próxima onda} de ofertas. Esta estrutura visualiza o conceito de funil abordado nesta pesquisa, onde a gamificação atua como o mecanismo regulador do fluxo de entrada nas fases de maior especialização.

\begin{figure}[htb]
    \centering
    \caption{Fluxograma do sistema de avaliação e progressão}
    \label{fig:fluxo_avaliacao}
    \includegraphics[width=0.9\textwidth]{figs/fluxo_progresso.png}
    \fonte{Elaborada pela autora com base nos processos do programa.}
\end{figure}

\subsection{Ações de Engajamento e Eventos Práticos}
\label{ss_acoes_engajamento_hdb}

Como parte das estratégias para fomentar o ecossistema de inovação em cibersegurança e incentivar a aplicação prática dos conhecimentos adquiridos, o programa Hackers do Bem promoveu um total de 13 eventos ao longo do seu desenvolvimento. Dentre essas iniciativas, destacam-se a realização de 3 \textit{CyberGames}, 4 \ac{CTFs}, sendo um deles exclusivo para mulheres, 3 \textit{Hackathons} e 3 \textit{Workshops} focados no debate sobre a formação de profissionais na área. A promoção de eventos competitivos e de imersão atuou não apenas como ferramenta de avaliação prática, mas também como vetor de atração e engajamento dos estudantes \cite{PalestraHDB}.

Para lidar com a sub-representação feminina e estimular a permanência das mulheres no programa, foram adotadas ações afirmativas específicas. Um marco importante foi a realização do \enquote{CTF Para Elas}, um evento de competição de segurança exclusivo para o público feminino, ocorrido em abril na Universidade de São Paulo (USP), que contou com a participação de 73 mulheres. A iniciativa teve apoio financeiro do Google, o que viabilizou o custeio de viagens para que residentes do programa de diferentes regiões do Brasil (Norte, Nordeste, Centro-Oeste e Sul) pudessem participar presencialmente em São Paulo. O evento foi antecedido por um ciclo de palestras focado no desenvolvimento de carreira e no compartilhamento de experiências de profissionais de destaque, contando inclusive com executivas do Google. A coordenação do programa notou que a realização de eventos exclusivos atua como um facilitador, criando um ambiente seguro para o aprendizado e motivando as alunas a participarem posteriormente de competições mistas \cite{PalestraHDB}.

Além dos eventos dedicados, o programa implementou estratégias de gamificação diferenciada em suas competições gerais para incentivar a participação e o reconhecimento feminino. No 4º CTF do Hackers do Bem, realizado em Belém, estabeleceu-se um prêmio financeiro extra de R\$ 1.000,00 para a competidora presencial com a melhor colocação; nesta mesma edição, uma mulher (residente do próprio programa) alcançou o 3º lugar na classificação geral. Em outras frentes, como nas competições \textit{CyberGame}, as mulheres com as melhores pontuações de cada rodada recebiam premiações exclusivas, atuando como um reforço positivo à sua performance \cite{PalestraHDB}. 

Por fim, o suporte e o acolhimento estenderam-se ao ambiente virtual e à organização estrutural da capacitação. O \enquote{Hub Hackers do Bem} foi concebido como um ambiente totalmente digital de encontro para profissionais e alunos, disponibilizando conteúdos extras e fóruns focados em diversas áreas de especialização. Destaca-se, neste ecossistema, a criação de fóruns dedicados exclusivamente às mulheres, funcionando como uma rede de apoio mútuo onde as alunas podem se reunir, compartilhar dicas e trocar orientações de forma acolhedora. Aliada a essa plataforma, a formação de turmas síncronas exclusivas para alunas, guiadas por mentoras mulheres, demonstrou ser uma estratégia fundamental e eficaz, elevando o engajamento e a taxa de progressão das participantes para as fases mais avançadas (curso Especializado e Residência Tecnológica) \cite{PalestraHDB}.

\section{Considerações Finais}
\label{s_cap2_consideracoes_finais}

A fundamentação teórica apresentada neste capítulo evidencia que o desafio da formação em cibersegurança no Brasil transcende a questão quantitativa do déficit de profissionais. A Estratégia Nacional de Cibersegurança (E-Ciber), instituída pelo Decreto nº 12.573/2025 \cite{Decreto12573}, estabelece a necessidade imperativa de capacitação em larga escala, uma demanda que o Programa \textit{Hackers} do Bem busca atender através de uma arquitetura pedagógica massiva, assíncrona e gamificada.

Contudo, a literatura analisada aponta para uma tensão latente entre os mecanismos de escalabilidade técnica e as diretrizes de inclusão de gênero. Enquanto a gamificação estrutural — baseada em pontos de experiência (\ac{XP}), emblemas e \ac{CTFs} estilo \textit{Jeopardy} — viabiliza a avaliação automatizada de milhares de alunos simultaneamente (QP2), estudos como os de \citeonline{Horcher2021} e \citeonline{Hogan2025} alertam que essas mesmas dinâmicas podem atuar como barreiras de entrada. A ênfase em competição de \enquote{soma zero} e rankings públicos corre o risco de exacerbar a ansiedade de desempenho e diminuir a autoeficácia de mulheres, que frequentemente valorizam abordagens colaborativas e contextualizadas socialmente (QP3).

O fenômeno do \textit{leaky pipeline}, discutido na Seção \ref{s_desigualdade_genero_ciberseguranca}, materializa-se no estudo de caso através da redução acentuada da participação feminina nas etapas de transição do programa (caindo de 22,45\% na entrada para cerca de 13\% nas fases especializadas). A teoria sugere que a intersecção entre barreiras culturais — como a falta de modelos de referência e a desvalorização das \textit{soft skills} \cite{Benson2025} — e um \textit{design} instrucional de alta pressão pode estar criando um filtro não intencional, que exclui candidatas qualificadas antes que atinjam a residência tecnológica.

Desta forma, consolida-se a premissa teórica desta dissertação: a neutralidade algorítmica dos critérios de ranqueamento do Programa \textit{Hackers} do Bem não garante, por si só, a equidade de oportunidades. Este capítulo forneceu os subsídios conceituais — abrangendo desde as políticas públicas até a psicologia da gamificação — necessários para compreender as variáveis que serão investigadas na análise dos dados e na proposição das diretrizes de inclusão.