\chapter{Fundamentação Teórica}
\label{c_fundamentacao_teorica}

Este capítulo apresenta os conceitos fundamentais necessários para a compreensão do problema de pesquisa e da análise realizada. A fundamentação está dividida em quatro eixos principais: o cenário global e nacional da força de trabalho em cibersegurança; as disparidades de gênero e barreiras na área tecnológica; as metodologias de ensino, com foco em gamificação e competições; e o detalhamento do objeto de estudo, o Programa \textit{Hackers} do Bem.

\section{Cenário da Força de Trabalho em Cibersegurança}
\label{s_cenario_forca_trabalho_ciberseguranca}

A escassez de profissionais qualificados em segurança da informação é um desafio global que impacta diretamente a capacidade de defesa de organizações e nações. Nesta seção, discute-se a evolução do déficit de talentos (\textit{workforce gap}) e as políticas públicas recentes para mitigação.

\subsection{O Déficit Global e Nacional de Talentos}
\label{ss_deficit_global_nacional_talentos}

Dados recentes do \textit{ISC2 Cybersecurity Workforce Study} indicam que a lacuna global da força de trabalho atingiu a marca histórica de 4 milhões de profissionais, representando um aumento expressivo em relação aos anos anteriores \cite{ISC2_2024}. Este déficit é calculado pela diferença entre a demanda das organizações para proteger adequadamente seus ativos e a oferta real de profissionais disponíveis para contratação.

É importante, no entanto, qualificar a natureza desta escassez. Conforme observam \citeonline{Musuva2025} ao analisarem programas de capacitação, o problema não reside apenas na ausência de candidatos, mas na desconexão entre as competências possuídas pelos estudantes e as habilidades técnicas específicas exigidas pelo mercado. O fenômeno, frequentemente descrito como \textit{skills mismatch}, sugere que a massificação do ensino sem o devido alinhamento com as demandas práticas da indústria, resulta em posições não preenchidas, mesmo em cenários de desemprego estrutural.

O \ac{WEF}, em seu relatório \textit{Global Cybersecurity Outlook 2025}, reforça este cenário ao destacar que a iniquidade da força de trabalho afeta desproporcionalmente setores críticos com menos recursos, como educação, governo e \ac{PMEs} \cite{WEF2025}. No contexto brasileiro, o \ac{WEF} aponta que, embora o país esteja avançando na maturidade cibernética através de iniciativas como a E-Ciber, a escassez de talentos permanece um gargalo, que limita a resiliência das infraestruturas nacionais, frente à sofisticação crescente das ameaças.

Portanto, a resolução deste déficit exige estratégias que ultrapassem o recrutamento tradicional, focando no desenvolvimento de competências práticas e na retenção de talentos, uma vez que a simples expansão numérica da força de trabalho, sem a devida qualificação técnica e comportamental, mostra-se insuficiente para mitigar os riscos cibernéticos atuais. Segundo dados recentes \cite{ISC2_2024}, o déficit global ultrapassa milhões de profissionais, com o Brasil figurando como um dos países com a maior carência de mão de obra especializada.

\subsection{Políticas Públicas e a Estratégia Nacional (E-Ciber)}
\label{ss_politicas_publicas_estrategia_nacional}

No contexto brasileiro, a resposta governamental à escassez de mão de obra qualificada materializou-se através da instituição da Estratégia Nacional de Cibersegurança (E-Ciber), formalizada pelo Decreto nº 12.573, de 4 de agosto de 2025 \cite{Decreto12573}. Este instrumento legal redefine a governança do setor e estabelece a educação e a capacitação massiva como pilares fundamentais para a garantia da soberania digital do país.

A E-Ciber está estruturada em eixos temáticos que orientam as ações do Estado. Destaca-se o Eixo 4 (Soberania Nacional e Governança), cujo Artigo 9º, inciso III, determina explicitamente a necessidade de "formação e capacitação técnico-profissional em cibersegurança em escala compatível com as necessidades nacionais" \cite{Decreto12573}. Esta diretriz legal justifica a transição de ações de treinamento isoladas para iniciativas de grande porte, visando reduzir o débito tecnológico do país.

Adicionalmente, o Eixo 1 (Proteção e Conscientização) preconiza o incentivo à inclusão de temas de cibersegurança nos currículos educacionais e a capacitação de profissionais, buscando criar uma cultura de segurança sustentável na sociedade \cite{GSINacional}. É neste cenário regulatório que se insere o Programa \textit{Hackers} do Bem, objeto deste estudo. A iniciativa não surge isoladamente, mas como uma resposta operacional às demandas da E-Ciber e da Política Nacional de Cibersegurança. Financiado pelo \ac{MCTI} no âmbito do \ac{PPI} da Softex, e executado pela \ac{RNP} em parceria com o \ac{SENAI}, o programa visa estruturar um ecossistema nacional de segurança digital, integrando formação, pesquisa e inovação para também atender à escala demandada pela legislação vigente \cite{MCTI2025}.

\section{Desigualdade de Gênero na Cibersegurança}
\label{s_desigualdade_genero_ciberseguranca}

A sub-representação feminina na cibersegurança não é apenas uma questão de diversidade, mas de eficiência econômica e inovação. Esta seção explora as barreiras estruturais e culturais que impedem a entrada e a permanência de mulheres na área.

\subsection{Panorama da Representatividade Feminina}
\label{ss_panorama_representatividade_feminina}

Apesar dos avanços na última década, as mulheres ainda representam uma parcela minoritária da força de trabalho em cibersegurança. Estudos como o de \citeonline{Pacheco2024} indicam que a participação feminina estagnou em torno de 25\%, com disparidades ainda maiores em cargos de liderança técnica, onde mulheres ocupam menos de 20\% das posições de \ac{CISO} nas grandes corporações globais.

Esta estagnação não é uniforme em todo o setor de tecnologia. Conforme analisado por \citeonline{SelmanHousein2025}, ao comparar a cibersegurança com outras subáreas da computação, observa-se que o índice de 25\% é superior ao de campos emergentes como Inteligência Artificial (12\%) e Robótica (11\%), porém significativamente inferior à área de Educação em Computação, onde a representatividade feminina atinge 42\%.

Estes dados sugerem que a barreira não é apenas tecnológica, mas cultural e estrutural, específica do domínio de segurança. A persistência desse \textit{gap} ao longo dos anos, mesmo com o aumento da demanda por profissionais, corrobora com a existência de um fenômeno de "vazamento de talentos" (\textit{leaky pipeline}), onde as mulheres que ingressam na área encontram dificuldades de permanência e ascensão profissional superiores às observadas em outros setores da \ac{TI} \cite{SelmanHousein2025}.

\subsection{Barreiras de Entrada e o Fenômeno do \textit{Leaky Pipeline}}
\label{ss_barreiras_entrada_fenomeno_leaky_pipeline}

O conceito de \textit{Leaky Pipeline} (vazamento de talentos) descreve a perda progressiva de participação feminina ao longo da jornada educacional e profissional. \citeonline{SelmanHousein2025} identificam que a falta de modelos de referência (\textit{role models}), a percepção de um ambiente hostil e a ausência de mentorias estruturadas são fatores determinantes para a evasão precoce em programas de formação. A revisão da literatura aponta que, mesmo quando o interesse inicial existe, a retenção decai drasticamente entre o ensino médio e a especialização profissional devido à falta de suporte contínuo.

Esta evasão é intensificada por barreiras culturais e estereótipos profundamente enraizados na área. \citeonline{Costa2025} argumentam que a imagem predominante do profissional de cibersegurança, associada ao estereótipo do "\textit{hacker} de capuz" isolado ou à cultura excludente do \textit{"brogrammer"}, cria uma dissonância cognitiva para mulheres, que não se enxergam pertencentes a esse grupo. Tais estereótipos perpetuam a falsa noção de que a competência técnica em segurança, exige um perfil comportamental antissocial ou agressivamente competitivo.

Consequentemente, a ausência de representatividade feminina em posições de liderança e a carência de redes de mentoria reforçam este ciclo de exclusão. Sem intervenções deliberadas que desconstruam essas imagens e ofereçam suporte vocacional, como as propostas em programas de capacitação inclusivos, o funil de talentos continuará a perder candidatas qualificadas, antes mesmo de sua entrada formal no mercado de trabalho \cite{SelmanHousein2025, Costa2025}.

\subsection{A Importância das \textit{Soft Skills} na Retenção}
\label{ss_importancia_soft_skills_retencao}

A valorização excessiva de competências puramente técnicas (\textit{hard skills}) em detrimento de habilidades comportamentais e analíticas (\textit{soft skills}) atua como uma barreira de entrada. A percepção da cibersegurança como uma área restrita à operação técnica de baixo nível, ignora a complexidade das ameaças modernas, que exigem gestão de crise, comunicação estratégica e liderança adaptativa.

\citeonline{Benson2025} argumentam que a reformulação dos currículos para integrar competências como comunicação, gestão de riscos e análise forense é uma estratégia eficaz para aumentar a atratividade da área para o público feminino. Seu estudo qualitativo com profissionais da área na Europa destaca que mulheres frequentemente demonstram níveis elevados de resiliência relacional, empatia e colaboração, atributos essenciais para a resposta a incidentes e gestão de equipes diversificadas.

No entanto, os autores identificam uma lacuna crítica: \textit{frameworks} de competências amplamente utilizados, como o \ac{ECSF}, falham ao não formalizar essas \textit{soft skills} como requisitos de carreira, perpetuando processos de recrutamento e avaliação enviesados para o técnico.

Consequentemente, programas de capacitação que focam exclusivamente em pontuação técnica (como \ac{CTFs} tradicionais), podem inadvertidamente desviar participantes que possuem o perfil analítico e comunicacional demandado pelo mercado, mas que não se identificam com a cultura puramente técnica. A integração formal de \textit{soft skills} na trilha de aprendizado não apenas valida a competência feminina, mas fortalece a resiliência organizacional como um todo.

\section{Metodologias de Ensino e Gamificação}
\label{s_metodologias_ensino_gamificacao}

Para escalar o ensino de cibersegurança, programas modernos recorrem a metodologias ativas e gamificação. Esta seção define os conceitos que sustentam a arquitetura pedagógica do programa analisado.

\subsection{\acf{GBL}}
\label{ss_gamificacao_aprendizagem_baseada_jogos}

A gamificação consiste na aplicação de elementos de design de jogos em contextos não lúdicos para aumentar o engajamento e a motivação. É fundamental distinguir este conceito de \ac{GBL} ou Jogos Sérios: enquanto a gamificação incorpora mecânicas isoladas (como pontuação) em um processo de ensino tradicional, o \ac{GBL} utiliza o jogo completo como o próprio meio de instrução.

A implementação mais comum de gamificação em plataformas de ensino baseia-se na tríade \ac{PBL}. Neste modelo, o aluno acumula \ac{XP} por tarefas concluídas, desbloqueia emblemas (\textit{badges}) ao atingir marcos de competência e compara seu desempenho com outros participantes através de rankings (\textit{leaderboards}). Estas mecânicas visam fornecer \textit{feedback} imediato e fomentar a competitividade saudável, elementos centrais na arquitetura do programa objeto deste estudo.

Em sua revisão sistemática, \citeonline{Coenraad2020} demonstram que o uso de pontuações, emblemas e narrativas imersivas em cibersegurança facilita a abstração de conceitos técnicos complexos e promove a aprendizagem ativa. Os autores analisaram 181 jogos digitais e identificaram que, embora eficazes para a conscientização inicial (\textit{cyber safety}), muitas iniciativas limitam-se a "quizzes gamificados" que falham em oferecer um engajamento profundo com o conteúdo técnico (\textit{deep content engagement}).

Segundo \citeonline{Coenraad2020}, para que a gamificação seja efetiva na formação profissional, ela deve transcender a recompensa extrínseca (pontos) e promover o engajamento epistêmico, onde o aluno assume o papel e a identidade do profissional de segurança (ex: o defensor de uma rede), simulando a prática real da profissão.

\subsection{\acf{CTFs} e Competições}
\label{ss_ctfs_competicoes}

As competições do tipo \ac{CTFs} consolidaram-se como o padrão para o treinamento prático e a avaliação de competências em segurança ofensiva e defensiva. Conforme categorizado por \citeonline{Cole2022}, estes exercícios dividem-se predominantemente em dois formatos: \textit{Jeopardy}, onde equipes resolvem desafios estáticos segregados por categorias (como criptografia, forense e exploração web), a fim de obter pontuações cumulativas; e \textit{Attack-Defense}, onde participantes defendem sua própria infraestrutura, enquanto tentam comprometer os sistemas adversários em tempo real.

No âmbito do Programa \textit{Hackers} do Bem, o modelo \textit{Jeopardy} constitui a base das avaliações práticas e do sistema de gamificação, sendo o principal mecanismo de aferição técnica para o ranqueamento dos alunos entre as fases de especialização \cite{HackersDoBemManual}.

Entretanto, a literatura aponta limitações críticas na aplicação universal deste modelo. \citeonline{Horcher2021} alerta que o formato tradicional de \ac{CTFs}, frequentemente focado em velocidade, agressividade competitiva e cenários de "soma zero", reflete um viés cultural que pode afetar grupos sub-representados. A autora argumenta que mulheres tendem a apresentar melhor desempenho e retenção em ambientes que priorizam a colaboração e a resolução analítica de problemas, sugerindo que a ausência de adaptações no design das competições atua como uma barreira de entrada estrutural, e não como um filtro de competência.

\subsection{Impacto do Ranqueamento na Inclusão}
\label{ss_impacto_ranqueamento_inclusao}

A utilização de rankings públicos (\textit{leaderboards}) é controversa na literatura de educação inclusiva. Enquanto promovem a competitividade, podem reduzir a autoeficácia de participantes que não se veem representados no topo da lista. \citeonline{Costa2025} demonstram que ambientes de aprendizado que priorizam a colaboração sobre a competição direta apresentam melhores taxas de retenção feminina.

A Teoria da Autoeficácia sugere que a confiança na própria capacidade de realizar tarefas é um preditor fundamental da persistência em áreas técnicas. Em contextos gamificados onde o \textit{feedback} é predominantemente comparativo (posição relativa no ranking) em vez de formativo (evolução individual), participantes de grupos sub-representados tendem a subestimar suas competências, especialmente quando expostos a estereótipos de gênero que associam a cibersegurança a um domínio masculino \cite{Costa2025}.

Corroborando esta visão, \citeonline{Horcher2021} argumenta que as mecânicas tradicionais de pontuação em \ac{CTFs}, frequentemente desenhadas para recompensar a velocidade e a agressividade ("soma zero"), favorecem perfis comportamentais socializados para a competição direta. A autora alerta que a falta de adaptação destas mecânicas para valorizar a resolução colaborativa de problemas pode aumentar a ansiedade de desempenho e atuar como um fator de exclusão.

No contexto do Programa \textit{Hackers} do Bem, onde o ranqueamento é um critério eliminatório para o acesso às vagas da Residência Tecnológica \cite{HackersDoBemManual}, a análise destes fatores torna-se crítica para compreender se o modelo de avaliação está medindo apenas a competência técnica ou se está, inadvertidamente, filtrando candidatos com base em seu perfil de competitividade.

\section{O Programa \textit{Hackers} do Bem}
\label{s_programa_hackers_bem}

O Programa Hackers do Bem, instituído pelo \acf{MCTI} e executado pela \acf{RNP} em parceria com o Senai e a Softex, constitui o objeto de estudo central desta dissertação.

\subsection{Objetivos e Estrutura da Formação}
\label{ss_objetivos_estrutura_formacao}

O programa estabeleceu como meta prioritária a capacitação de mais de 30 mil alunos no nível básico, além de formar 3 mil profissionais em nível intermediário e especializar cerca de 200 alunos em níveis avançados \cite{MCTI2025}. Para atingir tal escala, a arquitetura pedagógica foi dividida em cinco níveis progressivos de complexidade, combinando ensino assíncrono massivo com atividades síncronas práticas.

Conforme detalhado no Manual de Aprovação e Gamificação \cite{HackersDoBemManual}, a estrutura curricular organiza-se da seguinte forma:

\begin{enumerate}
    \item \textbf{Nivelamento (80h):} Etapa de entrada, aberta ao público geral e realizada de forma assíncrona. O currículo abrange fundamentos de \textit{hardware}, sistemas operacionais (\textit{Windows} e \textit{Linux}), redes de computadores (TCP/IP), lógica de programação e desenvolvimento de \textit{scripts}. O objetivo é uniformizar o conhecimento técnico dos candidatos antes das fases específicas de segurança.
    
    \item \textbf{Básico (64h):} Também na modalidade assíncrona, esta fase introduz conceitos centrais de cibersegurança, incluindo identificação de ameaças, \textit{malwares}, criptografia, e fundamentos de \ac{GRC}. A aprovação nesta etapa é pré-requisito para o ranqueamento nas fases subsequentes.
    
    \item \textbf{Fundamental (96h):} A partir deste nível, o programa adota um formato híbrido com aulas ao vivo e laboratórios práticos. O conteúdo aprofunda-se em arquitetura segura, autenticação, segurança em nuvem e resposta a incidentes. O acesso é limitado a um número específico de vagas, preenchidas via ranqueamento baseado no desempenho das fases anteriores.
    
    \item \textbf{Especializado (80h):} Nesta etapa, os alunos optam por uma trilha de especialização técnica específica. As trilhas disponíveis, conforme o cronograma do programa, são:
    \begin{itemize}
        \item \textit{\textit{Red Team}} (Segurança Ofensiva e \textit{Pentest});
        \item \textit{Blue Team} (Defesa e Monitoramento);
        \item Resposta a Incidentes e Forense Computacional;
        \item \textit{DevSecOps} (Desenvolvimento Seguro);
        \item \acl{GRC}.
    \end{itemize}
    
    \item \textbf{Residência Tecnológica (6 meses):} A fase final consiste em uma imersão prática intensiva, onde os alunos atuam em projetos reais ou simulados, visando a transição direta para o mercado de trabalho.
\end{enumerate}

O modelo de progressão entre estas fases não é automático. O programa utiliza um sistema de gamificação e ranqueamento (detalhado na \autoref{ss_ctfs_competicoes}) como mecanismo de filtro, onde o desempenho em \textit{quizzes} e atividades práticas determina a elegibilidade do aluno para avançar do nível Básico para o Fundamental, e assim sucessivamente, criando um funil de seleção meritocrático \cite{HackersDoBemManual}.

\subsection{Modelo de Gamificação e Critérios de Aprovação}
\label{ss_modelo_gamificacao_criterios_aprovacao}

O avanço entre os níveis do programa não é automático, sendo regido por um sistema de gamificação e ranqueamento. Conforme o Manual \cite{HackersDoBemManual}, os alunos acumulam \acf{XP} através do consumo de conteúdo, realização de \textit{quizzes} e atividades práticas. A classificação para as fases subsequentes, especialmente para a Residência, depende estritamente da posição do aluno no ranking geral, calculado pela soma de \ac{XP} e critérios de desempate baseados no desempenho em avaliações práticas. 

Este mecanismo de "funil competitivo" é o ponto focal da análise de dados que será apresentada nos capítulos seguintes. A arquitetura de gamificação do programa foi desenhada para operar em duas camadas distintas: a progressão visual (Emblemas e Selos) e a progressão funcional (Ranqueamento).

\subsubsection{Mecânica de Pontuação e Emblemas}
\label{sss_mecanica_pontuacao_emblemas}

A unidade fundamental de progresso no sistema é o \ac{XP}. Cada interação do aluno com a plataforma, seja a visualização de uma videoaula até a conclusão de um laboratório prático, atribui uma quantidade predeterminada de pontos. O acúmulo destes pontos desbloqueia emblemas de nível, que servem como indicadores visuais de proficiência, conforme detalhado na \autoref{tab:emblemas_xp} \cite{HackersDoBemManual}.

\begin{table}[htb]
\centering
\caption{Estrutura de Emblemas e Pontuação do Programa}
\label{tab:emblemas_xp}
\begin{tabular}{l|l|l}
\hline
\textbf{Nível do Emblema} & \textbf{\ac{XP} Necessário} & \textbf{Estágio Associado} \\ \hline
Aspirante a Hacker & 251 pontos & Conclusão do Nivelamento \\ \hline
Explorador de Códigos & 603 pontos & Início do Básico \\ \hline
Mestre de Segurança & 2.413 pontos & Conclusão do Fundamental \\ \hline
Guerreiro Digital & 3.413 pontos & Conclusão do Especializado \\ \hline
Hacker Lendário & A definir & Residência Tecnológica \\ \hline
\end{tabular}
\fonte{Adaptado de \citeonline{HackersDoBemManual}.}
\end{table}

É importante ressaltar que os emblemas possuem caráter motivacional e não garantem, isoladamente, a aprovação técnica. A aprovação em cada curso exige o cumprimento de requisitos acadêmicos específicos, como nota mínima de 60\% nas avaliações finais e nas atividades práticas obrigatórias \cite{HackersDoBemManual}.

\subsubsection{O Funil de Ranqueamento e Classificação}
\label{sss_funil_ranqueamento_classificacao}

Enquanto os cursos de Nivelamento e Básico operam em modelo assíncrono com vagas ilimitadas, a transição para o nível Fundamental marca o início do gargalo estrutural do programa. O acesso às aulas síncronas e laboratórios avançados é restrito a um número limitado de vagas (definido por "ondas" de oferta), exigindo a aplicação de um algoritmo de seleção.

O ranqueamento utiliza a pontuação acumulada como critério primário. Em cenários de empate, o manual estabelece uma hierarquia rigorosa de critérios de desempate para a seleção dos candidatos, priorizando a competência técnica sobre o consumo de conteúdo:

\begin{enumerate}
    \item Maior nota nas Atividades Práticas (laboratórios);
    \item Maior nota na Avaliação Final (prova teórica);
    \item Maior nota nos \textit{Quizzes} de fixação;
    \item Maior pontuação no consumo de conteúdo (\ac{XP} de engajamento).
\end{enumerate}

Por fim, o modelo de avaliação cria um ambiente onde o "erro" na avaliação prática penaliza o aluno no ranking final, impactando diretamente suas chances de acesso à Residência Tecnológica. A seleção ocorre em três chamadas (listas classificatórias), onde as vagas não preenchidas na primeira lista são redistribuídas para os candidatos subsequentes no ranking \cite{HackersDoBemManual}. Um resumo do fluxo de avaliação pode ser verificado na \autoref{fig:fluxo_progresso}.

\begin{figure}[htb]
    \centering
    \caption{Fluxo de avaliação do programa}
    \label{fig:fluxo_progresso}
    \includegraphics[width=0.8\textwidth]{figs/fluxo_progresso.png}
    \fonte{Elaborada pela autora}
\end{figure}