\chapter{Proposta}
\label{c_proposta}

Neste capítulo, detalha-se a solução proposta para responder às questões de pesquisa definidas na introdução: o desenvolvimento de um \textbf{\textit{Framework} de Diretrizes para Inclusão e Retenção em Escala}. A proposta consiste em correlacionar os dados quantitativos de desempenho e engajamento do Programa \textit{Hackers} do Bem com a análise qualitativa da percepção das participantes, fundamentando as diretrizes em evidências empíricas.
\section{O Framework de Diretrizes Proposto}
\label{s_framework_proposto}

A solução a ser desenvolvida é um artefato conceitual denominado \textbf{\textit{Framework} de Diretrizes para Retenção em Escala}. Este modelo foi desenhado para preencher a lacuna identificada na literatura entre a \enquote{pedagogia do acolhimento} (que exige alto nível de interação humana) e a \enquote{eficiência da escala} (necessária para reduzir o déficit de profissionais).
Com base na \ac{RSL} (Capítulo 3), o \textit{framework} estrutura-se preliminarmente em três eixos estratégicos, que serão refinados a partir da análise dos dados do estudo de caso:

\begin{itemize}
    \item \textbf{Eixo 1: Contextualização Narrativa:} Propõe diretrizes para a camada de apresentação do conteúdo, orientando a transição de exercícios técnicos abstratos para narrativas investigativas com propósito social, visando reduzir a barreira cultural de entrada.

    \item \textbf{Eixo 2: Mecânicas de Avaliação Inclusiva:} Visa reformular a interpretação das métricas de gamificação (\ac{XP} e Ranking). A proposta estabelece diretrizes para valorizar a \enquote{Eficiência de Engajamento} (progresso individual) em detrimento da pura comparação competitiva pública.

    \item \textbf{Eixo 3: Suporte Social Escalável:} Estabelece regras para a criação de redes de suporte em ambientes virtuais, operacionalizando a mentoria e o suporte de pares de forma escalável dentro da plataforma.
\end{itemize}

\section{Visão Geral da Metodologia}
\label{s_visao_geral}
A presente pesquisa classifica-se como aplicada e exploratória, adotando uma abordagem de métodos mistos. O estudo utiliza o Programa \textit{Hackers} do Bem como ambiente de validação empírica e fonte primária de dados.

Para a execução da pesquisa, serão observadas as normas éticas vigentes, com a devida submissão do projeto ao Comitê de Ética em Pesquisa (CEP) da Univali, via Plataforma Brasil, para a validação dos instrumentos de coleta de dados qualitativos \cite{UnivaliCEP}. O acesso à base de dados institucional será realizado mediante autorização formal da coordenação do programa.

\section{Estratégia de Coleta e Análise}
\label{s_estrategia_coleta}
Para responder às questões de pesquisa e refinar os eixos do \textit{framework}, a metodologia divide-se em duas frentes complementares de investigação:

\subsection{Frente 1: Mineração de Dados (O \enquote{Onde} e o \enquote{Quando})}
\label{ss_frente1}

Nesta etapa, serão analisados os registros acadêmicos e de interação da plataforma para mapear o \enquote{funil de evasão}. O objetivo é identificar estatisticamente em qual momento exato ocorre a maior perda de participantes femininas e se este fenômeno está correlacionado às regras de negócio do programa.

\begin{itemize}
    \item \textbf{Análise Volumétrica:} Comparativo da taxa de abandono entre as fases de Nivelamento (assíncrono/sem concorrência) e Fundamental (síncrono/competitivo).

    \item \textbf{Correlação de Desempenho:} Verificação da existência de participantes com bom desempenho técnico (notas altas) que desistem antes ou durante as fases de ranqueamento, o que indicaria barreiras não cognitivas.
\end{itemize}

\subsection{Frente 2: Levantamento de Percepção (O \enquote{Porquê})}
\label{ss_frente2}

Será aplicado um questionário eletrônico direcionado às mulheres que participaram do programa (tanto as que persistiram quanto as que evadiram). O instrumento buscará validar três construtos identificados na literatura:

\begin{enumerate}
    \item \textbf{Autoeficácia em Cibersegurança:} Medir a confiança da aluna em realizar as tarefas propostas, independentemente da nota atribuída pelo sistema.

    \item \textbf{Percepção de Competitividade:} Avaliar se o \textit{leaderboard} (ranking público) e a mecânica de pontos atuaram como fator motivacional ou gerador de ansiedade.

    \item \textbf{Pertencimento:} Verificar se a participante percebeu a existência de suporte social e valorização de competências colaborativas durante a formação.
\end{enumerate}

\section{Metodologia de Análise de Dados}
\label{s_metodologia_analise}
A análise quantitativa fundamenta-se na definição de métricas específicas que permitem isolar as variáveis de gamificação e desempenho técnico. A formalização matemática destas métricas visa garantir a objetividade na comparação entre os gêneros.

\subsection{Definição das Métricas}
\label{ss_definicao_metricas}

Define-se o conjunto total de participantes como $P$, onde $P_f \subset P$ representa o subconjunto de participantes do gênero feminino e $P_m \subset P$ o subconjunto do gênero masculino.

\subsubsection{Taxa de Evasão e Gap de Gênero}

A taxa de evasão ($\epsilon$) mensura a perda de participantes entre o início e o fim de um módulo específico. Seja $N_i$ o número de inscritos no início da fase e $N_c$ o número de concluintes aprovados, a métrica é definida pela Equação \ref{eq:taxa_evasao}:

\begin{equation}
    \epsilon = 1 - \left( \frac{N_{c}}{N_{i}} \right) \label{eq:taxa_evasao}
\end{equation}

A partir desta métrica, calcula-se o \textit{Gap de Retenção por Gênero} (GRG), subtraindo-se a taxa de evasão feminina da masculina. Um valor positivo indica que a evasão feminina é superior, evidenciando gargalos específicos naquela etapa do funil.

\subsubsection{Eficiência de Engajamento Gamificado ($\gamma$)}

Considerando que o programa utiliza \ac{XP} como critério de ranqueamento, é necessário medir se o acúmulo de pontos reflete a competência técnica de forma equitativa ou se privilegia perfis competitivos. Define-se a Eficiência de Engajamento ($\gamma$) como a razão entre a nota obtida nas atividades práticas ($Nota_{pratica}$) e o total de \ac{XP} acumulado ($XP_{total}$), conforme Equação \ref{eq:eficiencia_xp}:

\begin{equation}
    \gamma = \frac{Nota_{pratica}}{XP_{total}} \label{eq:eficiencia_xp}
\end{equation}

Uma alta taxa $\gamma$ sugere que a participante possui alta competência técnica (nota alta) com menor dependência dos mecanismos de engajamento massivo (XP baixo). Esta métrica será utilizada para testar a hipótese de que mulheres tendem a focar na resolução do problema e menos na competição por pontos.

\subsection{Variáveis de Análise}
\label{ss_sintese_variaveis}

A \autoref{tab:variaveis_analise} apresenta o resumo das variáveis que compõem o estudo.

\begin{table}[htb]
    \centering
    \caption{Variáveis e Métricas definidas para a análise}
    \label{tab:variaveis_analise}
    \begin{tabular}{l|l|l}
        \hline
        \textbf{Tipo} & \textbf{Variável/Métrica} & \textbf{Descrição e Fonte de Dados} \\ \hline
        Independente & Gênero (g) & Autodeclaração no cadastro (M/F) \\ \hline
        Independente & Fase do Funil (f) & Nivelamento, Básico, Fundamental, Residência \\ \hline
        Dependente & Score de Autoeficácia ($S_{ae}$) & Média obtida via questionário (Escala Likert) \\ \hline
        Dependente & Evasão ($\epsilon$) & Status de desistência antes da conclusão \\ \hline
        Dependente & Ranking (R) & Posição final na lista classificatória \\ \hline
        Dependente & Desempenho Técnico ($D_t$) & Notas em laboratórios práticos e simuladores \\ \hline
    \end{tabular}
    \fonte{Elaborada pela autora.}
\end{table}

\section{Estratégia de Análise do Funil}
\label{s_estrategia_analise_funil}
A avaliação da retenção será conduzida através de uma análise sequencial, modelada para espelhar a arquitetura de progressão do Programa \textit{Hackers} do Bem. O \textit{pipeline} de análise decompõe o fluxo em quatro estágios de transição críticos:

\begin{itemize}
    \item \textbf{Transição $T_1$ (Nivelamento $\rightarrow$ Básico):} Análise de desistência espontânea em fase assíncrona, onde não há barreiras de entrada competitivas.

    \item \textbf{Transição $T_2$ (Básico $\rightarrow$ Fundamental):} Este é o primeiro ponto crítico, onde o ranqueamento por \ac{XP} e notas define o acesso às vagas limitadas. A análise verificará se o critério de \ac{XP} afeta desproporcionalmente a classificação feminina.

    \item \textbf{Transição $T_3$ (Fundamental $\rightarrow$ Especializado):} Nesta fase, analisa-se a \enquote{Mecânica de Ondas} e listas de espera. O objetivo é identificar se mulheres são alocadas majoritariamente em listas de espera e se o tempo de aguardo impacta sua decisão de continuar no programa.

    \item \textbf{Transição $T_4$ (Especializado $\rightarrow$ Residência):} A etapa final envolve a escolha de trilhas (Red Team, Blue Team, GRC). A análise focará na segregação vertical, verificando se há concentração feminina em trilhas de gestão em detrimento de trilhas técnicas ofensivas/defensivas.
\end{itemize}

\section{Validação da Proposta}
\label{s_validacao_proposta}

A validação do \textit{Framework} proposto será realizada através da triangulação entre os achados quantitativos e a avaliação de especialistas e participantes.

\begin{enumerate}
    \item \textbf{Painel de Especialistas:} O \textit{framework} será submetido à avaliação de especialistas em educação em cibersegurança e diversidade. Será utilizado um instrumento estruturado para avaliar a clareza, a pertinência e a viabilidade de implementação das diretrizes propostas em escala nacional.

    \item \textbf{Validação de Percepção:} As diretrizes refinadas serão apresentadas a uma amostra de participantes do programa, que avaliarão se as intervenções propostas (ex: mudança na narrativa dos desafios, alteração no peso do ranking) teriam impacto positivo em sua intenção de permanência.
\end{enumerate}

\section{Definição dos Instrumentos de Coleta (Abordagem GQM)}
\label{s_definicao_gqm}

Para garantir o alinhamento entre os objetivos da pesquisa e os dados coletados nos questionários, adota-se o paradigma \textit{Goal-Question-Metric} (GQM). Originalmente proposto por \citeonline{Basili1994} para engenharia de software, o GQM é amplamente utilizado em pesquisas acadêmicas para definir métricas de avaliação de forma hierárquica, partindo de um objetivo conceitual (Nível Conceitual), refinando-o em questões operacionais (Nível Operacional) e definindo as métricas quantitativas ou qualitativas necessárias para respondê-las (Nível Quantitativo).

Nesta dissertação, o GQM é aplicado para estruturar os dois instrumentos de coleta primária: o questionário de percepção voltado às participantes do Programa \textit{Hackers} do Bem e o questionário de validação voltado aos especialistas.

\subsection{Instrumento 1: Percepção das Participantes}
\label{ss_gqm_participantes}

O objetivo deste instrumento é capturar os fatores subjetivos que influenciam a permanência ou a evasão feminina, validando as hipóteses levantadas na Revisão Sistemática da Literatura.

\begin{itemize}
    \item \textbf{Objetivo de Medição (Goal):} Analisar a mecânica de gamificação e a estrutura pedagógica do Programa \textit{Hackers} do Bem, com o propósito de caracterizar o seu impacto na autoeficácia e na motivação, sob o ponto de vista das participantes do gênero feminino, no contexto de um treinamento massivo em cibersegurança.
\end{itemize}

O desdobramento deste objetivo nas questões e métricas, associadas às hipóteses de pesquisa, é apresentado no \autoref{quadro:gqm_participantes}.

\begin{quadro}[htb]
    \centering
    \caption{Definição do GQM para o Questionário das Participantes}
    \label{quadro:gqm_participantes}
    \begin{footnotesize}
    \begin{tabular}{|p{4.5cm}|p{5cm}|p{4.5cm}|}
        \hline
        \textbf{Questão (Question)} & \textbf{Métrica / Indicador (Metric)} & \textbf{Hipótese Esperada} \\ \hline
        \textbf{Q1:} Como o sistema de classificação pública (\textit{ranking/leaderboard}) afeta a ansiedade de desempenho das participantes? & Escala Likert (1-5): \enquote{A visualização da minha posição no ranking me causa ansiedade.} & Baseado em \citeonline{Hogan2025}, espera-se que o ranking público tenha correlação positiva com a ansiedade em mulheres, ao contrário de feedbacks privados. \\ \hline
        \textbf{Q2:} A participante percebe a narrativa e o contexto dos desafios como relevantes para problemas sociais? & Escala Likert (1-5): \enquote{Sinto que as atividades do curso ajudam a resolver problemas reais da sociedade.} & Conforme \citeonline{Costa2025}, uma baixa percepção de impacto social correlaciona-se com menor engajamento feminino (dissonância de identidade). \\ \hline
        \textbf{Q3:} Qual é a percepção de autoeficácia técnica ao falhar em uma atividade prática? & Escala Likert (1-5): \enquote{Quando erro um exercício, sinto que não tenho capacidade para a área.} & \citeonline{Benson2025} sugerem que mulheres tendem a atribuir a falha à falta de capacidade interna (baixa autoeficácia) mais rapidamente que homens. \\ \hline
        \textbf{Q4:} A existência de canais de suporte ou comunidades (fóruns/grupos) influenciou a decisão de permanecer no curso? & Questão Binária (Sim/Não) + Campo Aberto para descrição do impacto. & Baseado em \citeonline{Musuva2025}, espera-se que o suporte social seja citado como fator determinante de retenção (fator de proteção). \\ \hline
    \end{tabular}
    \end{footnotesize}
    \fonte{Elaborado pela autora com base em \citeonline{Basili1994} e na RSL.}
\end{quadro}

\subsection{Instrumento 2: Validação com Especialistas}
\label{ss_gqm_especialistas}

Após a estruturação do \textit{Framework} de Diretrizes, este será submetido a um painel de especialistas para validação de conteúdo e viabilidade.

\begin{itemize}
    \item \textbf{Objetivo de Medição (Goal):} Avaliar o \textit{Framework} de Diretrizes proposto, com o propósito de validar sua viabilidade técnica e potencial de impacto, sob o ponto de vista de especialistas em educação em cibersegurança e gestores de programas, no contexto da implementação em larga escala.
\end{itemize}

As questões norteadoras para os especialistas buscam mitigar o risco de propor soluções pedagogicamente ideais, mas tecnicamente inviáveis para o volume de alunos do \textit{Hackers} do Bem.

\begin{quadro}[htb]
    \centering
    \caption{Definição do GQM para o Painel de Especialistas}
    \label{quadro:gqm_especialistas}
    \begin{footnotesize}
    \begin{tabular}{|p{4.5cm}|p{5cm}|p{4.5cm}|}
        \hline
        \textbf{Questão (Question)} & \textbf{Métrica / Indicador (Metric)} & \textbf{Hipótese / Critério} \\ \hline
        \textbf{Q1:} As diretrizes de \enquote{Contextualização Narrativa} são aplicáveis sem descaracterizar o rigor técnico do conteúdo? & Escala Likert (1-5) de Viabilidade Técnica. & Especialistas podem apontar resistência se a narrativa for percebida como infantilização do conteúdo (risco apontado por \citeonline{Casey2023}). \\ \hline
        \textbf{Q2:} A proposta de \enquote{Suporte Social Escalável} (ex: mentoria por pares) é gerenciável em um ambiente com mais de 30.000 alunos? & Escala Likert (1-5) de Escalabilidade e Custo-Benefício. & Conforme \citeonline{Tshekiso2025}, soluções que dependem de intervenção humana síncrona tendem a ser avaliadas como de baixa escalabilidade. \\ \hline
        \textbf{Q3:} O conjunto de diretrizes cobre as principais barreiras de gênero conhecidas? & Checklist de Cobertura (Barreiras Culturais, Estruturais e Pedagógicas). & O \textit{framework} deve endereçar não apenas o acesso, mas a permanência e a identidade profissional \cite{SelmanHousein2025}. \\ \hline
    \end{tabular}
    \end{footnotesize}
    \fonte{Elaborado pela autora.}
\end{quadro}

\subsection{Procedimento de Aplicação}

Os formulários serão implementados em plataforma digital e distribuídos anonimamente. Para as alunas do Programa \textit{Hackers} do Bem, a amostragem será por conveniência, divulgada nos canais oficiais de comunicação do programa (Discord/E-mail), visando atingir tanto alunas ativas quanto evadidas. Para os especialistas, será utilizada uma amostragem intencional, selecionando perfis com experiência comprovada em gestão de programas de capacitação tecnológica ou pesquisa em gênero em \ac{STEM}.

\section{Cronograma de Execução}
\label{s_cronograma}

O planejamento da pesquisa considera os trâmites de aprovação ética e a disponibilidade dos dados.

\begin{quadro}[htb]
    \centering
    \caption{Cronograma de Atividades}
    \label{qua:cronograma}
    \begin{tabular}{|l|c|c|c|c|}
        \hline
        \textbf{Atividade} & \textbf{Fev} & \textbf{Mar} & \textbf{Abr} & \textbf{Mai} \\ \hline
        Submissão ao CEP e Ajustes & X & & & \\ \hline
        Solicitação e Acesso aos Dados & X & & & \\ \hline
        Coleta (Mineração e Questionários) & X & X & & \\ \hline
        Análise de Dados e Refinamento do Framework & & X & & \\ \hline
        Validação (Especialistas/Participantes) & & & X & \\ \hline
        Escrita Final e Defesa & & & X & X \\ \hline
    \end{tabular}
    \fonte{Elaborado pela autora.}
\end{quadro}

\chapter{Considerações Finais}
\label{c_consideracoes_finais}

A presente pesquisa de mestrado insere-se em um contexto de urgência estratégica para o Brasil. A instituição da Política Nacional de Cibersegurança e da Estratégia Nacional de Cibersegurança (E-Ciber), por meio do Decreto nº 12.573/2025, oficializou a necessidade de formação de profissionais em \enquote{escala compatível com as necessidades nacionais} \cite{Decreto12573}. No entanto, a literatura e os dados preliminares indicam que a massificação do ensino, se não desenhada com intencionalidade inclusiva, corre o risco de replicar ou até exacerbar as disparidades de gênero que historicamente caracterizam o setor de tecnologia.
Ao longo da fundamentação teórica e da revisão sistemática, identificou-se uma lacuna crítica no estado da arte: a existência de uma dicotomia entre intervenções de \enquote{alto toque humano} (mentorias, grupos pequenos, currículos personalizados), comprovadamente eficazes para a retenção feminina, e a \enquote{eficiência da escala} (plataformas massivas, avaliação automatizada, ranqueamento), necessária para atender à demanda de mercado. O Programa \textit{Hackers} do Bem, ao adotar um modelo baseado em gamificação competitiva e funis de seleção por desempenho, situa-se no centro desta tensão, tornando-se o estudo de caso ideal para investigar os limites e as possibilidades da inclusão em larga escala.

\section{Síntese da Proposta}
\label{s_sintese_proposta}

Para responder a este desafio, esta dissertação propõe o desenvolvimento e a validação do \textbf{\textit{Framework} de Diretrizes para Inclusão e Retenção em Escala}. Diferentemente de soluções que focam apenas na atração (marketing) ou em ações pontuais, a proposta busca intervir na arquitetura pedagógica e nas regras de negócio dos programas de formação.

A abordagem metodológica mista, detalhada no Capítulo 4, foi desenhada para ir além da percepção superficial. Ao cruzar os dados quantitativos de evasão e desempenho no funil do programa (mineração de dados) com a análise qualitativa da autoeficácia das participantes (survey), a pesquisa permitirá isolar variáveis críticas. Busca-se responder se a evasão feminina ocorre por falta de competência técnica ou se é induzida por mecanismos de \textit{design} — como a pressão de \textit{rankings} públicos e a ausência de narrativas contextualizadas — que afetam desproporcionalmente a confiança deste grupo.

\section{Contribuições Esperadas}
\label{s_contribuicoes_esperadas}

Com a execução deste cronograma de pesquisa, espera-se entregar contribuições tanto para a academia quanto para a sociedade:

\begin{itemize}
    \item \textbf{Contribuição Científica:} O fornecimento de evidências empíricas sobre o impacto da gamificação competitiva na retenção de mulheres em cursos massivos (\ac{MOOCs}) de cibersegurança. A pesquisa pretende preencher a lacuna identificada por \citeonline{SelmanHousein2025} sobre a escassez de métricas rigorosas de avaliação de intervenções de gênero.

    \item \textbf{Contribuição Técnica:} A formalização de um \textit{framework} contendo diretrizes operacionais para os três eixos identificados (Narrativa, Avaliação e Suporte). Este artefato servirá de guia para que gestores educacionais e formuladores de políticas públicas possam desenhar programas que equilibrem a eficiência da automação com o acolhimento necessário para a diversidade.

    \item \textbf{Contribuição Social:} Ao propor caminhos para reduzir o \textit{leaky pipeline} no Programa \textit{Hackers} do Bem, o trabalho colabora diretamente para o aumento da participação feminina na força de trabalho, alinhando-se aos objetivos de desenvolvimento social e redução de desigualdades.
\end{itemize}

\section{Próximos Passos}
\label{s_proximos_passos}
A pesquisa encontra-se em fase de qualificação, com a fundamentação teórica consolidada e a metodologia definida. Os próximos passos, conforme o cronograma apresentado, concentram-se na submissão do protocolo ao Comitê de Ética em Pesquisa (CEP), na obtenção da base de dados junto à coordenação do programa e na execução da análise diagnóstica. A validação das diretrizes junto a especialistas e participantes encerrará o ciclo, culminando na defesa da dissertação e na entrega de um modelo replicável para a educação em cibersegurança no Brasil.