\chapter{Proposta de Pesquisa}
\label{c_proposta}

Neste capítulo, apresenta-se o plano de trabalho para o desenvolvimento da pesquisa aplicada, visando responder às questões de pesquisa (QP1, QP2, QP3 e QP4) definidas na introdução. A proposta consiste na análise quantitativa e qualitativa dos dados do Programa \textit{Hackers do Bem}, correlacionando-os com as diretrizes identificadas na literatura.

\section{Visão Geral e Premissas}
\label{s_visao_geral_premissas}

A presente pesquisa classifica-se como aplicada e exploratória, adotando uma abordagem de métodos mistos (\textit{mixed methods}) que triangula a análise quantitativa de dados educacionais com a análise qualitativa da percepção das participantes. O objetivo central é a proposição de um \textbf{Framework de Diretrizes de Inclusão}, fundamentado em evidências empíricas do Programa \textit{Hackers do Bem}.

Para a viabilização e integridade ética deste estudo, estabelecem-se as seguintes premissas e procedimentos mandatórios:

\begin{enumerate}
    \item \textbf{Protocolo Ético e Acesso aos Dados:} O acesso à base de dados dos participantes não é automático. A execução da pesquisa está condicionada à submissão e aprovação do projeto pelo Comitê de Ética em Pesquisa (CEP) da Univali, via Plataforma Brasil, conforme as normas institucionais vigentes \cite{UnivaliCEP}. Após a aprovação ética, será formalizada a solicitação de acesso aos dados junto à gestão do Programa \textit{Hackers do Bem} (RNP/Softex/Senai).
    
    \item \textbf{Privacidade e LGPD:} Todo o tratamento de dados respeitará rigorosamente a Lei Geral de Proteção de Dados Pessoais (Lei nº 13.709/2018). A análise quantitativa utilizará exclusivamente dados pseudoanonimizados, garantindo que nenhuma participante seja identificada individualmente nos resultados divulgados. Para a coleta de dados primários (formulários), será aplicado o Termo de Consentimento Livre e Esclarecido (TCLE).
    
    \item \textbf{Rastreabilidade da Gamificação:} Premissa-se que a plataforma de ensino registra de forma granular as interações dos alunos. Conforme o Manual de Aprovação \cite{HackersDoBemManual}, o sistema contabiliza XP, emblemas e notas, dados essenciais para correlacionar o desempenho técnico com as taxas de evasão feminina.
\end{enumerate}

\subsection{Estratégia de Coleta e Análise (Sugestão Metodológica)}
\label{ss_estrategia_coleta}

Para responder às questões de pesquisa (QP2 e QP3), a metodologia será dividida em duas frentes complementares:

\subsubsection{Frente 1: Mineração de Dados (O "Onde" e o "Quando")}
Nesta etapa, serão analisados os dados brutos solicitados para mapear o "funil de evasão". O objetivo é identificar estatisticamente em qual momento exato ocorre a maior perda de participantes femininas.
\begin{itemize}
    \item \textbf{Métrica de Evasão:} Comparativo da taxa de abandono entre as fases de Nivelamento (assíncrono/sem concorrência) e Fundamental (síncrono/competitivo).
    \item \textbf{Correlação de Desempenho:} Verificação se mulheres com notas altas (competência comprovada) desistem antes da fase de Ranqueamento, indicando barreiras não-técnicas.
\end{itemize}

\subsubsection{Frente 2: Levantamento de Percepção (O "Porquê")}
Será aplicado um questionário eletrônico direcionado às mulheres que participaram do programa (tanto as que persistiram quanto as que evadiram). Sugere-se que o instrumento seja fundamentado em escalas validadas na literatura para medir três construtos identificados na RSL:

\begin{enumerate}
    \item \textbf{Autoeficácia em Cibersegurança:} Baseado em \citeonline{Costa2025}, para medir se a participante se sentia capaz de realizar as tarefas, independentemente da nota real.
    \item \textbf{Percepção de Competitividade:} Baseado em \citeonline{Horcher2021}, para avaliar se o \textit{leaderboard} (ranking público) e a mecânica de "soma zero" atuaram como fator motivacional ou ansiogênico.
    \item \textbf{Pertencimento e Soft Skills:} Baseado em \citeonline{Benson2025}, para verificar se a participante percebeu valorização de habilidades colaborativas durante a formação.
\end{enumerate}


\section{Metodologia de Análise de Dados (Design da Solução)}
\label{s_metodologia_analise}

A abordagem metodológica proposta para esta pesquisa é de natureza mista (\textit{mixed-methods}), estruturada em um \textit{pipeline} de processamento de dados que integra a mineração de dados educacionais (quantitativo) com a análise de percepção subjetiva (qualitativo). O "design da solução", neste contexto, refere-se à arquitetura analítica desenvolvida para mensurar o impacto das variáveis do Programa \textit{Hackers do Bem} na retenção feminina.

A estratégia de análise foi desenhada para isolar variáveis de gamificação (XP, Nível) e desempenho (Notas) e correlacioná-las com os pontos de evasão no funil de formação.

\subsection{Pipeline de Processamento e Fontes de Dados}
\label{ss_pipeline_dados}

A execução da pesquisa seguirá um fluxo de trabalho dividido em três camadas, conforme a arquitetura de referência para análise de dados educacionais: (1) Coleta e Anonimização, (2) Processamento de Métricas e (3) Correlação. As fontes de dados primárias e os procedimentos de tratamento estão definidos a seguir:

\begin{enumerate}
    \item \textbf{Base de Dados Institucional (RNP):} Dados transacionais da plataforma de ensino, contendo registros de acesso, pontuação de gamificação (XP), notas em avaliações e progresso por trilha. O acesso a estes dados será formalizado junto à gestão do programa, respeitando os protocolos de segurança da informação \cite{HackersDoBemManual}.
    \item \textbf{Dados de Percepção (Levantamento):} Coleta primária via formulários eletrônicos aplicados às participantes, focando em construtos de autoeficácia e percepção da competitividade.
\end{enumerate}

\subsection{Definição das Métricas e Variáveis}
\label{ss_metricas_variaveis}

Para quantificar o fenômeno da evasão e permitir a comparação objetiva entre os gêneros, foram definidas métricas específicas baseadas na literatura de avaliação de treinamento e gamificação.

\subsubsection{Métricas de Evasão e Retenção}
A taxa de retenção ($TR$) será calculada para cada fase do programa (Nivelamento, Básico, Fundamental, Especializado). Seja $N_{i}$ o número de inscritos no início da fase e $N_{f}$ o número de concluintes aprovados, a métrica é definida pela Equação \ref{eq:taxa_retencao}:

\begin{equation}
    TR_{fase} = \left( \frac{N_{f}}{N_{i}} \right) \times 100
    \label{eq:taxa_retencao}
\end{equation}

Adicionalmente, será calculado o \textit{Gap de Retenção por Gênero} ($GRG$), que mede a disparidade de sucesso entre participantes do gênero feminino ($TR_{fem}$) e masculino ($TR_{masc}$), permitindo identificar gargalos específicos:

\begin{equation}
    GRG = TR_{fem} - TR_{masc}
    \label{eq:gap_retencao}
\end{equation}

Valores negativos de $GRG$ indicarão fases onde o programa falha desproporcionalmente na retenção de mulheres.

\subsubsection{Métricas de Gamificação e Desempenho}
Para investigar a hipótese de que a competição afeta a permanência (QP3), será analisada a correlação entre o acúmulo de Pontos de Experiência ($XP$) e a taxa de evasão. Conforme o manual do programa, o $XP$ é acumulado por engajamento e acertos \cite{HackersDoBemManual}.

Define-se a métrica de \textit{Eficiência de Engajamento} ($EE$) para isolar alunas que possuem alto desempenho técnico, mas baixo engajamento nos elementos competitivos (ranking):

\begin{equation}
    EE = \frac{Nota_{pratica}}{XP_{total}}
    \label{eq:eficiencia_engajamento}
\end{equation}

Uma $EE$ alta sugere que a participante domina o conteúdo técnico (nota alta), mas não se engaja nas mecânicas de gamificação massiva (XP baixo), o que corrobora a teoria de \citeonline{Horcher2021} sobre a preferência feminina por aprendizado menos competitivo.

\subsection{Instrumentos de Coleta Qualitativa}
\label{ss_instrumentos_quali}

Complementando a análise métrica, será aplicado um questionário estruturado baseado em escalas Likert de 5 pontos (1 = Discordo Totalmente, 5 = Concordo Totalmente), similar à metodologia de validação utilizada na dissertação de referência. O instrumento visa validar se os dados quantitativos de evasão correspondem a uma percepção de baixa autoeficácia ou falta de pertencimento.

Os construtos avaliados serão:
\begin{itemize}
    \item \textbf{Autoeficácia Técnica:} Baseado em \citeonline{Costa2025}, mede a confiança da aluna em realizar tarefas de cibersegurança.
    \item \textbf{Ansiedade Competitiva:} Avalia o impacto dos \textit{leaderboards} (rankings) na motivação da aluna.
    \item \textbf{Percepção de Suporte:} Avalia a eficácia das mentorias e da comunidade.
\end{itemize}

\subsection{Aspectos Éticos e Tratamento de Dados}
\label{ss_aspectos_eticos}

Considerando que a pesquisa envolve dados de seres humanos, todo o procedimento seguirá as diretrizes da Resolução 466/12 do Conselho Nacional de Saúde. O projeto será submetido ao Comitê de Ética em Pesquisa (CEP) da Univali via Plataforma Brasil.

Para garantir a conformidade com a Lei Geral de Proteção de Dados (LGPD), será aplicado um processo de pseudoanonimização nos dados fornecidos pela RNP antes de qualquer análise estatística, removendo identificadores diretos (CPF, E-mail, Nome) e mantendo apenas os atributos de interesse (Gênero, Notas, XP, Região).


\subsection{Definição das Métricas e Variáveis}
\label{ss_metricas_variaveis}

Para a avaliação quantitativa da eficácia do programa na retenção de talentos femininos, foram estabelecidas métricas baseadas na análise de funil e correlação de desempenho. A formalização matemática destas métricas visa isolar o comportamento dos grupos em relação aos mecanismos de gamificação (XP) e ranqueamento descritos no Manual do Programa \cite{HackersDoBemManual}.

Define-se o conjunto total de participantes como $P$, onde $P_{f} \subset P$ representa o subconjunto de participantes do gênero feminino e $P_{m} \subset P$ o subconjunto do gênero masculino. As métricas de avaliação são detalhadas a seguir.

\subsubsection{Taxa de Evasão por Fase ($\epsilon$)}
A taxa de evasão ($\epsilon$) mensura a perda de participantes entre o início e o fim de um módulo específico (Nivelamento, Básico, Fundamental). Seja $N_{i}$ o número de inscritos no início da fase e $N_{c}$ o número de concluintes aprovados, a métrica é definida pela Equação \ref{eq:taxa_evasao}:

\begin{equation}
    \epsilon = 1 - \left( \frac{N_{c}}{N_{i}} \right)
    \label{eq:taxa_evasao}
\end{equation}

O objetivo é calcular o $\Delta\epsilon = \epsilon_{f} - \epsilon_{m}$, onde um valor positivo indica que a evasão feminina é superior à masculina, evidenciando gargalos específicos naquela etapa do funil.

\subsubsection{Eficiência de Engajamento Gamificado ($\gamma$)}
Considerando que o programa utiliza Pontos de Experiência (XP) como critério de ranqueamento, é necessário medir se o acúmulo de XP reflete a competência técnica de forma equitativa. Define-se a Eficiência de Engajamento ($\gamma$) como a razão entre a nota obtida nas atividades práticas ($Nota_{pratica}$) e o total de XP acumulado por consumo de conteúdo ($XP_{conteudo}$), conforme Equação \ref{eq:eficiencia_xp}:

\begin{equation}
    \gamma = \frac{Nota_{pratica}}{XP_{conteudo}}
    \label{eq:eficiencia_xp}
\end{equation}

Uma alta taxa $\gamma$ sugere que a participante possui alta competência técnica (nota alta) com menor dependência dos mecanismos de engajamento massivo (XP), validando a hipótese de \citeonline{Horcher2021} sobre perfis de aprendizado.

\subsubsection{Síntese das Variáveis de Análise}
A \autoref{tab:variaveis_analise} apresenta o resumo das variáveis independentes e dependentes que compõem o estudo, estruturadas para permitir a comparação entre os grupos, seguindo o modelo de organização de experimentos adotado em trabalhos anteriores \cite{Nunes2018}.

\begin{table}[htb]
\centering
\caption{Variáveis e Métricas definidas para os experimentos}
\label{tab:variaveis_analise}
\begin{tabular}{l|l|l}
\hline
\textbf{Tipo} & \textbf{Variável/Métrica} & \textbf{Descrição e Fonte de Dados} \\ \hline
Independente & Gênero ($g$) & Autodeclaração no cadastro (M/F). \\ \hline
Independente & Fase do Funil ($f$) & Nivelamento, Básico, Fundamental, Residência. \\ \hline
Dependente & Score de Autoeficácia ($S_{ae}$) & Média obtida via questionário (Escala Likert 1-5) \cite{Costa2025}. \\ \hline
Dependente & Evasão ($\epsilon$) & Status de desistência antes da conclusão do módulo. \\ \hline
Dependente & Ranking ($R$) & Posição final na lista classificatória (Critério de corte). \\ \hline
Dependente & Desempenho Técnico ($D_t$) & Notas em laboratórios práticos e CTFs. \\ \hline
\end{tabular}
\fonte{Elaborada pela autora.}
\end{table}

\subsection{Coleta e Tratamento dos Dados do Programa}
\label{ss_coleta_dados}

A estratégia de aquisição de dados para esta pesquisa foi desenhada para operar em duas frentes complementares: a extração de dados secundários (logs e registros acadêmicos) e a coleta de dados primários (levantamento de percepção), garantindo uma análise mista robusta sobre a retenção feminina.

\subsubsection{Extração de Dados da Plataforma (RNP)}
A fonte primária de dados quantitativos reside no Ambiente Virtual de Aprendizagem (AVA) do Programa \textit{Hackers do Bem}, gerenciado pela Rede Nacional de Ensino e Pesquisa (RNP). A coleta será realizada mediante solicitação formal à coordenação do programa, visando a obtenção de \textit{dumps} ou relatórios estruturados (formatos CSV ou JSON) contendo os registros de interação dos alunos.

Para viabilizar a análise do "funil de engajamento", serão solicitados os seguintes conjuntos de dados, conforme a estrutura curricular descrita no Manual de Aprovação \cite{HackersDoBemManual}:

\begin{itemize}
    \item \textbf{Dados Demográficos (Anonimizados):} Gênero declarado, faixa etária e região geográfica.
    \item \textbf{Dados de Desempenho:} Notas nas avaliações finais, pontuação em atividades práticas e status de aprovação/reprovação por módulo (Nivelamento, Básico, Fundamental e Especializado).
    \item \textbf{Métricas de Gamificação:} Total de Pontos de Experiência (XP) acumulados, emblemas desbloqueados (ex: "Aspirante a Hacker", "Mestre de Segurança") e frequência de interação com \textit{quizzes}.
    \item \textbf{Logs de Evasão:} Data da última interação registrada na plataforma, permitindo calcular o tempo de permanência antes do abandono.
\end{itemize}

\subsubsection{Levantamento de Percepção (Dados Primários)}
Complementarmente, será aplicado um questionário eletrônico voltado especificamente às participantes do gênero feminino (tanto evadidas quanto concluintes). Este instrumento visa capturar aspectos subjetivos não registrados nos logs do sistema, como a percepção de autoeficácia e o impacto do ambiente competitivo (ranqueamento) na motivação de aprendizado.

\subsubsection{Aspectos Éticos e Conformidade (LGPD e CEP)}
Considerando que a pesquisa envolve o tratamento de dados pessoais e a participação de seres humanos, todos os procedimentos seguirão rigorosamente as diretrizes éticas e legais vigentes.

O projeto será submetido à apreciação do Comitê de Ética em Pesquisa (CEP) da Universidade do Vale do Itajaí (Univali), via Plataforma Brasil, conforme os fluxos institucionais de validação documental e relatoria \cite{UnivaliCEP}. A coleta de dados primários será condicionada à assinatura digital do Termo de Consentimento Livre e Esclarecido (TCLE) pelas participantes.

No que tange à Lei Geral de Proteção de Dados (LGPD - Lei nº 13.709/2018), será aplicado um protocolo de pseudoanonimização nos dados fornecidos pela RNP antes de qualquer processamento analítico. Identificadores diretos (como Nome, CPF e E-mail) serão substituídos por chaves alfanuméricas únicas (\textit{hashs}), garantindo que a análise estatística de evasão e desempenho ocorra sem expor a identidade individual das alunas, preservando a privacidade e a segurança das informações \cite{Wangham2021}.

\subsection{Estratégia de Análise do Funil (Pipeline)}
\label{ss_analise_funil}

A avaliação da retenção feminina será conduzida através de um \textit{pipeline} de análise sequencial, modelado para espelhar a arquitetura de progressão do Programa \textit{Hackers do Bem}. O processo de formação não é linear, mas sim composto por "portões de controle" (\textit{gates}) que utilizam algoritmos de ranqueamento para filtrar os participantes aptos à próxima etapa.

A estratégia de análise consiste em decompor o fluxo de alunos em quatro estágios de transição crítica, confrontando os dados de evasão real com as regras de negócio estabelecidas no Manual do Programa \cite{HackersDoBemManual}. O \textit{pipeline} de análise é definido pelas seguintes etapas:

\begin{itemize}
    \item \textbf{Transição $T_1$ (Nivelamento $\rightarrow$ Básico):} Nesta etapa, a análise é puramente volumétrica, visto que não há critérios de corte eliminatórios, apenas a conclusão dos módulos. O foco será mensurar a taxa de desistência espontânea ($\delta_{espontanea}$) entre alunas que concluíram o Nivelamento mas não iniciaram o Básico.
    
    \item \textbf{Transição $T_2$ (Básico $\rightarrow$ Fundamental):} Este é o primeiro ponto de ranqueamento competitivo. A análise verificará se a pontuação de gamificação (XP), utilizada como critério de classificação nesta fase, apresenta correlação negativa com o gênero feminino, validando a hipótese de que mecânicas de competição podem desestimular este grupo \cite{Horcher2021}.
    
    \item \textbf{Transição $T_3$ (Fundamental $\rightarrow$ Especializado - Mecânica de Ondas):} O acesso ao nível Especializado é regido por um limite rígido de vagas distribuídas em "Ondas" (1ª e 2ª Onda). A estratégia aqui consiste em simular o algoritmo de desempate descrito na Seção 6.3 do Manual, que prioriza a "Nota Prática" sobre o "Engajamento (XP)" \cite{HackersDoBemManual}. O objetivo é identificar se mulheres são desproporcionalmente alocadas para a "Lista de Espera" ou "2ª Onda", o que pode aumentar a probabilidade de evasão por perda de ênfase (tempo de espera).
    
    \item \textbf{Transição $T_4$ (Especializado $\rightarrow$ Residência Tecnológica):} A etapa final utiliza um ranqueamento acumulativo por trilha de especialização (Blue Team, Red Team, GRC, etc.). A análise focará na segregação vertical, verificando se há concentração feminina em trilhas de Gestão (GRC) em detrimento de trilhas técnicas (Red/Blue Team), e se a taxa de conversão para a Residência varia conforme a trilha escolhida.
\end{itemize}

Para cada transição $T_i$, será calculado o índice de sobrevivência ($S_{gen}$) por gênero, permitindo isolar se a perda de participantes ocorre por falha técnica (nota insuficiente) ou por desistência sistêmica (abandono apesar de notas suficientes), correlacionando estes eventos com as datas de liberação das listas de classificação.

\section{Validação da Proposta}
\label{s_validacao}

A validação do \textit{Framework de Diretrizes} proposto não se dará pela implementação imediata de um novo curso, dada a inviabilidade temporal, mas sim através de uma validação analítica comparativa (\textit{Benchmarking}). O objetivo é demonstrar que as diretrizes propostas cobrem as lacunas quantitativas identificadas no programa, utilizando como referência os indicadores de sucesso reportados no estado da arte.

A estratégia de validação consiste em confrontar as métricas de evasão e engajamento extraídas do Programa \textit{Hackers do Bem} (Diagnóstico - QP2 e QP3) com os resultados de intervenções internacionais bem-sucedidas mapeadas na RSL (QP1).

\subsection{Definição dos Benchmarks de Referência}
\label{ss_benchmarks}

Para estabelecer um critério de "sucesso" ou "falha" nas taxas de retenção e engajamento do programa, foram definidos indicadores-chave baseados nos trabalhos correlatos selecionados. O \autoref{qua:benchmarks_validacao} apresenta a matriz de referência que será utilizada para validar a gravidade dos gargalos encontrados no \textit{Hackers do Bem}.

\begin{quadro}[htb]
\centering
\caption{Matriz de Benchmarks para Validação dos Indicadores do Programa}
\label{qua:benchmarks_validacao}
\begin{footnotesize}
\begin{tabular}{|l|p{5.5cm}|p{6cm}|}
    \hline
    \textbf{Dimensão} & \textbf{Indicador de Referência (Benchmark)} & \textbf{Fonte e Justificativa} \\
    \hline
    Escalabilidade & Capacidade de manter a qualidade e retenção com o aumento exponencial de inscritos (ex: 600 para 1.700 alunos). & \citeonline{Tshekiso2025}: Estabelece parâmetros para programas massivos em países em desenvolvimento. \\
    \hline
    Autoeficácia & Aumento mensurável na confiança técnica após intervenções práticas (\textit{hands-on}), independente da nota final. & \citeonline{Costa2025}: Valida que a percepção de competência é mais crítica que a competência bruta para a retenção feminina. \\
    \hline
    Empregabilidade & Taxa de conversão de alunos formados para o mercado de trabalho (transição Academia-Indústria). & \citeonline{Musuva2025}: Define o sucesso não apenas pela conclusão do curso, mas pela inserção profissional efetiva. \\
    \hline
    Métricas de Gênero & Existência de dados desagregados e monitoramento contínuo de \textit{gaps} específicos. & \citeonline{SelmanHousein2025}: Critica a ausência de métricas quantitativas na maioria das intervenções. \\
    \hline
\end{tabular}
\end{footnotesize}
\fonte{Elaborado pela autora.}
\end{quadro}

\subsection{Análise de Gap e Aderência}
\label{ss_analise_gap}

A validação final consistirá no cálculo do \textit{Gap de Desempenho} ($\Delta P$) entre o cenário atual do \textit{Hackers do Bem} e os benchmarks estabelecidos. Para cada dimensão $d$ listada no \autoref{qua:benchmarks_validacao}, será realizada uma análise qualitativa e quantitativa da discrepância:

\begin{equation}
    \Delta P_{d} = Métrica_{HdB} - Benchmark_{Literatura}
    \label{eq:gap_performance}
\end{equation}

Se $\Delta P_{d}$ apresentar um valor negativo significativo, a diretriz proposta para aquela dimensão será considerada validada se, e somente se, atacar a causa raiz identificada na análise do funil (Seção \ref{ss_analise_funil}). Por exemplo, se a evasão feminina no nível "Fundamental" for 30\% superior à do estudo de \citeonline{Tshekiso2025}, a diretriz de "Gamificação Colaborativa" será validada como uma intervenção necessária para mitigar este desvio específico.

Desta forma, assegura-se que o \textit{framework} não é composto por sugestões genéricas, mas por soluções cirúrgicas para os problemas métricos reais do programa, respondendo à QP4.

\section{Cronograma de Execução}
\label{ss_cronograma}

O planejamento temporal desta pesquisa foi elaborado considerando a natureza aplicada do trabalho e o prazo regimental para a defesa da dissertação, estipulado para ocorrer impreterivelmente até 23 de maio de 2026 \cite{EmailCoordenacao2026}.

A execução das atividades está condicionada à aprovação do projeto pelo Comitê de Ética em Pesquisa (CEP), cuja submissão está programada para fevereiro de 2026. Para assegurar a viabilidade do cronograma, as etapas foram organizadas de forma a permitir o paralelismo entre os trâmites burocráticos e a preparação técnica do ambiente de análise.

As macroatividades foram distribuídas conforme o detalhamento a seguir:

\begin{enumerate}
    \item \textbf{Aprovação Ética e Regulatória (Fev):} Submissão do protocolo de pesquisa na Plataforma Brasil e solicitação formal de acesso aos dados junto à RNP, em conformidade com as normas da Univali \cite{UnivaliCEP}.
    \item \textbf{Coleta e Pré-processamento (Fev/Mar):} Extração dos dados brutos das turmas finalizadas (Ondas 1 e 2) e envio dos formulários de pesquisa para as participantes, iniciando-se imediatamente após a aprovação do CEP (previsão para o fim de fevereiro).
    \item \textbf{Mineração e Análise (Mar/Abr):} Execução do \textit{pipeline} de análise de dados para cálculo das métricas de evasão ($\epsilon$) e correlação de gamificação, simultaneamente à tabulação dos dados qualitativos dos formulários.
    \item \textbf{Redação e Defesa (Abr/Mai):} Escrita dos capítulos de resultados e discussão, revisão final do documento e realização da defesa pública.
\end{enumerate}

O \autoref{qua:cronograma_detalhado} sintetiza a alocação temporal das atividades, evidenciando a concentração de esforços na análise de dados durante o mês de março para garantir a entrega da versão final em abril.

\begin{quadro}[htb]
\centering
\caption{Cronograma de atividades para conclusão da dissertação (2026)}
\label{qua:cronograma_detalhado}
\begin{footnotesize}
\begin{tabular}{|l|c|c|c|c|}
\hline
\textbf{Atividade / Etapa} & \textbf{Fev} & \textbf{Mar} & \textbf{Abr} & \textbf{Mai} \\ \hline
Submissão ao CEP e Ajustes da Qualificação & X & & & \\ \hline
Solicitação e Acesso aos Dados (RNP) & X & & & \\ \hline
Aplicação de Questionários e Coleta de Dados & X & X & & \\ \hline
Processamento dos Dados e Análise do Funil & & X & & \\ \hline
Redação dos Capítulos de Resultados (4 e 5) & & X & X & \\ \hline
Revisão Final e Entrega da Dissertação & & & X & \\ \hline
Defesa da Dissertação (Prazo: 23/05) & & & & X \\ \hline
\end{tabular}
\end{footnotesize}
\fonte{Elaborado pela autora com base no calendário acadêmico \cite{EmailCoordenacao2026}.}
\end{quadro}

\section{Considerações Finais}
\label{s_consideracoes}

Neste capítulo, foi detalhada a proposta metodológica para o diagnóstico e a mitigação da evasão feminina no Programa \textit{Hackers do Bem}. A abordagem mista adotada, combinando a mineração de dados educacionais com a análise qualitativa da percepção, foi desenhada para superar as limitações identificadas na revisão da literatura, especificamente a escassez de métricas quantificáveis sobre inclusão em larga escala \cite{SelmanHousein2025}.

A estratégia de pesquisa fundamenta-se na premissa de que a gamificação, quando não ajustada para a diversidade, pode atuar como uma barreira inadvertida. O \textit{pipeline} de análise de dados proposto (Seção \ref{ss_analise_funil}) e as métricas de eficiência de engajamento (Seção \ref{ss_metricas_variaveis}) fornecem as ferramentas analíticas necessárias para testar essa hipótese com rigor estatístico.

Ressalta-se que a viabilidade da execução está assegurada pelo cronograma estabelecido, que prevê o paralelismo entre os trâmites éticos (CEP) e a preparação do ambiente de análise. A utilização de instrumentos validados para a coleta de dados primários, alinhada às melhores práticas de pesquisa social em computação \cite{Denscombe2014}, garante a confiabilidade dos resultados qualitativos que complementarão os dados massivos da plataforma.

Desta forma, a proposta aqui apresentada não apenas busca responder às questões de pesquisa formuladas, mas também visa entregar um artefato prático — o Framework de Diretrizes — capaz de influenciar a evolução do \textit{design} instrucional de programas de capacitação cibernética em nível nacional.
