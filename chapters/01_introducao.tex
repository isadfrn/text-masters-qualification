\chapter{Introdução}
\label{c_introducao}

A segurança cibernética consolidou-se como um pilar estratégico para a soberania nacional e a estabilidade econômica global. No entanto, o setor enfrenta um déficit de profissionais qualificados, fenômeno globalmente conhecido como \textit{Cybersecurity Skills Gap}. Relatórios recentes indicam que a força de trabalho atual precisa crescer significativamente para atender à demanda de proteção de dados e infraestruturas \cite{ISC2_2024}.

No Brasil, como uma das respostas a este desafio, foi instituído o Programa \textit{Hackers do Bem}, uma iniciativa financiada pelo \ac{MCTI} e executada pela \ac{RNP}, em parceria com o Senai e a Softex. O programa visa capacitar mais de 30 mil profissionais através de um modelo de ensino massivo, gamificado e escalável \cite{HackersDoBemManual}.

Paralelamente à escassez geral, observa-se uma disparidade de gênero acentuada no setor. Mulheres permanecem sub-representadas, enfrentando barreiras culturais e estruturais que dificultam não apenas o ingresso, mas especialmente, a permanência e a progressão na carreira. A Estratégia Nacional de Cibersegurança e o recente Decreto nº 12.573/2025 reforçam a necessidade de fomentar linhas de pesquisa que estimulem a formação de recursos humanos e a diversidade no setor \cite{Decreto12573}.

Neste contexto, esta dissertação investiga a eficácia das metodologias de ensino e retenção aplicadas em larga escala, especificamente sob a perspectiva de gênero, buscando compreender como mecanismos de gamificação e ranqueamento impactam a trajetória das estudantes no programa.

\section{Problema de Pesquisa}
\label{s_problema_pesquisa}

Apesar dos esforços para aumentar o ingresso de mulheres em cursos de tecnologia, a literatura aponta para um fenômeno de "vazamento de talentos" (\textit{leaky pipeline}), onde as taxas de evasão femininas superam as masculinas à medida que o nível técnico e a competitividade aumentam.

No Programa \textit{Hackers} do Bem, a progressão entre os níveis (Nivelamento, Básico, Fundamental, Especializado e Residência) é estruturada por critérios de desempenho, baseados em pontos de experiência, \textit{quizzes} e competições práticas do tipo \ac{CTFs}, resultando em um ranqueamento \cite{HackersDoBemManual}.

O problema de pesquisa reside na hipótese de que o \textit{design} instrucional e os mecanismos de gamificação competitiva, se não ajustados para a diversidade, podem atuar inadvertidamente como fatores de exclusão. Diante disso, define-se a seguinte Questão de Pesquisa Principal:

\begin{quote}
\textbf{QP:} Como programas de capacitação em larga escala, especificamente o \textit{Hackers} do Bem, podem ser otimizados para mitigar o \textit{gap} de gênero e aumentar a retenção de talentos femininos na força de trabalho de cibersegurança?
\end{quote}

Esta questão desdobra-se nas seguintes questões específicas:
\begin{itemize}
    \item \textbf{QP1:} Quais são as principais barreiras de entrada e estratégias de permanência para mulheres identificadas no estado da arte?
    \item \textbf{QP2:} Qual é o perfil de participação e a taxa de evasão feminina ao longo das trilhas de formação do programa?
    \item \textbf{QP3:} De que maneira elementos como a gamificação e o ranqueamento influenciam o engajamento e o desempenho das participantes do gênero feminino?
    \item \textbf{QP4:} Que diretrizes estratégicas podem ser formuladas para adaptar o programa visando a equidade de gênero?
\end{itemize}

\subsection{Solução Proposta}
\label{ss_solucao_proposta}

A solução proposta nesta pesquisa é o desenvolvimento de um \textbf{Framework de Diretrizes de Inclusão e Retenção}, fundamentado em evidências empíricas. A construção deste artefato baseia-se em uma abordagem de métodos mistos, que combina:
\begin{enumerate}
    \item A mineração de dados educacionais dos participantes do \textit{Hackers} do Bem, correlacionando métricas de desempenho técnico e gamificação com as taxas de evasão por gênero.
    \item A análise qualitativa da percepção das alunas, para compreender os fatores subjetivos que motivam a desistência ou a permanência.
\end{enumerate}

O \textit{framework} visa fornecer recomendações práticas para ajustes no \textit{design} do programa, focando em neutralizar vieses nos critérios de avaliação e promover um ambiente de aprendizado que favoreça a autoeficácia feminina.

\subsection{Delimitação de Escopo}
\label{ss_delimitacao_escopo}

O escopo da pesquisa limita-se à análise dos dados das turmas do Programa \textit{Hackers} do Bem realizadas entre 2024 e 2025. A análise quantitativa é voltada para as transições do funil de formação (do Nivelamento à Residência Tecnológica). Serão considerados apenas dados anonimizados, respeitando a Lei Geral de Proteção de Dados (LGPD). Não faz parte do escopo a alteração do código-fonte da plataforma de ensino ou a intervenção direta nas turmas em andamento, caracterizando-se como um estudo observacional e propositivo.

\subsection{Justificativa}
\label{ss_justificativa}

A justificativa para este trabalho é tripla: social, acadêmica e estratégica.
\begin{itemize}
    \item \textbf{Social e Estratégica:} O Decreto nº 12.573, de 4 de agosto de 2025, estabelece como objetivo da Política Nacional de Cibersegurança o incentivo à criação de linhas de pesquisa e o estímulo à formação de recursos humanos \cite{Decreto12573}. Garantir que mulheres concluam essa formação é essencial para maximizar o retorno do investimento público realizado no programa.
    \item \textbf{Acadêmica:} Revisões sistemáticas recentes, como a de \citeonline{SelmanHousein2025}, apontam a escassez de estudos que apresentem "métricas de impacto quantificáveis" sobre intervenções de gênero. Esta dissertação contribui para preencher essa lacuna ao analisar dados reais de um programa massivo.
    \item \textbf{Técnica:} Estudos indicam que a diversidade em equipes de segurança aumenta a eficiência na resolução de problemas complexos \cite{Pacheco2024}. Mitigar a evasão feminina não é apenas uma questão de justiça social, mas um requisito para a eficácia da defesa cibernética nacional.
\end{itemize}

\section{Objetivos}
\label{s_objetivos}

\subsection{Objetivo Geral}
\label{ss_objetivo_geral}

Propor e validar um conjunto de diretrizes para o aumento da retenção feminina em programas de capacitação em cibersegurança, baseadas na análise de dados de engajamento e evasão do Programa \textit{Hackers} do Bem.

\subsection{Objetivos Específicos}
\label{ss_objetivos_especificos}

\begin{itemize}
    \item Mapear, através de \ac{RSL}, as estratégias de retenção e barreiras de gênero documentadas na literatura.
    \item Analisar quantitativamente o funil de aprovação do programa, identificando os pontos críticos de evasão feminina.
    \item Avaliar a correlação entre os mecanismos de gamificação (\ac{XP}, Ranking) e o desempenho das participantes.
    \item Elaborar o \textit{Framework de Diretrizes} e validá-lo analiticamente através da comparação com \textit{benchmarks} da literatura.
\end{itemize}

\section{Metodologia}
\label{s_metodologia}

\subsection{Metodologia da Pesquisa}
\label{ss_metodologia_pesquisa}

Esta pesquisa classifica-se como aplicada, de natureza exploratória e descritiva, utilizando uma abordagem de métodos mistos. A vertente quantitativa utiliza técnicas de \ac{EDM} para processar os \textit{logs} da plataforma de ensino. A vertente qualitativa utiliza levantamentos de percepção (\textit{surveys}) fundamentados em escalas de autoeficácia e motivação \cite{Costa2025}.

\subsection{Procedimentos Metodológicos}
\label{ss_procedimentos_metodologicos}

Os procedimentos foram estruturados nas seguintes etapas:
\begin{enumerate}
    \item \textbf{\acl{RSL}:} Executada conforme o protocolo \ac{PICO}, nas bases ACM, IEEE, Scopus e Google Scholar, para fundamentar as categorias de análise.
    \item \textbf{Coleta e Tratamento de Dados:} Obtenção dos registros do programa junto à \ac{RNP} e aplicação de questionários.
    \item \textbf{Análise de Dados:} Processamento estatístico para cálculo de taxas de evasão e correlação de variáveis.
    \item \textbf{Definição das Diretrizes:} Síntese dos achados em recomendações práticas para o programa.
\end{enumerate}

\section{Estrutura da Dissertação}
\label{s_estrutura_dissertacao}

O \autoref{c_fundamentacao_teorica} apresenta a fundamentação teórica bem como o Programa \textit{Hackers} do Bem e toda sua estrutura. O \autoref{c_trabalhos_relacionados} detalha o diagnóstico do estado da arte bem como os procedimentos metodológicos para se levantar os trabalhos utilizados. Por fim, o \autoref{c_proposta} apresenta a proposta que esse trabalho pretende desenvolver, incluindo o detalhamento do método de análise de dados e o cronograma.
