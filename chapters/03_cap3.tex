\chapter{Trabalhos Relacionados}
\label{c_trabalhos_relacionados}

Neste capítulo são apresentados os trabalhos relacionados que forneceram a base teórica para o desenvolvimento desta pesquisa. O objetivo foi identificar o estado da arte atual, sobre programas de capacitação em cibersegurança voltados para o público feminino, bem como as estratégias de retenção e engajamento utilizadas. A partir desta análise, busca-se evidenciar as lacunas que este trabalho pretende preencher ao investigar o cenário do programa Hackers do Bem. O restante do capítulo descreve a metodologia utilizada para a seleção das obras e discute os principais achados.

\section{Metodologia}
\label{s_metodologia}

Para a condução deste estudo, adotou-se uma abordagem metodológica baseada em uma \ac{RSL}. O processo de seleção e análise dos trabalhos seguiu as diretrizes do método \ac{PRISMA}, a fim de garantir a transparência, a qualidade e a replicabilidade da pesquisa bibliográfica.

As etapas estabelecidas para a revisão incluíram a definição do protocolo de busca, a identificação das fontes de informação, a aplicação de critérios de inclusão e exclusão em uma primeira etapa olhando apenas para o título dos trabalhos. Na segunda etapa, foi realizada a leitura do resumo dos artigos, onde foram observados a interseção e rigor metodológico desses trabalhos com as questões de pesquisa que este trabalho busca responder. O objetivo da revisão foi viabilizar a extração de dados qualitativos e quantitativos, bem como, sorver de literatura para a fundamentação teórica.

\subsection{Pesquisa bibliográfica}
\label{s_s_pesquisa_bibliografica}

Esta etapa consistiu no mapeamento do estado da arte sobre iniciativas de treinamento em cibersegurança e a inclusão de mulheres no setor tecnológico. Para a recuperação dos trabalhos, foram consultadas quatro bases de dados:

\begin{itemize}
    \item \hyperlink{https://ieeexplore.ieee.org/Xplore/home.jsp}{IEEE Xplore}
    \item \hyperlink{https://www.elsevier.com/pt-br/products/scopus/search}{Scopus} 
    \item \hyperlink{https://scholar.google.com/}{Google Scholar}
    \item \hyperlink{https://dl.acm.org/}{ACM}
\end{itemize}

A estratégia de busca foi construída com base no \ac{PICO}. Os termos foram definidos para abranger conceitos de gênero, segurança da informação e ações de capacitação, conforme descritos na \autoref{qua:estrategia_pico}:

\begin{quadro}[htb]
\centering
\caption{Estratégia PICO utilizada na revisão sistemática}
\label{qua:estrategia_pico}
\begin{footnotesize}
\begin{tabular}{|l|p{10cm}|}
    \hline
    \textbf{Elemento} & \textbf{Definição na Pesquisa} \\
    \hline
    \textbf{População} & Meninas, mulheres, estudantes e profissionais em cibersegurança. \\
    \hline
    \textbf{Intervenção} & Projetos, programas, ações de inclusão, capacitação, mentoria, oficinas, eventos e \textit{hackathons}. \\
    \hline
    \textbf{Comparação} & Projetos em cibersegurança (gerais/com recorte de gênero). \\
    \hline
    \textbf{Resultado (\textit{Outcome})} & Aumento do número de ações voltadas às mulheres na área de cibersegurança. \\
    \hline
\end{tabular}
\end{footnotesize}
\fonte{Elaborado pela autora.}
\end{quadro}

Considerando os três termos gerais usados na estratégia \ac{PICO}, foram construídas três conjuntos de palavras:

\begin{itemize}
    \item \textit{("women" OR "girls" OR "female" OR "gender")}
    \item \textit{("cybersecurity" OR "digital security" OR "information security" OR "cyber security")}
    \item \textit{("inclusion" OR "outreach" OR "mentor" OR "train" OR "empower" OR "project" OR "hackathon" OR "workshop" OR "boot camp")}
\end{itemize}

Resultando por fim em uma string de busca maior, formada pela união desses três grupos de palavras:

\textit{("women" OR "girls" OR "female" OR "gender") AND ("cybersecurity" OR "digital security" OR "information security" OR "cyber security") AND ("inclusion" OR "outreach" OR "mentor" OR "train" OR "empower" OR "project" OR "hackathon" OR "workshop" OR "boot camp")}

Para o refinamento dos resultados, além da string de busca, foram incluídos outros filtros nos motores de busca. O levantamento deveria retornar apenas livros, artigos completos publicados em periódicos ou eventos, entre o período de 2020 a 2025. Após a execução nas bases de dados, todos os metadados dos artigos foram exportados para a ferramenta \hyperlink{https://www.rayyan.ai/}{Rayyan}, que foi utilizada para decidir quais artigos deveriam ser incluídos na segunda etapa do refinamento, incluir critérios de aceito e exclusão, bem como busca de referências duplicadas e referências incompletas.

Na primeira etapa da revisão, 661 artigos foram encontrados, sendo 342 na IEEE Xplore, 50 no Google Scholar, 54 na Scopus e 215 na ACM.

\begin{table}[htb]
\centering
\ABNTEXfontereduzida
\IBGEtab{%
  \caption{Processo de seleção dos artigos}
  \label{tab:selecao_artigos}
}{%
  \begin{tabular}{l|c|c|c|c}
  \toprule
  \textbf{Base de dados} & \textbf{1ª seleção} & \textbf{2ª seleção} & \textbf{3ª seleção} & \textbf{Seleção final} \\
  \midrule
Google Scholar & 50  & 29 & 6  & 1 \\
Scopus         & 54  & 23 & 5  & 1 \\
IEEE Xplore    & 342 & 32 & 11 & 2 \\
ACM            & 215 & 30 & 10 & 4 \\
  \midrule
  \textbf{Total} & \textbf{651} & \textbf{114} & \textbf{32} & \textbf{8} \\
  \bottomrule
  \end{tabular}%
}{%
  \fonte{Elaborada pela autora}
}
\end{table}

\section{Comparação dos trabalhos selecionados}
\label{s_comparacao_trabalhos_selecionados}

Este capítulo apresenta a análise dos trabalhos selecionados através da \ac{RSL}, com o objetivo de identificar o estado da arte referente às estratégias de mitigação do gap de gênero em cibersegurança e o impacto de metodologias ativas, como a gamificação, em programas de capacitação em larga escala.

A seleção final resultou em 8 artigos que abordam diretamente as questões de pesquisa (QP) deste trabalho. A análise a seguir está estruturada para demonstrar como a literatura atual trata o tripé: Gênero, Gamificação e Escalabilidade, e como a presente dissertação se posiciona frente às lacunas identificadas.

\subsection{Panorama da Seleção Final}
\label{s_c3_panorama}

Para avaliar a atualidade e a relevância dos estudos selecionados, foi realizada uma análise temporal das publicações. A \autoref{fig:timeline} ilustra a distribuição dos trabalhos por ano de publicação. É possível observar uma concentração expressiva de trabalhos no ano de 2025, indicando que a discussão sobre a intersecção entre inclusão de gênero e metodologias de ensino em cibersegurança é um tema emergente e de alta relevância no cenário acadêmico atual.

\begin{figure}[htb]
    \centering
    \caption{Distribuição temporal dos trabalhos selecionados}
    \label{fig:timeline}
    \includegraphics[width=0.8\textwidth]{figs/distribuicao-temporal.png}
    \fonte{Elaborada pela autora.}
\end{figure}

Diferente de revisões que abrangem décadas, a seleção final demonstra que a discussão sobre escala e intervenções específicas de gênero atingiu seu ápice de publicação recentemente. Isso valida a justificativa desta pesquisa, pois o programa Hackers do Bem opera neste contexto temporal e tecnológico.

% ----------------------------------------------------------------------- %
\subsection{Análise Multidimensional dos Trabalhos}
\label{s_c3_analise_multi}

A fim de compreender as contribuições individuais e as limitações de cada estudo em relação ao objeto desta dissertação, os trabalhos foram avaliados sob cinco dimensões críticas, derivadas das questões de pesquisa:

\begin{enumerate}
    \item \textbf{Foco em Gênero (QP1):} Profundidade da análise sobre barreiras para mulheres.
    \item \textbf{Gamificação/CTF (QP3):} Abordagem de jogos e competições (Capture The Flag).
    \item \textbf{Escalabilidade (QP2):} Aplicabilidade em programas de massa.
    \item \textbf{Empirismo:} Presença de dados reais versus revisões teóricas.
    \item \textbf{Soft Skills (QP4):} Inclusão de competências não-técnicas como estratégia de retenção.
\end{enumerate}

A \autoref{fig:radar} apresenta o gráfico de radar resultante desta avaliação, permitindo a visualização dos \textit{clusters} temáticos e das lacunas existentes na literatura.

\begin{figure}[htb]
    \centering
    \caption{Comparativo multidimensional dos trabalhos selecionados}
    \label{fig:radar}
    \includegraphics[width=0.8\textwidth]{figs/analise-multidimensional.png}
    \fonte{Elaborada pela autora}
\end{figure}

Observando a \autoref{fig:radar}, nota-se a formação de três grupos distintos de trabalhos, detalhados nas seções a seguir.

\subsection{Gamificação e Adaptação de CTFs}
Enquanto o programa \textit{Hackers do Bem} utiliza gamificação massiva através de sistemas de XP e ranqueamento, a literatura aponta ressalvas críticas sobre a aplicação genérica destes métodos. \citeonline{Coenraad2020} fornecem a base teórica de que jogos digitais são eficazes para o engajamento geral, mas \citeonline{Horcher2021} introduz um contraponto essencial para a \textbf{QP3}: competições tradicionais de CTF podem alienar grupos sub-representados se não forem adaptadas.

Corroborando esta visão, \citeonline{Costa2025} demonstram empiricamente, através de intervenções práticas, que a adaptação da narrativa do jogo e o ambiente colaborativo aumentam significativamente a autoeficácia feminina. Este resultado sugere que a gamificação, sem um design inclusivo, pode atuar como barreira e não como incentivo, hipótese que será verificada nos dados de evasão do \textit{Hackers do Bem}.

\subsection{Escala e Programas de Capacitação}
A escalabilidade é o desafio central abordado na QP2 desta pesquisa. Os trabalhos de \citeonline{Musuva2025} e \citeonline{Tshekiso2025} destacam-se no gráfico de radar (eixo verde) por atingirem a pontuação máxima em escalabilidade. Ambos descrevem iniciativas na África, como o programa \textit{Cyber Shujaa}, que enfrentam barreiras similares de infraestrutura e acesso às encontradas no Brasil.

A análise destes trabalhos oferece um \textit{benchmark} importante: ambos sugerem que a mentoria híbrida e o suporte comunitário são vitais para evitar a evasão em massa em programas de grande porte. No entanto, estes estudos possuem menor foco na análise específica das mecânicas de gamificação (eixo azul do gráfico), lacuna que esta dissertação pretende preencher ao analisar os dados do \textit{Hackers do Bem}.

\subsection{Gênero, Soft Skills e Barreiras de Entrada}
Para a formulação das diretrizes (\textbf{QP4}), \citeonline{Benson2025} e \citeonline{Pacheco2024} oferecem a fundamentação teórica necessária. \citeonline{Benson2025} argumenta que a valorização excessiva de \textit{hard skills} em detrimento de \textit{soft skills} cria uma barreira de entrada artificial para mulheres.

Cruzando esta informação com a Revisão Sistemática realizada por \citeonline{SelmanHousein2025}, observa-se que programas de sucesso integram o desenvolvimento de competências comportamentais como parte do currículo técnico. A \autoref{fig:radar} evidencia que estes trabalhos (eixo vermelho/laranja) são fortes na discussão de gênero, mas carecem de validação em ambientes de larga escala, limitando-se muitas vezes a estudos de caso em universidades ou revisões teóricas.

\section{Comparação e Lacunas Identificadas}
\label{s_c3_comparacao}

A \autoref{tab:comparacao_trabalhos} sintetiza a comparação entre os trabalhos selecionados e a proposta desta dissertação. É possível verificar que, embora existam estudos sobre gamificação inclusiva e estudos sobre programas de larga escala, há uma ausência de trabalhos que integrem ambas as dimensões sob a ótica de gênero.

\begin{quadro}[!htbp]
\centering
\caption{Comparação dos trabalhos relacionados com a proposta atual}
\label{tab:comparacao_trabalhos}
\begin{footnotesize}
\resizebox{\textwidth}{!}{%
\begin{tabular}{|c|c|c|c|c|c|}
\hline
\rowcolor[HTML]{EFEFEF} 
\textbf{Trabalho} & \textbf{Foco Principal} & \textbf{Escala} & \textbf{Gamificação} & \textbf{Gênero} & \textbf{Limitação Principal} \\ \hline
\citeonline{Coenraad2020} & Jogos em Cyber & Baixa & Alta & Baixa & Foco técnico, pouca análise de inclusão. \\ \hline
\citeonline{Horcher2021} & CTF Inclusivo & Baixa & Alta & Alta & Estudo em ambiente controlado/pequeno. \\ \hline
\citeonline{SelmanHousein2025} & RSL de Intervenções & Baixa & Baixa & Alta & Foco em graduação, revisão teórica sem dados primários. \\ \hline
\citeonline{Pacheco2024} & Desafios Culturais & Baixa & Baixa & Alta & Propositivo/Teórico, carece de validação empírica. \\ \hline
\citeonline{Musuva2025} & Capacitação (África) & Alta & Baixa & Média & Foco em empregabilidade geral, não em gênero. \\ \hline
\citeonline{Tshekiso2025} & Escala em Educação & Alta & Média & Baixa & Foco em infraestrutura/acesso, gênero é secundário. \\ \hline
\citeonline{Costa2025} & Gap de Gênero & Média & Alta & Alta & Intervenção específica (workshop), não longitudinal. \\ \hline
\citeonline{Benson2025} & Soft Skills & Baixa & Baixa & Alta & Estudo qualitativo/teórico sobre perfil profissional. \\ \hline
\rowcolor[HTML]{C0C0C0} 
\textbf{Esta Dissertação} & \textbf{Diagnóstico e Diretrizes} & \textbf{Alta} & \textbf{Alta} & \textbf{Alta} & \textbf{Foco específico no Programa Hackers do Bem.} \\ \hline
\end{tabular}%
}
\end{footnotesize}
\fonte{Elaborado pela autora.}
\end{quadro}

\section{Trabalhos Selecionados}
\label{s_trabalhos_selecionados}

As próximas subseções descrevem as características principais, metodologias e as contribuições específicas de cada um dos trabalhos selecionados para o contexto do Programa \textit{Hackers} do Bem.

\subsection{Improving the Representation of Undergraduate Women in Cybersecurity: A Literature Review \cite{SelmanHousein2025}}
\label{ss_selmanhousein2025}

Este trabalho, publicado nos anais da ACM SIGCSE 2025, consiste em uma Revisão Sistemática da Literatura (RSL) focada em mapear e avaliar intervenções para aumentar a representatividade feminina em cursos de graduação em cibersegurança. A metodologia adotada pelos autores partiu de uma busca abrangente que retornou inicialmente mais de 41.000 resultados, os quais foram submetidos a um processo rigoroso de filtragem em duas etapas (busca ampla e busca focada), resultando em um \textit{corpus} final de 50 artigos únicos analisados.

A análise categorizou as intervenções identificadas na literatura em três eixos estruturantes: (i) recrutamento; (ii) suporte envolvente (\textit{wrap-around support}) e oportunidades extraclasse; e (iii) design curricular. Os resultados indicam que a maioria dos estudos se concentra em suporte extraclasse, como mentorias e clubes, em detrimento de mudanças estruturais no currículo.

Uma contribuição crítica deste estudo para a presente dissertação é a identificação de uma lacuna metodológica prevalente no estado da arte: a escassez de estudos que apresentem "métricas de impacto quantificáveis". \citeonline{SelmanHousein2025} argumentam que, para desenvolver intervenções sustentáveis, é imperativo documentar as melhores práticas com dados empíricos que validem não apenas o acesso, mas a eficácia das ações na promoção do senso de pertencimento e na retenção das estudantes.

\subsection{Gender Diversity in Cybersecurity: Gaps, Challenges, and Proposals \cite{Pacheco2024}}
\label{ss_pacheco2024}

O trabalho de \citeonline{Pacheco2024}, apresentado no Congresso Bienal do IEEE na Argentina (ARGENCON), investiga a disparidade de gênero no setor de cibersegurança sob a premissa fundamental de que "toda ação transformadora baseia-se em questões educacionais". O estudo realiza uma análise qualitativa da representatividade feminina, categorizando as barreiras identificadas em três dimensões estruturantes: culturais, educacionais e profissionais.

O autor avalia o impacto da diversidade na eficácia das equipes de trabalho (\textit{work teams}), argumentando que a inclusão não é apenas uma questão de equidade, mas um requisito para a melhoria da defesa cibernética. Como resultado, são propostas diretrizes e estratégias práticas direcionadas a três atores principais: governos, organizações privadas e instituições de ensino.

Uma contribuição relevante deste estudo para a presente dissertação reside na identificação de uma lacuna geográfica na literatura. \citeonline{Pacheco2024} destaca a escassez de pesquisas sobre gênero em cibersegurança no contexto latino-americano, enfatizando a necessidade de investigações locais que considerem as especificidades culturais da região. Este apontamento valida a importância de analisar programas nacionais, como o \textit{Hackers do Bem}, para compor um corpo de conhecimento regionalizado.

\subsection{Why Soft Skills Matter for Women in Cybersecurity \cite{Benson2025}}
\label{ss_benson2025}

O estudo de \citeonline{Benson2025}, publicado no \textit{Journal of Computer Information Systems}, aborda a desconexão crítica entre as competências valorizadas nos programas de formação tradicionais e as reais necessidades da força de trabalho moderna. As autoras investigam como a ênfase excessiva em \textit{hard skills} (conhecimentos técnicos operacionais) atua como um mecanismo de exclusão para mulheres, que frequentemente possuem vantagens competitivas em \textit{soft skills} como comunicação, gestão de crises e colaboração.

A pesquisa destaca que, embora a indústria de cibersegurança demande cada vez mais profissionais capazes de traduzir riscos técnicos para a linguagem de negócios, os currículos acadêmicos e os treinamentos corporativos falham em formalizar essas competências em seus critérios de avaliação. O trabalho critica a ausência de \textit{soft skills} em frameworks de competências padronizados, argumentando que essa lacuna perpetua a percepção da área como um domínio estritamente técnico e masculino.

Para a presente dissertação, a contribuição deste trabalho é fundamental para questionar a métrica de "sucesso" adotada pelo Programa \textit{Hackers do Bem}. Se o sistema de ranqueamento do programa prioriza exclusivamente a pontuação em laboratórios técnicos (CTF) e ignora a colaboração ou o engajamento na comunidade (elementos ligados a \textit{soft skills}), ele pode estar sistematicamente subavaliando o potencial de candidatas qualificadas para a Residência Tecnológica.

\subsection{Changing the Game: Adapting Capture the Flag To Underrepresented Groups \cite{Horcher2021}}
\label{ss_horcher2021}

O estudo conduzido por \citeonline{Horcher2021}, apresentado na conferência IEEE RESPECT, investiga as barreiras estruturais e culturais intrínsecas às competições de cibersegurança tradicionais, especificamente os eventos do tipo \textit{Capture The Flag} (CTF). A autora analisa como o \textit{design} padrão destas competições — frequentemente caracterizado por ambientes hipercompetitivos, restrições de tempo agressivas e narrativas de combate (ataque-defesa) — reforça vieses que alienam grupos sub-representados, com ênfase no público feminino.

A pesquisa argumenta que a mecânica de jogo tradicional dos CTFs valoriza desproporcionalmente a velocidade e a assunção de riscos em detrimento da análise metódica e da colaboração. \citeonline{Horcher2021} propõe adaptações no desenho destas atividades para transformar a experiência de um filtro excludente para uma ferramenta de engajamento, sugerindo a incorporação de narrativas inclusivas e objetivos que premiem a persistência e o aprendizado coletivo, em vez de apenas a pontuação bruta.

Para a presente dissertação, a contribuição deste trabalho é fundamental para a análise do Programa \textit{Hackers do Bem}. Considerando que o programa utiliza o desempenho em desafios práticos e o ranqueamento por XP como critérios eliminatórios para o acesso à Residência Tecnológica, a crítica de Horcher fornece a base teórica para investigar se este modelo de avaliação está medindo a competência técnica real das candidatas ou se está, inadvertidamente, replicando os vieses culturais que historicamente afastam mulheres da área de segurança.

\subsection{Experiencing Cybersecurity One Game at a Time: A Systematic Review of Cybersecurity Digital Games \cite{Coenraad2020}}
\label{ss_coenraad2020}

O estudo conduzido por \citeonline{Coenraad2020}, publicado no periódico \textit{Simulation \& Gaming}, estabelece um panorama abrangente sobre o uso de jogos digitais no ensino de segurança. Os autores realizaram uma Revisão Sistemática analisando um \textit{corpus} robusto de 181 jogos desenvolvidos nas últimas décadas, classificando-os quanto ao público-alvo, mecânicas de jogo e objetivos pedagógicos.

A análise taxonômica proposta pelos autores distingue duas categorias predominantes: jogos de \textit{Cyber Safety} (focados em conscientização e comportamento seguro para o público geral) e jogos de \textit{Cyber Security} (focados em competências técnicas para profissionais). Um dos achados críticos do estudo é a prevalência de mecânicas superficiais: a maioria das iniciativas analisadas limita-se a "quizzes gamificados" que utilizam sistemas de pontos e emblemas (PBL) para recompensar a memorização de conceitos, falhando em promover o "engajamento epistêmico" — onde o jogador assume a identidade e as práticas reais de um profissional da área.

Para a presente dissertação, a contribuição de \citeonline{Coenraad2020} é estrutural para a análise da plataforma do Programa \textit{Hackers do Bem}. O referencial teórico fornece os critérios para avaliar se as atividades de gamificação do programa (Nivelamento e Básico) operam apenas no nível de \textit{Cyber Safety} (quizzes e vídeos) ou se oferecem a simulação técnica necessária (\textit{Cyber Security}) para a retenção de talentos e desenvolvimento de autoeficácia, especialmente para grupos sub-representados que carecem de modelos de referência prévios.

\subsection{Tackling the Gender Gap in Cybersecurity Education \cite{Costa2025}}
\label{ss_costa2025}

O trabalho de \citeonline{Costa2025}, publicado nos anais da \textit{ACM Technical Symposium on Computer Science Education} (SIGCSE 2025), apresenta os resultados de uma intervenção prática voltada para estudantes do ensino médio, com o objetivo de mitigar o \textit{gap} de gênero antes da escolha universitária. Os autores partem da premissa de que a baixa representatividade feminina não decorre de falta de aptidão técnica, mas da ausência de exposição precoce e da prevalência de estereótipos dissuasores.

A metodologia adotada pelo estudo envolveu a aplicação de treinamentos práticos baseados em \textit{Capture The Flag} (CTF), desenhados especificamente para desmistificar a complexidade da área. Os resultados indicam que a introdução controlada a cenários de ataque e defesa aumenta significativamente a autoeficácia das participantes, combatendo a percepção de que a cibersegurança é um domínio exclusivamente masculino ou que exige habilidades inatas inalcançáveis.

Para a presente dissertação, a contribuição de \citeonline{Costa2025} é fundamental para analisar o módulo de "Nivelamento" do Programa \textit{Hackers do Bem}. O estudo sugere que a forma como o conteúdo técnico inicial é apresentado — se de maneira colaborativa ou puramente competitiva — é determinante para a retenção. A validação empírica trazida pelos autores reforça a hipótese de que o \textit{design} instrucional do programa deve priorizar o engajamento prático ("mão na massa") como ferramenta de inclusão, e não apenas como filtro de seleção.

\subsection{Addressing the Cybersecurity Workforce Gap: Lessons from the Cyber Shujaa Program in Kenya \cite{Musuva2025}}
\label{ss_musuva2025}

O estudo de \citeonline{Musuva2025}, apresentado na conferência IST-Africa 2025, analisa o programa \textit{Cyber Shujaa} no Quênia, uma iniciativa desenhada para mitigar simultaneamente dois problemas críticos: o déficit global de força de trabalho em cibersegurança e as altas taxas de desemprego juvenil na região. A pesquisa documenta a implementação de um modelo de "ponte entre academia e indústria" (\textit{Academia-Industry Bridging}), focado em converter o conhecimento teórico de graduados em competências práticas de mercado.

A metodologia do programa baseia-se em um currículo orientado por competências, combinando treinamento técnico intensivo, mentoria de carreira e suporte direto para colocação profissional (\textit{job placement}). Os autores argumentam que, em contextos de economias em desenvolvimento, a capacitação técnica isolada é insuficiente; é necessário um ecossistema de suporte que inclua o desenvolvimento de habilidades interpessoais e o alinhamento direto com as demandas do setor privado para garantir a empregabilidade sustentável.

Para a presente dissertação, a contribuição de \citeonline{Musuva2025} reside na validação de modelos de capacitação em larga escala em contextos socioeconômicos similares ao do Brasil. Embora o foco primário do estudo seja a empregabilidade geral e não exclusivamente o recorte de gênero, as lições aprendidas sobre a transição da formação para o mercado de trabalho (a fase de "Residência" no contexto do \textit{Hackers do Bem}) fornecem um comparativo essencial para avaliar se as estruturas de suporte oferecidas pelo programa brasileiro são suficientes para reter talentos femininos na etapa crítica de profissionalização.

\subsection{Scaling Cybersecurity Education in Africa: Insights from a Capacity-Building Initiative \cite{Tshekiso2025}}
\label{ss_tshekiso2025}

O trabalho de \citeonline{Tshekiso2025}, apresentado na \textit{ACM Global Computing Education Conference} (CompEd 2025), documenta a implementação e a escalabilidade da iniciativa \textit{picoCTF-Africa}, liderada pela \textit{Carnegie Mellon University Africa}. O estudo é classificado como um relato de experiência que analisa a expansão de um programa de educação em cibersegurança baseado em competições do tipo \textit{Capture The Flag} (CTF), visando mitigar a escassez de talentos e a falta de exposição precoce à área no continente africano.

Em termos de métricas de impacto, os autores relatam um crescimento expressivo na participação, evoluindo de aproximadamente 600 estudantes em 2022 para mais de 1.700 em 2025, abrangendo 44 países (81\% das nações africanas). A metodologia adotada para sustentar essa escala baseou-se em uma abordagem "multi-stakeholder" e em um sistema hierárquico de capítulos (nacional e escolar). Um diferencial crítico identificado no estudo foi a integração de "estratégias de engajamento sensíveis ao gênero" e a criação de categorias específicas de competição focadas em gênero, desenhadas para reduzir barreiras de entrada culturais e estruturais.

Para a presente dissertação, a relevância de \citeonline{Tshekiso2025} reside na validação de estratégias para programas massivos em regiões com desafios de infraestrutura e diversidade similares aos do Brasil. A análise demonstra que a simples disponibilização de uma plataforma de CTF é insuficiente; a retenção de grupos sub-representados em larga escala exige a adaptação contextual do conteúdo e a criação de sub-redes de suporte (capítulos locais), diretrizes que podem ser contrastadas com a estrutura centralizada do Programa \textit{Hackers do Bem}.


\section{Considerações Finais}
\label{s_consideracoes_finais}

A análise dos trabalhos relacionados permite concluir que a decomposição do problema em fatores isolados (apenas gamificação ou apenas políticas de gênero) é insuficiente para abordar a complexidade da evasão feminina em programas massivos. O \textit{Hackers do Bem}, ao combinar ensino em larga escala com um sistema agressivo de ranqueamento gamificado, apresenta um cenário único que não foi completamente explorado pela literatura vigente.

Desta forma, esta pesquisa se diferencia ao correlacionar dados reais de participação e desempenho em larga escala com as mecânicas de jogo, visando propor diretrizes que unam a escalabilidade técnica (\citeonline{Tshekiso2025}) com a sensibilidade de inclusão (\citeonline{Horcher2021}, \citeonline{Benson2025}).
