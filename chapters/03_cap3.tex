\chapter{Trabalhos Relacionados}
\label{c_trabalhos_relacionados}

Neste capítulo, são apresentados os trabalhos relacionados que forneceram a base teórica para o desenvolvimento desta pesquisa. O objetivo foi identificar o estado da arte atual sobre programas de capacitação em cibersegurança com participação do público feminino, bem como as estratégias empregadas para ampliar o ingresso e a permanência de talentos femininos. A partir desta análise, busca-se evidenciar as lacunas que este trabalho pretende preencher ao investigar o cenário do programa \textit{Hackers} do Bem. O restante do capítulo descreve a metodologia utilizada para a seleção das obras e discute os principais achados.

\section{Metodologia}
\label{s_metodologia}

Para a condução deste estudo, adotou-se uma abordagem metodológica baseada em uma \ac{RSL}. O processo de seleção e análise dos trabalhos seguiu as diretrizes do método \ac{PRISMA}, a fim de garantir a transparência, a qualidade e a replicabilidade da pesquisa bibliográfica

As etapas estabelecidas para a revisão incluíram a definição do protocolo de busca, a identificação das fontes de informação e a aplicação de critérios de inclusão e exclusão. O objetivo da revisão foi viabilizar a extração de dados qualitativos e quantitativos, bem como obter subsídios na literatura para a fundamentação teórica.

\subsection{Pesquisa bibliográfica}
\label{s_s_pesquisa_bibliografica}

Esta etapa consistiu no mapeamento do estado da arte sobre iniciativas de treinamento em cibersegurança com foco na participação de mulheres. Para a recuperação dos trabalhos, foram consultadas quatro bases de dados:

\begin{itemize}
    \item \hyperlink{https://ieeexplore.ieee.org/Xplore/home.jsp}{IEEE Xplore}
    \item \hyperlink{https://www.elsevier.com/pt-br/products/scopus/search}{Scopus} 
    \item \hyperlink{https://scholar.google.com/}{Google Scholar}
    \item \hyperlink{https://dl.acm.org/}{ACM}
\end{itemize}

A estratégia de busca foi construída com base no \ac{PICO}. Os termos foram definidos para abranger conceitos de gênero, ações de capacitação em cibersegurança e estratégias voltadas à ampliação do ingresso e da permanência de talentos femininos, conforme descrito no quadro a seguir:

\begin{quadro}[htb]
\centering
\caption{Estratégia PICO utilizada na revisão sistemática}
\label{qua:estrategia_pico}
\begin{footnotesize}
\begin{tabular}{|l|p{10cm}|}
    \hline
    \textbf{Elemento} & \textbf{Definição na Pesquisa} \\
    \hline
    \textbf{População} & Meninas, mulheres, estudantes e profissionais em cibersegurança. \\
    \hline
    \textbf{Intervenção} & Projetos, programas, ações de inclusão, capacitação, mentoria, oficinas, eventos e \textit{hackathons}. \\
    \hline
    \textbf{Comparação} & Projetos em cibersegurança (gerais/com recorte de gênero). \\
    \hline
    \textbf{Resultado (\textit{Outcome})} & Aumento do número de ações voltadas às mulheres na área de cibersegurança. \\
    \hline
\end{tabular}
\end{footnotesize}
\fonte{Elaborado pela autora}
\end{quadro}

Considerando os três termos usados na estratégia \ac{PICO}, foram construídas três conjuntos de palavras:

\begin{itemize}
    \item \textit{(\enquote{women} OR \enquote{girls} OR \enquote{female} OR \enquote{gender})}
    \item \textit{(\enquote{cybersecurity} OR \enquote{digital security} OR \enquote{information security} OR \enquote{cyber security})}
    \item \textit{(\enquote{inclusion} OR \enquote{outreach} OR \enquote{mentor} OR \enquote{train} OR \enquote{empower} OR \enquote{project} OR \enquote{hackathon} OR \enquote{workshop} OR \enquote{bootcamp})}
\end{itemize}

Resultando por fim em uma \textit{string} de busca maior, formada pela união desses três grupos:

\textit{(\enquote{women} OR \enquote{girls} OR \enquote{female} OR \enquote{gender}) AND (\enquote{cybersecurity} OR \enquote{digital security} OR \enquote{information security} OR \enquote{cyber security}) AND (\enquote{inclusion} OR \enquote{outreach} OR \enquote{mentor} OR \enquote{train} OR \enquote{empower} OR \enquote{project} OR \enquote{hackathon} OR \enquote{workshop} OR \enquote{bootcamp})}

Para o refinamento dos resultados, além da \textit{string} de busca, foram incluídos outros filtros nos motores de busca. O levantamento deveria retornar apenas livros, artigos completos publicados em periódicos ou eventos, entre o período de 2020 a 2025. Após a execução nas bases de dados, todos os metadados dos artigos foram exportados para a ferramenta Rayyan\footnote{\url{https://www.rayyan.ai/}}, que foi utilizada para gerenciar quais artigos deveriam ser incluídos nas etapas subsequentes da \ac{RSL}, bem como incluir critérios de aceito e exclusão, eliminação de referências duplicadas e incompletas.

Na etapa de Identificação, a execução inicial da string de busca retornou 661 registros, distribuídos entre IEEE Xplore (342), Google Scholar (50), Scopus (54) e ACM (215). Após a importação para a ferramenta Rayyan e a remoção de 16 duplicatas, os trabalhos foram submetidos a três fases de seleção. A primeira fase consistiu na Triagem por Título, na qual cada trabalho foi avaliado quanto à sua disponibilidade integral e à aderência temática inicial, buscando-se responder às seguintes perguntas:

\begin{itemize}
    \item O artigo completo está disponível?
    \item O artigo aborda treinamento em cibersegurança ou a inclusão de mulheres no setor?
\end{itemize}

Esta fase resultou na seleção de 114 artigos. Na sequência, procedeu-se à Triagem por Resumo (segunda fase), com o objetivo de verificar a conexão direta entre o ensino de cibersegurança e estratégias de inclusão ou análise de gênero. Nesta etapa, descartaram-se estudos puramente técnicos, focados em gestão sem viés educacional ou que não apresentassem dados com recorte de gênero, restando 32 artigos elegíveis. Utilizou-se como critério de inclusão a seguinte questão:

\begin{itemize}
    \item O artigo relata alguma estratégia utilizada para ampliar o ingresso e a permanência de talentos femininos em programas de treinamento em cibersegurança?
\end{itemize}

Por fim, a fase de Elegibilidade compreendeu a leitura na íntegra (Full Text Screening) dos 32 textos selecionados. Nesta fase final, foram excluídos estudos sem dados empíricos ou relatos de experiência sem validação. Ao concluir o processo, o portfólio bibliográfico final foi composto por 9 artigos, que fundamentam as análises desta revisão. A \autoref{tab:selecao_artigos} detalha a redução do volume de trabalhos em cada base de dados ao longo do funil de seleção.

\begin{table}[htb]
\centering
\ABNTEXfontereduzida
\IBGEtab{%
  \caption{Processo de seleção dos artigos}
  \label{tab:selecao_artigos}
}{%
  \begin{tabular}{l|c|c|c|c}
  \toprule
  \textbf{Base de dados} & \textbf{Identificação} & \textbf{Primeira fase} & \textbf{Segunda fase} & \textbf{Seleção final} \\
  \midrule
Google Scholar & 50  & 29 & 6  & 1 \\
Scopus         & 54  & 23 & 5  & 1 \\
IEEE Xplore    & 342 & 32 & 11 & 1 \\
ACM            & 215 & 30 & 10 & 6 \\
  \midrule
  \textbf{Total} & \textbf{661} & \textbf{114} & \textbf{32} & \textbf{9} \\
  \bottomrule
  \end{tabular}%
}{%
  \fonte{Elaborada pela autora}
}
\end{table}

A predominância de artigos selecionados provenientes da base ACM (cerca de 67\% da seleção final) reflete a concentração de publicações sobre educação em computação (\textit{Computing Education}) nesta comunidade. Destacam-se conferências como SIGCSE TS, ITiCSE e Koli Calling, além de outras conferências de segurança como a ARES, demonstrando que a discussão sobre inclusão de gênero em cibersegurança está fortemente inserida nos fóruns de educação em computação.

\subsection{Panorama da Seleção Final}
\label{ss_panorama_selecao_final}

Para avaliar a atualidade e a relevância dos estudos selecionados, foi realizada uma análise temporal das publicações. A \autoref{fig:timeline} ilustra a distribuição dos trabalhos por ano de publicação. É possível observar uma concentração expressiva de trabalhos no ano de 2025, indicando que a discussão sobre a intersecção entre inclusão de gênero e metodologias de ensino em cibersegurança é um tema emergente e de alta relevância no cenário acadêmico atual.

\begin{figure}[htb]
    \centering
    \caption{Distribuição temporal dos trabalhos selecionados}
    \label{fig:timeline}
    \includegraphics[width=0.8\textwidth]{figs/distribuicao-temporal.png}
    \fonte{Elaborada pela autora}
\end{figure}

A seleção final demonstra que a discussão sobre escala e intervenções específicas de gênero teve um salto de publicações recentemente. Isso ajuda a validar a justificativa desta pesquisa, pois o programa \textit{Hackers} do Bem opera neste contexto temporal.

\subsection{Análise Multidimensional dos Trabalhos}
\label{ss_analise_multidimensional_trabalhos}

Com o objetivo de compreender as contribuições individuais e as limitações de cada estudo em relação ao objeto desta dissertação, especialmente ao considerar a estrutura de programas massivos como o \textit{Hackers} do Bem, os trabalhos foram avaliados sob cinco dimensões críticas derivadas diretamente das questões de pesquisa:

\begin{enumerate}
    \item \textbf{Foco em Gênero e Barreiras (QP1):} Profundidade da análise sobre os fatores socioculturais e estruturais determinantes para a baixa adesão feminina, bem como as intervenções específicas adotadas para mitigar a evasão. \item \textbf{Abordagens Pedagógicas (QP2):} Avaliação das estratégias de ensino utilizadas, tais como gamificação, laboratórios práticos e aprendizado ativo, com foco no impacto dessas abordagens na adesão e permanência do público feminino.
    \item \textbf{Colaboração e Autoeficácia (QP3):} Análise da implementação de mecanismos colaborativos, mentoria e customização de trilhas que influenciam a percepção de autoeficácia e a intenção de permanência das estudantes.
    \item \textbf{Escalabilidade e Alcance:} Viabilidade de aplicação das estratégias em programas de capacitação em larga escala e em ambientes distribuídos ou online, alinhando-se à característica massiva do \textit{Hackers} do Bem.
    \item \textbf{Validação Empírica:} Presença de dados reais e rigor metodológico na validação das intervenções propostas, distinguindo propostas teóricas de métodos efetivamente testados.
\end{enumerate}

Para consolidar a análise comparativa, os nove trabalhos selecionados foram avaliados qualitativamente e receberam atribuição de notas em uma escala de 1 (insuficiente ou não aborda) a 5 (excelente ou foco central) nas cinco dimensões estabelecidas. Esta pontuação reflete a aderência de cada estudo às questões de pesquisa (QP1, QP2 e QP3), bem como sua robustez metodológica e capacidade de escala. O \autoref{quadro:notas_radar} detalha a pontuação atribuída a cada trabalho:

\clearpage

\begin{quadro}[htb]
\centering
\caption{Avaliação multidimensional dos trabalhos selecionados (Escala 1-5)}
\label{quadro:notas_radar}
\ABNTEXfontereduzida
\begin{tabular}{|l|c|c|c|c|c|}
\hline
\textbf{Trabalho} & \textbf{\shortstack{Gênero\\(QP1)}} & \textbf{\shortstack{Pedagogia\\(QP2)}} & \textbf{\shortstack{Colab.\\(QP3)}} & \textbf{\shortstack{Escala-\\bilidade}} & \textbf{\shortstack{Vali-\\dação}} \\ \hline
\citeonline{Costa2025} & 5 & 5 & 3 & 4 & 5 \\ \hline
\citeonline{Musuva2025} & 4 & 3 & 5 & 3 & 4 \\ \hline
\citeonline{Tshekiso2025} & 3 & 2 & 3 & 5 & 4 \\ \hline
\citeonline{Benson2025} & 5 & 2 & 5 & 3 & 3 \\ \hline
\citeonline{Costa2023} & 5 & 5 & 4 & 3 & 4 \\ \hline
\citeonline{Casey2023} & 4 & 5 & 4 & 2 & 4 \\ \hline
\citeonline{Thomas2024} & 4 & 5 & 2 & 2 & 3 \\ \hline
\citeonline{Rahman2022} & 5 & 3 & 4 & 3 & 4 \\ \hline
\citeonline{Hogan2025} & 2 & 3 & 5 & 5 & 4 \\ \hline
\end{tabular}
\fonte{Elaborado pela autora}
\end{quadro}

A projeção visual destes dados, apresentada na \autoref{fig:radar}, permite a identificação dos pontos de convergência e divergência entre os estudos. O gráfico de radar evidencia a formação de eixos temáticos específicos.

\begin{figure}[htb]
    \centering
    \caption{Comparativo multidimensional dos trabalhos selecionados}
    \label{fig:radar}
    \includegraphics[width=0.8\textwidth]{figs/analise-multidimensional.png}
    \fonte{Elaborada pela autora}
\end{figure}

\subsection{Foco em Gênero e Barreiras (QP1)}
\label{ss_foco_genero}

A primeira dimensão do gráfico de radar, \textbf{Foco em Gênero (QP1)}, avalia a profundidade com que cada estudo analisa as barreiras socioculturais específicas para mulheres e propõe intervenções desenhadas exclusivamente para este público. Neste quesito, destacam-se os trabalhos de \citeonline{Costa2025} e \citeonline{Benson2025}, que obtiveram a pontuação máxima. O estudo de \citeonline{Costa2025}, referente ao programa \textit{CyberTrials}, diferencia-se por não apenas incluir mulheres, mas por estruturar todo o ambiente de aprendizado para combater estereótipos de gênero, alcançando 779 alunas com uma abordagem gamificada inclusiva que mitiga a percepção de masculinidade associada à área.

De forma complementar, \citeonline{Benson2025} foca nas barreiras de ascensão profissional, estabelecendo as \textit{soft skills} e a mentoria como ferramentas essenciais para a permanência feminina na carreira. \citeonline{Rahman2022} foca no nicho de mulheres em fase de reingresso no mercado de trabalho (\textit{re-entry}), propondo intervenções específicas para recuperar a confiança técnica. Em contrapartida, estudos como o de \citeonline{Hogan2025}, embora relevantes para a dinâmica de grupos, pontuaram menos nesta categoria por analisarem competições com times mistos sem uma intervenção desenhada primariamente para o recorte de gênero.

\subsection{Abordagens Pedagógicas (QP2)}
\label{ss_abordagens_pedagogicas}

No que tange às \textbf{Abordagens Pedagógicas (QP2)}, a análise priorizou trabalhos que detalham metodologias instrucionais capazes de reduzir a carga cognitiva e aumentar o engajamento de iniciantes. As pontuações mais altas foram atribuídas a \citeonline{Costa2023} e \citeonline{Costa2025}, devido à implementação bem-sucedida de narrativas (\textit{storytelling}) e \ac{RPG}. Nestes estudos, os autores demonstram que transformar exercícios técnicos em missões investigativas aumenta significativamente o interesse do público feminino, alterando a percepção da cibersegurança de uma área puramente técnica para uma voltada à resolução de problemas sociais.

Igualmente relevantes, \citeonline{Casey2023} e \citeonline{Thomas2024} destacam-se pela aplicação de \textit{scaffolding} (andaimes cognitivos) e currículos baseados em problemas reais (\textit{Problem-Based Learning}). Na Teoria Sociointeracionista de Lev Vygotsky, \textit{scaffolding} (andaimes cognitivos) é o suporte temporário e ajustável oferecido por um instrutor (ou uma ferramenta) para ajudar o estudante a realizar uma tarefa que ele ainda não consegue completar sozinho. \citeonline{Thomas2024}, especificamente, detalha como a decomposição de tarefas complexas em etapas menores é vital para evitar a frustração em grupos sub-representados. Trabalhos focados primariamente em gestão ou infraestrutura, como os de \citeonline{Tshekiso2025}, receberam pontuações menores nesta dimensão por utilizarem plataformas sem detalhar modificações curriculares profundas no escopo do artigo.

\subsection{Colaboração e Autoeficácia (QP3)}
\label{ss_colaboracao_autoeficacia}

A dimensão de \textbf{Colaboração e Autoeficácia (QP3)} examina a presença de mecanismos que fomentam o trabalho em equipe, o suporte de pares e a mentoria, fatores importantes para a retenção em programas massivos onde o isolamento é um risco. \citeonline{Hogan2025} apresenta a contribuição mais significativa neste aspecto, recebendo nota máxima ao investigar como times distribuídos geograficamente constroem confiança e colaboram eficazmente em ambientes online, uma descoberta diretamente aplicável ao modelo do \textit{Hackers} do Bem. Similarmente, \citeonline{Musuva2025} e \citeonline{Benson2025} são fundamentais por posicionarem a mentoria e o \textit{networking} não como acessórios, mas como pilares centrais para a construção da autoeficácia das participantes. \citeonline{Musuva2025}, no contexto do programa \textit{Cyber Shujaa}, demonstra que a conexão direta com mentores da indústria valida a identidade profissional das alunas, combatendo a síndrome do impostor.

\subsection{Escalabilidade e Alcance}
\label{ss_escalabilidade_alcance}

Em relação à \textbf{Escalabilidade}, buscou-se identificar estratégias viáveis para programas de abrangência nacional ou continental. O trabalho de \citeonline{Tshekiso2025} sobressai-se com pontuação máxima ao descrever a implementação de competições em diversos países africanos, demonstrando como o uso de plataformas abertas e agnósticas de infraestrutura permite escalar o ensino de cibersegurança em regiões com recursos limitados. \citeonline{Hogan2025} e \citeonline{Costa2025} também apresentam alta escalabilidade devido ao uso de ambientes virtuais e modelos híbridos que suportam centenas ou milhares de alunos simultâneos sem a necessidade de laboratórios físicos complexos. Por outro lado, intervenções como as de \citeonline{Casey2023} e \citeonline{Thomas2024}, embora pedagogicamente ricas, baseiam-se em \textit{workshops} presenciais (como acampamentos de verão) ou estudos de caso locais com turmas reduzidas, apresentando desafios maiores para uma replicação massiva imediata no contexto de um programa nacional.

\subsection{Validação Empírica}
\label{ss_validacao_empirica}

Por fim, a categoria de \textbf{Validação Empírica} ponderou o rigor metodológico e a presença de dados quantitativos ou qualitativos que sustentem as conclusões dos autores. \citeonline{Costa2025} estabelece o padrão de referência nesta seleção, apresentando resultados estatisticamente significantes derivados de grupos de controle e testes pré e pós-intervenção com uma amostra expressiva de participantes. \citeonline{Casey2023} também contribui com dados longitudinais coletados ao longo de três anos, oferecendo uma visão sólida sobre a eficácia de currículos inclusivos na mudança de percepção de carreira a longo prazo. 

\citeonline{Musuva2025} valida seu modelo através de métricas concretas de empregabilidade e taxas de conclusão. Estudos classificados com pontuação intermediária nesta dimensão, como \citeonline{Benson2025} e \citeonline{Thomas2024}, apoiam-se majoritariamente em análises qualitativas ou revisões de literatura que, apesar de oferecerem \textit{insights} valiosos sobre o comportamento e as barreiras subjetivas, carecem da validação estatística de larga escala presente nos demais trabalhos selecionados.

\section{Trabalhos Selecionados}
\label{s_trabalhos_selecionados}

As próximas subseções descrevem as características principais, metodologias e as contribuições específicas de cada um dos trabalhos selecionados.

\subsection{Gamificação Narrativa e Redução de Ansiedade em CyberTrials} \label{ss_costa_2025}

O trabalho de \citeonline{Costa2025} apresenta o desenho e a validação empírica do programa \textit{CyberTrials}, uma iniciativa realizada na Itália voltada para estudantes do ensino médio. Diferentemente de competições tradicionais de \ac{CTF}, que frequentemente priorizam a velocidade e o conhecimento técnico prévio, fatores que a literatura aponta como barreiras para iniciantes (QP1), este estudo propõe uma abordagem pedagógica baseada em gamificação narrativa e \ac{RPG}. O objetivo central dos autores foi investigar se o uso de metáforas lúdicas e ambientes colaborativos poderia mitigar a ansiedade tecnológica e aumentar a autoeficácia das participantes em tópicos avançados, como criptografia, segurança web e \ac{OSINT}.

Em relação aos resultados, \citeonline{Costa2025} demonstram, através de grupos de controle e pré/pós-testes, que a intervenção aumentou significativamente a intenção das participantes em seguir carreiras nas áreas de \ac{STEM} e cibersegurança. Para o desenvolvimento do framework de diretrizes desta dissertação, a contribuição crucial deste estudo reside na validação da gamificação estrutural (QP2): ao envolver os desafios técnicos em uma narrativa de investigação (onde as alunas atuam como \enquote{detetives} em vez de \enquote{hackers}), o programa reduziu a barreira de entrada e o estereótipo masculino associado à área. Além disso, o estudo confirma a hipótese da QP3, evidenciando que a formação de times e a colaboração intrínseca ao jogo foram determinantes para a manutenção do engajamento, contrastando com o isolamento comum em cursos massivos autodidatas.

Apesar dos resultados positivos, o trabalho apresenta limitações que devem ser consideradas na adaptação para outros programas. Primeiramente, a validação foca na mudança de intenção de carreira a curto prazo, carecendo de dados longitudinais que confirmem se essa motivação se traduz em ingresso efetivo no mercado de trabalho. Ademais, embora o modelo híbrido/online descrito suporte escalabilidade (atingindo cerca de 1.000 alunas), o estudo não detalha profundamente como a mentoria personalizada pode ser sustentada em programas de escala massiva (dezenas de milhares de alunos) sem incorrer em custos proibitivos, um desafio central para a QP2 em políticas públicas nacionais.

\subsection{Mentoria e Empregabilidade no Programa \textit{Cyber Shujaa}} \label{ss_musuva_2025}

O estudo de \citeonline{Musuva2025} detalha a implementação e os resultados do programa \textit{Cyber Shujaa}, uma iniciativa desenvolvida no Quênia para combater o desemprego juvenil e a escassez de mão de obra qualificada em cibersegurança. Diferentemente de cursos puramente técnicos, este programa adota uma abordagem holística que integra capacitação prática, imersão (residência) e suporte direto para colocação no mercado de trabalho. No contexto da QP3, o trabalho é fundamental por demonstrar que a formação técnica isolada é insuficiente para garantir a transição de carreira; a colaboração estruturada entre a academia e a indústria, materializada através de mentorias intensivas e feiras de carreira, provou ser o fator determinante para a retenção e o sucesso profissional dos participantes.

Em termos de resultados, o programa alcançou uma taxa de participação feminina de 41\%, um número expressivo obtido através de estratégias de recrutamento afirmativo e parcerias com comunidades locais. Para o \textit{framework} de diretrizes proposto nesta dissertação, a contribuição central de \citeonline{Musuva2025} reside na validação da mentoria de carreira como um mecanismo de redução da evasão. O estudo indica que o acompanhamento próximo por profissionais da indústria não apenas desenvolve competências técnicas, mas também fortalece as \textit{soft skills} e a identidade profissional das alunas, mitigando a síndrome do impostor frequentemente relatada por mulheres na área (QP1). A estrutura do programa sugere que a criação de pontes claras com o mercado de trabalho aumenta a percepção de utilidade do curso, incentivando a permanência.

Contudo, o trabalho apresenta limitações relacionadas à escalabilidade (QP2). O modelo do \textit{Cyber Shujaa} depende fortemente de interações presenciais ou híbridas e de um alto número de mentores humanos para um grupo relativamente menor de alunos, o que representa um desafio logístico e financeiro para programas nacionais que visam atingir dezenas de milhares de estudantes simultaneamente. Portanto, o desafio para o \textit{framework} será adaptar esses mecanismos de \enquote{alta interação} descritos pelo estudo para um ambiente virtual massivo, possivelmente através de sistemas de mentoria em pares ou comunidades de prática online, sem perder a qualidade do suporte humano que os autores identificaram como crucial.

\subsection{Escalabilidade Continental via Plataformas Abertas: O Caso \textit{picoCTF-Africa}}
\label{ss_tshekiso_2025}

O trabalho de \citeonline{Tshekiso2025} documenta a implementação e os desafios da iniciativa \textit{picoCTF-Africa}, um programa de capacitação desenhado para mitigar a escassez crítica de força de trabalho em cibersegurança no continente africano. Diferentemente de intervenções locais ou de curta duração, o objetivo central deste estudo foi estabelecer um modelo de escalabilidade sustentável que pudesse operar em múltiplos países simultaneamente, superando barreiras de infraestrutura e custos. O contexto da pesquisa envolve a adaptação da plataforma global de competições \textit{picoCTF} para o cenário regional, integrando escolas de ensino médio e universidades através de uma estrutura hierárquica de capítulos (\textit{chapters}) locais. Para a \textbf{QP2}, este estudo é particularmente relevante por demonstrar como o uso de ambientes de treinamento gamificados e agnósticos de \textit{hardware} permite a democratização do acesso ao conhecimento técnico avançado em regiões com recursos limitados.

Em relação aos resultados, os autores relatam um crescimento expressivo na participação, saltando de números incipientes para milhares de estudantes engajados em competições anuais, validando a eficácia da estratégia de descentralização via (\textit{chapters}) locais. Uma contribuição vital para o \textit{framework} de diretrizes desta dissertação reside na abordagem de comunidades de prática distribuídas (QP3): o estudo evidencia que a criação de clubes locais de cibersegurança, liderados por embaixadores regionais, foi essencial para manter a motivação dos alunos e fornecer o suporte técnico inicial que a plataforma online sozinha não supriria. Além disso, a iniciativa implementou uma categoria específica para mulheres e meninas como ação afirmativa, reconhecendo que a competição aberta, por si só, não garante a equidade de gênero, alinhando-se às preocupações da \textbf{QP1} sobre barreiras estruturais.

Entretanto, ao analisar o trabalho sob a ótica da customização pedagógica, nota-se uma limitação importante. Embora \citeonline{Tshekiso2025} tenham alcançado sucesso em escala, a metodologia pedagógica baseia-se fundamentalmente no modelo padrão de \ac{CTF} estilo \textit{Jeopardy}\footnote{Um \ac{CTF} do tipo Jeopardy é um modelo de competição de cibersegurança onde os participantes ou equipes devem resolver uma série de desafios técnicos organizados por categorias e níveis de dificuldade. Diferente do modelo Attack-Defense (onde as equipes atacam umas às outras), no Jeopardy os competidores enfrentam uma plataforma centralizada que disponibiliza \enquote{cartões} de questões. Cada desafio resolvido fornece uma sequência de caracteres única, chamada de \textit{flag} (bandeira), que é submetida ao sistema em troca de pontos.}, sem modificações profundas no desenho instrucional para torná-lo intrinsecamente mais inclusivo ou menos intimidante para iniciantes absolutos. O estudo foca mais na logística de expansão e na infraestrutura de competição do que na adaptação curricular ou em estratégias de \textit{scaffolding} cognitivo para retenção de longo prazo. Essa lacuna reforça a necessidade de o \textit{framework} proposto nesta pesquisa ir além da disponibilização massiva de desafios, integrando narrativas e trilhas de aprendizado adaptativas que sustentem a permanência do público feminino para além do evento competitivo inicial.

\subsection{Soft Skills e Mentoria como Estratégias de Retenção} \label{ss_benson_2025}

O trabalho de \citeonline{Benson2025} investiga o papel crítico das competências não técnicas (\textit{soft skills}) na permanência e ascensão de mulheres na carreira de cibersegurança. Diferentemente de abordagens focadas exclusivamente na capacitação técnica, este estudo qualitativo utiliza a Teoria Social Cognitiva de Carreira para analisar como atributos como resiliência, comunicação e liderança atuam como fatores de proteção contra barreiras estruturais e o isolamento profissional, respondendo diretamente à QP1. O objetivo central dos autores foi identificar quais competências comportamentais e redes de suporte são percebidas pelas próprias mulheres como determinantes para a superação de inseguranças e para a manutenção da identidade profissional em um ambiente predominantemente masculino.

Em seus resultados, \citeonline{Benson2025} identificaram quatro pilares temáticos essenciais: confiança e resiliência, mentoria e patrocínio (\textit{sponsorship}), colaboração na resolução de problemas e políticas inclusivas. Para a construção do \textit{framework} de diretrizes desta dissertação, a contribuição mais significativa reside na validação da colaboração e mentoria (QP3) como mecanismos indispensáveis de retenção. O estudo demonstra que ambientes de aprendizado que fomentam a resolução colaborativa de problemas (\textit{collaborative problem-solving}) não apenas desenvolvem habilidades técnicas, mas criam um senso de comunidade que ajuda a mitigar a síndrome do impostor. Os dados indicam que a inclusão de programas formais de mentoria e \textit{workshops} colaborativos é tão vital quanto o conteúdo técnico para garantir a autoeficácia das participantes, sugerindo que a estrutura do curso deve prever espaços seguros para o desenvolvimento destas competências interpessoais.

Entretanto, a aplicação direta destes achados no contexto de programas massivos apresenta desafios significativos relacionados à escalabilidade. O estudo de \citeonline{Benson2025} baseia-se em entrevistas de profundidade e dinâmicas interpessoais ricas, cuja replicação em um ambiente virtual com milhares de alunos exige adaptações complexas. A principal lacuna do trabalho, sob a ótica da QP2, é a ausência de uma proposta curricular técnica integrada; o foco recai quase inteiramente sobre aspectos comportamentais, sem detalhar como essas \textit{soft skills} podem ser pedagogicamente embutidas nos exercícios técnicos de um curso \ac{EAD}. Portanto, o desafio para o \textit{framework} será operacionalizar a \enquote{mentoria} e a \enquote{colaboração} descritas pelos autores de forma assíncrona ou escalável, sem depender exclusivamente da interação humana síncrona intensiva que é financeiramente inviável em larga escala.

\subsection{Narrativa e Engajamento Feminino: O Caso \textit{Why Mary Can Hack}} \label{ss_costa_2023}

O trabalho de \citeonline{Costa2023} aborda a sub-representação feminina na área de cibersegurança através de uma intervenção pedagógica denominada \textit{CyberTrials}, voltada para estudantes do ensino médio. O contexto da pesquisa parte da premissa de que a baixa adesão feminina (QP1) não decorre da falta de habilidade técnica, mas sim de estereótipos culturais e da forma como o conteúdo é tradicionalmente apresentado. O objetivo central dos autores foi investigar se a introdução de técnicas de \textit{storytelling} (narrativa) e gamificação não competitiva poderia alterar a percepção das alunas sobre a área, transformando exercícios técnicos em missões de investigação com propósito social.

Em termos de resultados, o estudo demonstrou que a contextualização dos desafios técnicos dentro de uma narrativa de mistério (onde as alunas atuam como investigadoras para resolver um caso de \textit{ransomware}\footnote{Um tipo de \textit{software} malicioso que sequestra os dados de um dispositivo por meio de criptografia. Uma vez infectado, os arquivos tornam-se ilegíveis para o usuário. Os criminosos então exigem o pagamento de um resgate para fornecer a chave que desbloqueia os dados.}) aumentou significativamente o engajamento e a autoeficácia das participantes. Para a construção do \textit{framework} de diretrizes desta dissertação, a contribuição fundamental de \citeonline{Costa2023} para a QP2 é a validação de que a \enquote{camada de apresentação} do conteúdo é tão importante quanto o conteúdo em si. O uso de metáforas lúdicas e a conexão dos laboratórios práticos com problemas do mundo real permitiram que as alunas superassem a barreira da ansiedade tecnológica, vendo a ferramenta técnica como um meio para um fim socialmente relevante, e não como um fim em si mesmo.

Apesar do sucesso da intervenção, o trabalho apresenta limitações quando contrastado com a escala massiva de programas como o \textit{Hackers} do Bem. O modelo descrito em \citeonline{Costa2023} beneficiou-se de uma interação síncrona e monitorada, com forte componente humano de suporte, o que é difícil de replicar em cursos autoinstrucionais com milhares de alunos. Além disso, embora a narrativa tenha sido eficaz para a atração inicial, o estudo não aprofunda como essa estratégia de \textit{storytelling} pode ser sustentada em níveis avançados de formação técnica, onde a complexidade dos conceitos exige maior abstração e rigor. O desafio para o \textit{framework} será, portanto, adaptar os princípios de narrativa inclusiva para trilhas de aprendizado assíncronas, criando um senso de propósito e contexto sem depender da mediação humana constante.

\subsection{Aprendizado Baseado em Problemas e Contexto Social no \textit{Cyber Sleuth Science Lab}}
\label{ss_casey_2023}

O trabalho de \citeonline{Casey2023} apresenta o desenvolvimento e a validação do \ac{CSSL}, um ambiente de aprendizagem inteligente desenhado especificamente para aumentar a participação de grupos sub-representados, com ênfase no público feminino, em carreiras de tecnologia. Diferenciando-se de cursos tradicionais que focam na aquisição de habilidades técnicas isoladas (como configuração de \textit{firewalls} ou codificação de \textit{scripts}), a abordagem pedagógica adotada fundamenta-se no \ac{PBL} e na narrativa investigativa. O objetivo central dos autores foi contextualizar a cibersegurança e a forense digital dentro de problemas do mundo real que possuem relevância social imediata para os estudantes, como o combate ao \textit{cyberbullying} e a recuperação de dados perdidos, alterando a percepção da área de uma disciplina puramente técnica para uma ferramenta de justiça social (QP1).

Em relação aos resultados, o estudo demonstrou que a utilização de casos práticos guiados por uma estrutura de \enquote{andaimes cognitivos} (\textit{scaffolding}) aumentou significativamente a autoeficácia das participantes. Para a construção do \textit{framework} de diretrizes desta dissertação, a contribuição vital de \citeonline{Casey2023} para a QP2 e QP3 reside na demonstração de que a redução da carga cognitiva inicial, obtida através de guias passo a passo e decomposição de tarefas complexas, é essencial para evitar a frustração em iniciantes. Além disso, o estudo validou que a colaboração em pequenos grupos e a presença de modelos femininos (\textit{role models}) atuando como facilitadoras foram determinantes para que as meninas se sentissem pertencentes ao ambiente, sugerindo que mecanismos colaborativos não são opcionais, mas sim componentes estruturais da retenção.

O sucesso das intervenções descritas por \citeonline{Casey2023} dependeu fortemente de mediação humana síncrona, com instrutores e facilitadores presentes em sala de aula ou em sessões remotas intensivas para guiar a discussão e prover suporte emocional e técnico imediato. Em um programa massivo e assíncrono, replicar esse nível de \enquote{toque humano} é financeiramente e logisticamente complexo. Portanto, a lacuna que o \textit{framework} proposto precisará preencher é como simular esse \textit{scaffolding} e esse senso de colaboração investigativa utilizando recursos automatizados da plataforma e comunidades de prática, sem depender da proporção de instrutor-aluno utilizada no estudo de caso descrito pelo estudo.

\subsection{Adaptação Curricular e \textit{Scaffolding} para Populações Sub-representadas}
\label{ss_thomas_2024}

O estudo de \citeonline{Thomas2024} aborda a implementação de currículos de cibersegurança voltados para estudantes do ensino fundamental provenientes de comunidades sub-representadas e economicamente desfavorecidas. Diferentemente de treinamentos corporativos padronizados, o contexto desta pesquisa foca na equidade do acesso (QP1), investigando como as barreiras de letramento digital e a complexidade técnica inerente à área podem ser mitigadas através de intervenções pedagógicas intencionais. O objetivo central dos autores foi avaliar a eficácia de um currículo ajustado que utiliza atividades práticas (\textit{hands-on}) e a redução da carga teórica expositiva para manter o engajamento de alunos com diferentes níveis de proficiência prévia.

Em relação aos resultados, \citeonline{Thomas2024} identificaram que a estrutura tradicional de aulas expositivas gerava desengajamento rápido, sendo necessário limitar explanações teóricas a curtos intervalos de tempo, intercalados com experimentação prática. Para a construção do \textit{framework} de diretrizes desta dissertação, a contribuição fundamental deste trabalho para a QP2 é a validação da necessidade de \textit{scaffolding} (andaimes cognitivos) multinível. Os autores demonstram que, para garantir a permanência de grupos heterogêneos, é crucial oferecer \enquote{extensões de lições} para alunos avançados e suportes visuais ou pictóricos para aqueles com dificuldades de leitura ou barreiras linguísticas. Isso sugere que o \textit{design} instrucional deve ser flexível o suficiente para acomodar o ritmo individual, evitando que a frustração técnica inicial se converta em evasão.

Apesar da relevância pedagógica, a aplicação direta do modelo de \citeonline{Thomas2024} encontra limitações na dimensão da escalabilidade. A intervenção descrita baseou-se em acampamentos de verão presenciais (\textit{summer camps}) com forte mediação de instrutores que ajustavam o conteúdo em tempo real conforme a reação da turma. Em um programa massivo e predominantemente assíncrono, essa personalização humana intensiva é inviável. Portanto, a falha do trabalho em propor mecanismos de \textit{scaffolding} automatizado representa uma lacuna que o \textit{framework} proposto precisará preencher, traduzindo o suporte presencial descrito pelos autores em funcionalidades de plataforma e trilhas adaptativas que possam operar em larga escala sem intervenção manual constante.

\subsection{Iniciativas de Reingresso e Recuperação da Confiança Técnica}
\label{ss_rahman_2022}

O trabalho de \citeonline{Rahman2022} aborda um segmento crítico e frequentemente negligenciado nas políticas de inclusão: mulheres que buscam retornar à força de trabalho em computação após um período de afastamento (\textit{career break}). O contexto da pesquisa identifica que este grupo enfrenta barreiras distintas (QP1), como a defasagem técnica percebida e uma severa diminuição da autoeficácia profissional, exacerbada pelo ritmo acelerado de atualização das tecnologias de cibersegurança. O objetivo central dos autores é validar intervenções de \textit{re-entry} (reingresso) que combinam atualização técnica rápida com suporte emocional e reconstrução de identidade profissional, questionando a eficácia de cursos que ignoram a bagagem prévia e as inseguranças específicas deste público.

Em termos de contribuições para o \textit{framework} de diretrizes, o estudo é particularmente relevante para a \textbf{QP3} ao demonstrar que a recuperação da confiança técnica não ocorre em isolamento. \citeonline{Rahman2022} evidenciam que o elemento de \textit{networking} e a convivência com pares que compartilham a mesma trajetória de retorno são tão vitais quanto o conteúdo técnico em si. Isso sugere que a criação de comunidades ou trilhas específicas para alunas em transição de carreira ou retorno ao mercado pode ser um fator decisivo de retenção, mitigando a sensação de não pertencimento que ocorre quando estas alunas são colocadas em competição direta com jovens nativos digitais sem as mesmas responsabilidades familiares ou lacunas curriculares.

Contudo, a transposição deste modelo apresenta desafios de escalabilidade e validação massiva. A intervenção descrita baseia-se em \textit{workshops} presenciais ou síncronos de menor escala, típicos de conferências acadêmicas. Uma limitação do trabalho, sob a ótica da \textbf{QP2}, é a falta de detalhamento sobre como a pedagogia de \enquote{atualização rápida} pode ser automatizada em uma plataforma \ac{EAD} sem a mediação humana constante de mentores. O desafio para o \textit{framework} será, portanto, adaptar os princípios de acolhimento e validação de identidade descritos pelos autores para mecanismos escaláveis, como sistemas de recomendação de pares ou mentorias coletivas virtuais.

\subsection{Dinâmicas de Colaboração em Times Distribuídos e Confiança Rápida} \label{ss_hogan_2025}

O trabalho de \citeonline{Hogan2025} investiga os comportamentos e padrões de comunicação de equipes de alto desempenho em competições de cibersegurança do tipo \ac{CTF} realizadas em formato totalmente remoto. O contexto da pesquisa é particularmente relevante para o estudo de caso do \textit{Hackers} do Bem (QP2) pois, diferentemente de intervenções presenciais, este estudo analisa como a aprendizagem e a resolução de problemas ocorrem em ambientes virtuais distribuídos, onde a interação física é inexistente. O objetivo central dos autores foi identificar como grupos geograficamente dispersos constroem a chamada \enquote{confiança rápida} (\textit{swift trust}) e estabelecem lideranças emergentes para superar desafios técnicos complexos sem a supervisão direta de instrutores.

Em relação aos resultados, o estudo revela que o sucesso na retenção e na performance técnica não depende apenas da habilidade individual, mas da capacidade do grupo em realizar \textit{scaffolding} entre pares, ou seja, membros mais experientes suportando o aprendizado dos novatos em tempo real. Para o \textit{framework} de diretrizes desta dissertação, a contribuição de \citeonline{Hogan2025} para a QP3 é a constatação de que a colaboração online eficaz exige canais de comunicação que permitam não apenas a troca de dados técnicos, mas também a validação emocional e o gerenciamento de frustração. Os autores demonstram que a autoeficácia dos participantes é ampliada quando o ambiente virtual simula a camaradagem de um laboratório físico, sugerindo que plataformas de ensino massivo devem incorporar ferramentas de chat e formação de comunidades persistentes para mitigar o isolamento.

Apesar dos \textit{insights} valiosos sobre colaboração remota, o trabalho apresenta limitações importantes quando contrastado com o objetivo de inclusão feminina (QP1). A amostra do estudo é composta por times \enquote{altamente bem-sucedidos}, o que introduz um viés de sobrevivência; não há uma análise profunda sobre os times que falharam ou desistiram, onde provavelmente se encontram as barreiras que afetam desproporcionalmente as mulheres e iniciantes inseguros. Além disso, ao focar na observação de dinâmicas orgânicas em competições mistas, o trabalho não propõe intervenções estruturais para combater a toxicidade ou o domínio masculino nesses espaços, uma lacuna que o \textit{framework} proposto precisará preencher para garantir que a \enquote{colaboração} descrita não se torne um ambiente hostil para o público feminino.

\section{Comparação e Lacunas Identificadas}
\label{s_comparacao_lacunas_identificadas}

O \autoref{quadro:comparacao_trabalhos} sintetiza a comparação entre os nove trabalhos selecionados e a proposta desta dissertação. A análise da literatura revela que, embora existam iniciativas robustas em isolamento, há uma carência de modelos que integrem simultaneamente pedagogias de inclusão profunda (como narrativas e \textit{scaffolding}) com estratégias de larga escala (como plataformas \ac{EAD} massivas).

Enquanto trabalhos como os de \citeonline{Costa2025}, \citeonline{Casey2023} e \citeonline{Thomas2024} oferecem excelentes evidências sobre a eficácia de adaptações curriculares e gamificação narrativa para o público feminino (QP1 e QP2), eles operam majoritariamente em escalas locais ou regionais (\textit{workshops}, turmas controladas), limitando a generalização para políticas públicas nacionais. Por outro lado, estudos focados em alta escalabilidade, como \citeonline{Tshekiso2025} e \citeonline{Hogan2025}, abordam a infraestrutura e a colaboração em times distribuídos (QP3), mas frequentemente carecem de um recorte de gênero específico ou de intervenções pedagógicas desenhadas para mitigar a ansiedade tecnológica de iniciantes.

A lacuna central identificada reside entre escalabilidade massiva e customização da experiência de aprendizado. Embora \citeonline{Musuva2025} e \citeonline{Benson2025} evidenciem a eficácia de práticas de \enquote{alto toque humano}, como mentoria intensiva, acolhimento e desenvolvimento de \textit{soft skills}, para a retenção feminina (QP1 e QP3), não foi encontrado na literatura um modelo que operacionalize essas estratégias em ambientes assíncronos e de alcance nacional. Desta forma, esta dissertação propõe preencher essa lacuna utilizando o programa \textit{Hackers} do Bem como estudo de caso, com o objetivo de desenvolver um \textit{framework} de diretrizes que oriente a adaptação e a integração dessas abordagens inclusivas já validadas para o contexto de formação em larga escala.

\begin{quadro}[!htbp]
    \centering
    \caption{Comparação dos trabalhos relacionados com a proposta atual}
    \label{quadro:comparacao_trabalhos}
    \begin{footnotesize}
    \resizebox{\textwidth}{!}{%
    \begin{tabular}{|l|c|c|c|c|l|}
        \hline
        \rowcolor[HTML]{EFEFEF} 
        \textbf{Trabalho} & \textbf{Foco Gênero} & \textbf{Escalabilidade} & \textbf{Pedagogia Inclusiva} & \textbf{Colab./Mentoria} & \textbf{Limitação Principal (Gap)} \\ \hline
        
        \citeonline{Costa2025}    & Alta  & Média & Alta  & Média & Intervenção pontual, falta dados de carreira a longo prazo. \\ \hline
        \citeonline{Musuva2025}   & Média & Média & Média & Alta  & Modelo presencial intensivo, difícil de escalar massivamente. \\ \hline
        \citeonline{Tshekiso2025} & Baixa & Alta  & Baixa & Média & Foco em infraestrutura/CTF padrão, sem pedagogia de gênero. \\ \hline
        \citeonline{Benson2025}   & Alta  & Baixa & Baixa & Alta  & Estudo teórico/qualitativo, sem plataforma de ensino aplicada. \\ \hline
        \citeonline{Costa2023}    & Alta  & Média & Alta  & Alta  & Foco em atração (Ensino Médio), não em formação profissional. \\ \hline
        \citeonline{Casey2023}    & Alta  & Baixa & Alta  & Alta  & Depende de mediação humana intensa (workshops presenciais). \\ \hline
        \citeonline{Thomas2024}   & Alta  & Baixa & Alta  & Baixa & Estudo de caso pequeno (N=24), foco em suporte individual. \\ \hline
        \citeonline{Rahman2022}   & Alta  & Baixa & Média & Alta  & Foco específico em retorno ao trabalho, escala de workshop. \\ \hline
        \citeonline{Hogan2025}    & Baixa & Alta  & Média & Alta  & Analisa times mistos \enquote{de sucesso}, viés de sobrevivência. \\ \hline
        
        \rowcolor[HTML]{C0C0C0} 
        \textbf{Esta Dissertação} & \textbf{Alta} & \textbf{Alta} & \textbf{Alta} & \textbf{Média} & \textbf{Validação em ambiente real massivo (\textit{Hackers do Bem}).} \\ \hline
    \end{tabular}%
    }
    \end{footnotesize}
    \fonte{Elaborado pela autora com base na análise sistemática.}
\end{quadro}

\section{Considerações Finais}
\label{s_consideracoes_finais}

A análise dos trabalhos relacionados permite concluir que a decomposição do problema em fatores isolados, focando puramente em gamificação técnica ou exclusivamente em políticas de gênero, é insuficiente para abordar a complexidade da evasão feminina em programas de formação massiva. Existe uma clara dicotomia na literatura vigente: de um lado, intervenções altamente eficazes e inclusivas baseadas em narrativas e \textit{scaffolding}, como demonstrado por \citeonline{Costa2025} e \citeonline{Casey2023}, que operam em escalas reduzidas e controladas; do outro, modelos de alta escalabilidade técnica focados em infraestrutura, como apresentado por \citeonline{Tshekiso2025} e \citeonline{Hogan2025}, que frequentemente negligenciam as nuances pedagógicas necessárias para a retenção de grupos sub-representados.

O programa \textit{Hackers} do Bem, que será utilizado como estudo de caso desta dissertação, situa-se no centro desta dicotomia. Ao combinar o ensino em larga escala com um sistema de ranqueamento gamificado e pontos de experiência (\ac{XP}), o programa apresenta um cenário único onde as mecânicas de competição podem atuar de forma ambivalente: como fator de engajamento para alguns ou como barreira de exclusão para iniciantes inseguras, conforme alertam os achados sobre autoeficácia de \citeonline{Rahman2022}. A ausência de um modelo na literatura que integre a \enquote{pedagogia do acolhimento} (baseada em mentoria e suporte de pares) com a \enquote{eficiência da escala} (baseada em automação e \ac{EAD}) evidencia a lacuna teórica e prática que este trabalho busca preencher.