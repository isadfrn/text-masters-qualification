%% Como usar o pacote acronym


% Na primeira vez que for citado o acronimo, o nome completo 
% irá aparecer seguido do acronimo entre parênteses. Na 
% proxima vez somente o acronimo irá aparecer. Se usou a 
% opção footnote no pacote, entao o nome por extenso irá 
% aparecer no rodapé \ac{acronimo}


% Para aparecer com nome completo + acronimo
% \acf{acronimo}

% Para aparecer somente o acronimo
% \acs{acronimo}

% Nome por extenso somente, sem o acronimo
% \acl{acronimo}

% igual o \ac mas deixando no plural com S (ingles)
% \acp{acronimo}

% \acfp{acronimo}

% \acsp{acronimo}

% \aclp{acronimo}

%% ATENCAO
% Criei o comando \acfe{}, resultando em: Extenso -- ACRO

\chapter*{Lista de Abreviaturas}%
% \addcontentsline{toc}{chapter}{Lista de abreviaturas}
\markboth{Lista de abreviaturas}{}


\begin{acronym}







%A
\acro{AA}{\textit{Attribute Authority}}
\acro{ABAC}{\textit{Attribute-Based Access Control}}
\acro{AC}{Autoridade Certificadora}
\acro{AD}{\textit{Active Directory}}
\acro{AES}{\textit{Advanced Encryption Standard}}
\acro{AP}{\textit{Attribute Provider}}
\acro{APIs}{\textit{Application Programming Interfaces}}
\acro{APP}{\textit{Application}}
\acro{ARC}{\textit{Advanced Resource Connector}}
\acro{AS}{\textit{Attribute Service}}
\acro{ASM}{\textit{Authenticator Abstraction Layer}}
\acro{ASP}{\textit{Application Service Provider}}
\acro{ATN}{\textit{Automated Trust Negotiation}}
%B
\acro{BLE}{\textit{Bluetooth Low Energy}}
%C
\acro{CA}{\textit{Certificate Authority}}
\acro{CN}{\textit{Collaborative Networks}}
\acro{CNH}{Carteira Nacional de Habilitação}
\acro{CNPJ}{Cadastro Nacional de Pessoa Jurídica}
\acro{CPF}{Cadastro de Pessoa Física}
\acro{CS}{\textit{Consent Service}}
\acro{CSR}{\textit{Certificate Signing Request}}
\acro{CPU}{\textit{Central Processing Unit}}
%D
\acro{DFN}{\textit{Deutsches Forschungsnetz}}
\acro{DN}{\textit{Distinguised Name}}
\acro{DS}{\textit{Discovery Service}}
\acro{DVO}{\textit{Dynamic Virtual Organization}}
%E
\acro{e-CPF}{Cadastro de Pessoal Física Eletrônico}
\acro{e-CNPJ}{Cadastro Nacional de Pessoa Jurídica Eletrônico}
\acro{e-Gov}{\textit{Electronic Government}}
\acro{e-PING}{Padrões de Interoperabilidade em Governo Eletrônico}
\acro{eID}{\textit{Electronic Identity}}
\acro{eIDMS}{\textit{electronic Identity Management System}}
\acro{EGDI}{\textit{e-Government Development Index}}
\acro{ePWG}{Padrões Web para Governo Eletrônico}
\acro{ETSI}{\textit{European Telecommunications Standards Institute}}
%F
\acro{FIDO}{\textit{Fast IDentity Online}}
\acro{FIM}{\textit{Federated Identity Management}}
\acro{FO}{\textit{Federation Operator}}
%G
\acro{G2C}{Governo para Cidadão}
\acro{G2E}{\textit{Government to Employees}}
\acro{G2G}{Governo para Governo}
\acro{GId}{Gestão de Identidade}
\acro{GSI}{\textit{Grid Security Infrastructure}}
\acro{GTTI}{Grupo de Trabalho Interministerial}
%H
\acro{HCE}{\textit{Host-based Card Emulation}}
\acro{HCI}{\textit{Human Capital Index}}
\acro{HPC}{\textit{High Performance Computing}}
\acro{HSM}{\textit{Hardware Security Module}}
\acro{HTTP}{\textit{Hypertext Transfer Protocol}}
%I
\acro{IAA}{\textit{Identity, Authentication, Authorization}}
\acro{IAF}{\textit{Identity Assurance Framework}}
\acro{IAM}{\textit{Identity and Access Management}}
\acro{IBAC}{\textit{Identity-Based Access Control}}
\acro{ICFF}{\textit{Intercloud Federation Framework}}
\acro{ICP}{Infraestrutura de Chave Pública}
\acro{Id}{\textit{Identity}}
\acro{ID}{Identificador}
\acro{IDE}{\textit{Integrated Development Environment}}
\acro{IdM}{\textit{Identity Management System}}
\acro{IdP}{\textit{Identity Provider}}
\acro{IGF}{\textit{Identity Governance Framework}}
\acro{IMSI}{\textit{International Mobile Subscriber Identity}}
\acro{IP}{\textit{Internet Protocol}}
\acro{IRPF}{Imposto de Renda de Pessoal Física}
%J
\acro{JSON}{\textit{JavaScript Object Notation}}
%K
%L
\acro{LDAP}{\textit{Lightweight Directory Access Protocol}}
\acro{LSDMA}{\textit{Large Scale Data Management}}
%M
\acro{MITM}{\textit{Man-in-The-Middle}}
\acro{myVOCS}{\textit{my Virtual Organization Collaboration System}}
%N
\acro{NFC}{\textit{Near Field Communication}}
\acro{NREN}{\textit{National Research Education Networks}}
%O
\acro{OASIS}{\textit{Organization for the Advancement of Structured Information Standards}}
\acro{OCSP}{\textit{Online Certificate Status Protocol}}
\acro{ONG}{Organizações Não Governamentais}
\acro{OSI}{\textit{Online Service Index}}
\acro{OTP}{\textit{One-time Password}}
\acro{OV}{Organização Virtual}
%P
\acro{PaaS}{\textit{Plataform-as-a-Service}}
\acro{PAPI}{\textit{Point of Access to Providers of Information}}
\acro{PC}{\textit{Professional Communities}}
\acro{PDP}{\textit{Policy Decision Point}}
\acro{PEP}{\textit{Policy Enforcement Point}}
\acro{PERMIS}{\textit{Privilege Managerment Infrastructure}}
\acro{PKI}{\textit{Public Key Infrastructure}}
\acro{PIC}{\textit{Personal Identification Code}}
\acro{PIN}{\textit{Personal Identification Number}}
\acro{PoA}{\textit{Point of Authentication}}
\acro{PPP}{Parceria Público-Privada}
\acro{PVC}{\textit{Professional Virtual Communities}}
%Q
%R
\acro{RA}{\textit{Registration Authority}}
\acro{RBAC}{\textit{Role-Based Access Control}}
\acro{RCN}{Registro Civil Nacional}
\acro{REE}{\textit{Rich Execution Environment}}
\acro{RIC}{Registro de Identidade Civil}
\acro{RG}{Registro Geral}
\acro{RP}{\textit{Relying Party}}
\acro{RSA}{}
\acro{RSL}{Revisão Sistemática da Literatura}
%S
\acro{SAML}{\textit{Security Assertion Markup Language}}
\acro{SAT}{\textit{SIM Application Toolkit}}
\acro{SB}{\textit{Service Broker}}
\acro{SCE}{\textit{Secure Collaborative Environment}}
\acro{SE}{\textit{Secure Element}}
\acro{SEM}{\textit{Scanning Electron Microscope}}
\acro{SGId}{Sistema de Gestão de Identidade}
\acro{SIG}{\textit{Special Interest Group}}
\acro{SIM}{\textit{Subscriber Identification Module}}
\acro{SISP}{Sistema de Administração dos Recursos de Tecnologia da Informação}
\acro{SLCS}{\textit{Short Lived Credential Service}}
\acro{SMC}{\textit{Secure Memory Card}}
\acro{SO}{Sistema Operacional}
\acro{SOAP}{\textit{Simple Object Access Protocol}}
\acro{SoC}{\textit{System on Chip}}
\acro{SP}{\textit{Service Provider}}
\acro{SSL}{\textit{Secure Sockets Layer}}
\acro{SSO}{\textit{Single Sign-On}}
\acro{SSTC}{\textit{Security Services Technical Committee}}
\acro{STORK}{\textit{Secure idenTity acrOss boRders linKed}}
%T
\acro{TAL}{\textit{Trust Anchor List}}
\acro{TEE}{\textit{Trusted Execution Environment}}
\acro{TIC}{Tecnologia da Informação e Comunicação}
\acro{TII}{\textit{Telecommunication Infrastructure Index}}
\acro{TLS}{\textit{Transport Layer Security}}
\acro{TSE}{Tribunal Superior Eleitoral}
%U
\acro{U2F}{\textit{Universal Second Factor}}
\acro{UA}{\textit{User Agent}}
\acro{UAF}{\textit{Universal Authentication Framework}}
\acro{UICC}{\textit{Universal Integrated Circuit Card}}
\acro{URL}{\textit{Uniform Resource Locator}}
\acro{USB}{\textit{Universal Serial Bus}}
\acro{USIM}{\textit{Universal Subscriber Identity Module}}
%V
\acro{VC}{\textit{Virtual Communities}}
\acro{VE}{\textit{Virtual Enterprises}}
\acro{VL}{\textit{Virtual Laboratories}}
\acro{VMM}{\textit{Virtual Machine Monitor}}
\acro{VO}{\textit{Virtual Organizations}}
\acro{VOMS}{\textit{Virtual Organization Management System}}
\acro{VRE}{\textit{Virtual Research Environment}}
%W
\acro{WAYF}{\textit{Where Are You From}}
\acro{WWW}{\textit{World Wide Web}}
%X
\acro{XACML}{\textit{eXtensible Access Control Markup Language}}
\acro{XML}{\textit{EXtensible Markup Language}}
%Y
%Z






\end{acronym}