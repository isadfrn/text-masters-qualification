\begin{resumo}
    A crescente demanda por profissionais de cibersegurança e a instituição da Estratégia Nacional de Cibersegurança (E-Ciber) impulsionaram a criação de programas de formação em larga escala no Brasil. No entanto, a persistente sub-representação feminina no setor, estagnada em cerca de 25\% globalmente, desafia a eficácia dessas iniciativas em promover a diversidade. Esta dissertação investiga o fenômeno do \textit{leaky pipeline} (vazamento de talentos) no contexto do Programa \textit{Hackers} do Bem, analisando se as mecânicas de gamificação competitiva e ranqueamento, típicas de formações massivas, atuam como barreiras de permanência para mulheres. A pesquisa adota uma abordagem de métodos mistos, combinando a mineração de dados educacionais para mapear os pontos críticos de evasão no funil de formação com a análise qualitativa da percepção de autoeficácia das participantes. Como resultado, propõe-se o desenvolvimento e validação de um \textbf{\textit{Framework} de Diretrizes para Inclusão e Retenção}. Este artefato estrutura-se em três eixos estratégicos: Contextualização Narrativa, Mecânicas de Avaliação Inclusiva e Suporte Social Escalável, visando conciliar a eficiência técnica da formação massiva com as práticas pedagógicas necessárias para ampliar o ingresso e a retenção de talentos femininos.
\end{resumo}