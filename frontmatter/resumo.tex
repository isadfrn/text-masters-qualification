\begin{resumo}
    A demanda por profissionais em segurança cibernética cresce, mas a baixa participação feminina no setor permanece um problema. O Programa \textit{Hackers} do Bem busca capacitar profissionais em larga escala, usando métodos de gamificação e rankings. Este estudo investiga se esses mecanismos competitivos, sem ajustes adequados, podem aumentar a evasão de mulheres durante a formação. O objetivo é propor um conjunto de diretrizes para melhorar a inclusão e a retenção, baseado em uma análise quantitativa e qualitativa da participação feminina no programa. A pesquisa justifica-se pela necessidade de avaliar se o modelo atual promove a equidade de gênero, oferecendo métricas concretas para intervenções educacionais massivas. A metodologia utiliza métodos mistos: uma Revisão Sistemática da Literatura para estabelecer parâmetros de comparação, mineração de dados educacionais para calcular taxas de evasão e eficiência em cada etapa do curso, e questionários de autoeficácia para entender fatores subjetivos da desistência. Como resultados, espera-se identificar gargalos de retenção por gênero, correlacionar pontos de experiência com desempenho e fornecer bases para ajustes na gamificação, visando um ambiente de aprendizado mais inclusivo e eficaz.
\end{resumo}