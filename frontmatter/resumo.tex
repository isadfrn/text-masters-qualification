% Formatação de Resumo - pg 267 do arquivo "univali_custom.sty"
\setlength{\absparsep}{14pt} 
\begin{resumo}
O Resumo é um dos componentes mais importantes do trabalho. É partir dele que o leitor irá decidir se vale a pena continuar lendo o trabalho ou não. O resumo deve ser escrito como um parágrafo único, sem utilizar referências bibliográficas e evitando ao máximo, o uso de siglas/abreviações. O resumo deve conter entre 200 e 400 palavras, sendo composto das seguintes partes (organização lógica): introdução, objetivos, justificativa, metodologia, resultados esperados ou obtidos. Esta é a seqüência lógica, não devendo ser utilizados títulos e subtítulos. Não abuse na contextualização, pois o foco deve ser nos objetivos, resultados esperados e resultados obtidos. Escreva o resumo apenas após a conclusão do trabalho. Ele deve refletir bem aquilo que foi desenvolvido.
\end{resumo}