\begin{resumo}
    A demanda por profissionais em cibersegurança cresce, mas a baixa participação feminina no setor permanece um problema. Diversos relatórios mostram essa disparidade em âmbito internacional e nacional. Dentre as ações criadas a fim de aumentar o número de profissionais capacitados no mercado brasileiro, destaca-se o programa \textit{Hackers} do Bem busca capacitar profissionais em larga escala, usando métodos de gamificação e rankings. Este estudo examina se a ênfase em competição e ranqueamento pode atuar como uma barreira involuntária à continuidade da formação de mulheres em programas deste tipo. O objetivo é propor um conjunto de diretrizes para melhorar a inclusão e a retenção, baseado em uma análise quantitativa e qualitativa da participação feminina, usando o programa \textit{Hackers} do Bem como estudo de caso. A metodologia utiliza métodos mistos: revisão sistemática da literatura, mineração de dados educacionais para calcular taxas de evasão e eficiência em cada etapa do curso, e questionários de autoeficácia para entender fatores subjetivos da desistência. Como resultados, espera-se identificar gargalos de retenção por gênero, correlacionar pontos de experiência com desempenho e fornecer bases para possíveis ajustes na gamificação.
\end{resumo}