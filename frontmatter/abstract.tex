\begin{abstract}
 \begin{otherlanguage*}{english}
    The growing demand for cybersecurity professionals and the establishment of the National Cybersecurity Strategy (E-Ciber) have driven the creation of large-scale training programs in Brazil. However, the persistent underrepresentation of women in the sector, stagnant at around 25\% globally, challenges the effectiveness of these initiatives in promoting diversity. This dissertation investigates the leaky pipeline phenomenon in the context of the Hackers do Bem Program, analysing whether competitive gamification and ranking mechanics, typical of massive training initiatives, act as barriers to retention for women. The research adopts a mixed-methods approach, combining educational data mining to map critical dropout points in the training funnel with qualitative analysis of participants' self-efficacy perceptions. As a result, the development and validation of a \textbf{Framework of Guidelines for Inclusion and Retention} is proposed. This artefact is structured around three strategic axes: Narrative Contextualization, Inclusive Assessment Mechanics, and Scalable Social Support, aiming to reconcile the technical efficiency of mass training with the pedagogical practices needed to broaden the entry and retention of female talent.
 \end{otherlanguage*}
\end{abstract}