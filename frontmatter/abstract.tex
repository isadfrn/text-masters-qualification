\begin{abstract}
 \begin{otherlanguage*}{english}
    The demand for cybersecurity professionals is growing, yet low female participation in the sector remains a persistent issue. The \textit{Hackers} do Bem Program aims to train professionals on a large scale, utilizing gamification methods and rankings. This study investigates whether these competitive mechanisms, without adequate adjustments, might exacerbate female attrition during training. The objective is to propose a set of guidelines to enhance inclusion and retention, based on a quantitative and qualitative analysis of female participation in the program. The research is justified by the need to evaluate whether the current model promotes gender equity, providing concrete metrics for massive educational interventions. The methodology employs a mixed-methods approach: a Systematic Literature Review to establish benchmarks, educational data mining to calculate attrition rates and efficiency at each course stage, and self-efficacy questionnaires to understand subjective factors behind dropout. Expected results include identifying gender-specific retention bottlenecks, correlating experience points with performance, and providing a basis for gamification adjustments, aiming for a more inclusive and effective learning environment.

\end{otherlanguage*}
\end{abstract}